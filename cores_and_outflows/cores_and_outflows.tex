\documentclass{aa}
%\documentclass[defaultstyle,11pt]{thesis}
%\documentclass[]{report}
%\documentclass[]{article}
%\usepackage{aastex_hack}
%\usepackage{deluxetable}
%\documentclass[preprint]{aastex}
%\documentclass{aa}
\newcommand\arcdeg{\mbox{$^\circ$}\xspace} 

\pdfminorversion=4


%%%%%%%%%%%%%%%%%%%%%%%%%%%%%%%%%%%%%%%%%%%%%%%%%%%%%%%%%%%%%%%%
%%%%%%%%%%%  see documentation for information about  %%%%%%%%%%
%%%%%%%%%%%  the options (11pt, defaultstyle, etc.)   %%%%%%%%%%
%%%%%%%  http://www.colorado.edu/its/docs/latex/thesis/  %%%%%%%
%%%%%%%%%%%%%%%%%%%%%%%%%%%%%%%%%%%%%%%%%%%%%%%%%%%%%%%%%%%%%%%%
%		\documentclass[typewriterstyle]{thesis}
% 		\documentclass[modernstyle]{thesis}
% 		\documentclass[modernstyle,11pt]{thesis}
%	 	\documentclass[modernstyle,12pt]{thesis}

%%%%%%%%%%%%%%%%%%%%%%%%%%%%%%%%%%%%%%%%%%%%%%%%%%%%%%%%%%%%%%%%
%%%%%%%%%%%    load any packages which are needed    %%%%%%%%%%%
%%%%%%%%%%%%%%%%%%%%%%%%%%%%%%%%%%%%%%%%%%%%%%%%%%%%%%%%%%%%%%%%
\usepackage{latexsym}		% to get LASY symbols
\usepackage{graphicx}		% to insert PostScript figures
%\usepackage{deluxetable}
\usepackage{rotating}		% for sideways tables/figures
\usepackage{natbib}  % Requires natbib.sty, available from http://ads.harvard.edu/pubs/bibtex/astronat/
\usepackage{savesym}
\usepackage{pdflscape}
\usepackage{amssymb}
\usepackage{morefloats}
%\savesymbol{singlespace}
\savesymbol{doublespace}
%\usepackage{wrapfig}
%\usepackage{setspace}
\usepackage{xspace}
\usepackage{color}
\usepackage{multicol}
\usepackage{mdframed}
\usepackage{url}
\usepackage{subfigure}
%\usepackage{emulateapj}
\usepackage{lscape}
\usepackage{grffile}
\usepackage{standalone}
\standalonetrue
\usepackage{import}
\usepackage[utf8]{inputenc}
\usepackage{longtable}
\usepackage{booktabs}
\usepackage[yyyymmdd,hhmmss]{datetime}
\usepackage{fancyhdr}
\usepackage[colorlinks=true,citecolor=blue,linkcolor=cyan]{hyperref}
\usepackage{ifpdf}






%\renewcommand\ion[2]{#1$\;${%
%\ifx\@currsize\normalsize\small \else
%\ifx\@currsize\small\footnotesize \else
%\ifx\@currsize\footnotesize\scriptsize \else
%\ifx\@currsize\scriptsize\tiny \else
%\ifx\@currsize\large\normalsize \else
%\ifx\@currsize\Large\large
%\fi\fi\fi\fi\fi\fi
%\rmfamily\@Roman{#2}}\relax}% 

\newcommand{\paa}{Pa\ensuremath{\alpha}}
\newcommand{\brg}{Br\ensuremath{\gamma}}
\newcommand{\msun}{\ensuremath{M_{\odot}}\xspace}			%  Msun
\newcommand{\mdot}{\ensuremath{\dot{M}}\xspace}
\newcommand{\lsun}{\ensuremath{L_{\odot}}\xspace}			%  Lsun
\newcommand{\rsun}{\ensuremath{R_{\odot}}\xspace}			%  Rsun
\newcommand{\lbol}{\ensuremath{L_{\mathrm{bol}}\xspace}}	%  Lbol
\newcommand{\ks}{K\ensuremath{_{\mathrm{s}}}}		%  Ks
\newcommand{\hh}{\ensuremath{\textrm{H}_{2}}\xspace}			%  H2
\newcommand{\dens}{\ensuremath{n(\hh) [\percc]}\xspace}
\newcommand{\formaldehyde}{\ensuremath{\textrm{H}_2\textrm{CO}}\xspace}
\newcommand{\formamide}{\ensuremath{\textrm{NH}_2\textrm{CHO}}\xspace}
\newcommand{\formaldehydeIso}{\ensuremath{\textrm{H}_2~^{13}\textrm{CO}}\xspace}
\newcommand{\methanol}{\ensuremath{\textrm{CH}_3\textrm{OH}}\xspace}
\newcommand{\ortho}{\ensuremath{\textrm{o-H}_2\textrm{CO}}\xspace}
\newcommand{\para}{\ensuremath{\textrm{p-H}_2\textrm{CO}}\xspace}
\newcommand{\oneone}{\ensuremath{1_{1,0}-1_{1,1}}\xspace}
\newcommand{\twotwo}{\ensuremath{2_{1,1}-2_{1,2}}\xspace}
\newcommand{\threethree}{\ensuremath{3_{1,2}-3_{1,3}}\xspace}
\newcommand{\threeohthree}{\ensuremath{3_{0,3}-2_{0,2}}\xspace}
\newcommand{\threetwotwo}{\ensuremath{3_{2,2}-2_{2,1}}\xspace}
\newcommand{\threetwoone}{\ensuremath{3_{2,1}-2_{2,0}}\xspace}
\newcommand{\fourtwotwo}{\ensuremath{4_{2,2}-3_{1,2}}\xspace} % CH3OH 218.4 GHz
\newcommand{\methylcyanide}{\ensuremath{\textrm{CH}_{3}\textrm{CN}}\xspace}
\newcommand{\ketene}{\ensuremath{\textrm{H}_{2}\textrm{CCO}}\xspace}
\newcommand{\ethylcyanide}{\ensuremath{\textrm{CH}_3\textrm{CH}_2\textrm{CN}}\xspace}
\newcommand{\cyanoacetylene}{\ensuremath{\textrm{HC}_{3}\textrm{N}}\xspace}
\newcommand{\methylformate}{\ensuremath{\textrm{CH}_{3}\textrm{OCHO}}\xspace}
\newcommand{\dimethylether}{\ensuremath{\textrm{CH}_{3}\textrm{OCH}_{3}}\xspace}
\newcommand{\gaucheethanol}{\ensuremath{\textrm{g-CH}_3\textrm{CH}_2\textrm{OH}}\xspace}
\newcommand{\acetone}{\ensuremath{\left[\textrm{CH}_{3}\right]_2\textrm{CO}}\xspace}
\newcommand{\methyleneamidogen}{\ensuremath{\textrm{H}_{2}\textrm{CN}}\xspace}
\newcommand{\Rone}{\ensuremath{\para~S_{\nu}(\threetwoone) / S_{\nu}(\threeohthree)}\xspace}
\newcommand{\Rtwo}{\ensuremath{\para~S_{\nu}(\threetwotwo) / S_{\nu}(\threetwoone)}\xspace}
\newcommand{\JKaKc}{\ensuremath{J_{K_a K_c}}}
\newcommand{\water}{H$_{2}$O\xspace}		%  H2O
\newcommand{\feii}{\ion{Fe}{2}}		%  FeII
\newcommand{\uchii}{\ion{UCH}{ii}\xspace}
\newcommand{\UCHII}{\ion{UCH}{ii}\xspace}
\newcommand{\hchii}{\ion{HCH}{ii}\xspace}
\newcommand{\HCHII}{\ion{HCH}{ii}\xspace}
\newcommand{\hii}{\ion{H}{ii}\xspace}
\newcommand{\hi}{H~{\sc i}\xspace}
\newcommand{\Hii}{\hii}
\newcommand{\HII}{\hii}
\newcommand{\Xform}{\ensuremath{X_{\formaldehyde}}}
\newcommand{\kms}{\textrm{km~s}\ensuremath{^{-1}}\xspace}	%  km s-1
\newcommand{\nsample}{456\xspace}
\newcommand{\CFR}{5\xspace} % nMPC / 0.25 / 2 (6 for W51 once, 8 for W51 twice) REFEDIT: With f_observed=0.3, becomes 3/2./0.3 = 5
\newcommand{\permyr}{\ensuremath{\mathrm{Myr}^{-1}}\xspace}
\newcommand{\pers}{\ensuremath{\mathrm{s}^{-1}}\xspace}
\newcommand{\tsuplim}{0.5\xspace} % upper limit on starless timescale
\newcommand{\ncandidates}{18\xspace}
\newcommand{\mindist}{8.7\xspace}
\newcommand{\rcluster}{2.5\xspace}
\newcommand{\ncomplete}{13\xspace}
\newcommand{\middistcut}{13.0\xspace}
\newcommand{\nMPC}{3\xspace} % only count W51 once.  W51, W49, G010
\newcommand{\obsfrac}{30}
\newcommand{\nMPCtot}{10\xspace} % = nmpc / obsfrac
\newcommand{\nMPCtoterr}{6\xspace} % = sqrt(nmpc) / obsfrac
\newcommand{\plaw}{2.1\xspace}
\newcommand{\plawerr}{0.3\xspace}
\newcommand{\mmin}{\ensuremath{10^4~\msun}\xspace}
%\newcommand{\perkmspc}{\textrm{per~km~s}\ensuremath{^{-1}}\textrm{pc}\ensuremath{^{-1}}\xspace}	%  km s-1 pc-1
\newcommand{\kmspc}{\textrm{km~s}\ensuremath{^{-1}}\textrm{pc}\ensuremath{^{-1}}\xspace}	%  km s-1 pc-1
\newcommand{\sqcm}{cm$^{2}$\xspace}		%  cm^2
\newcommand{\percc}{\ensuremath{\textrm{cm}^{-3}}\xspace}
\newcommand{\perpc}{\ensuremath{\textrm{pc}^{-1}}\xspace}
\newcommand{\persc}{\ensuremath{\textrm{cm}^{-2}}\xspace}
\newcommand{\persr}{\ensuremath{\textrm{sr}^{-1}}\xspace}
\newcommand{\peryr}{\ensuremath{\textrm{yr}^{-1}}\xspace}
\newcommand{\perkmspc}{\textrm{km~s}\ensuremath{^{-1}}\textrm{pc}\ensuremath{^{-1}}\xspace}	%  km s-1 pc-1
\newcommand{\perkms}{\textrm{per~km~s}\ensuremath{^{-1}}\xspace}	%  km s-1 
\newcommand{\um}{\ensuremath{\mu \textrm{m}}\xspace}    % micron
\newcommand{\microjy}{\ensuremath{\mu\textrm{Jy}}\xspace}    % micron
\newcommand{\mum}{\um}
\newcommand{\htwo}{\ensuremath{\textrm{H}_2}}
\newcommand{\Htwo}{\ensuremath{\textrm{H}_2}}
\newcommand{\HtwoO}{\ensuremath{\textrm{H}_2\textrm{O}}}
\newcommand{\htwoo}{\ensuremath{\textrm{H}_2\textrm{O}}}
\newcommand{\ha}{\ensuremath{\textrm{H}\alpha}}
\newcommand{\hb}{\ensuremath{\textrm{H}\beta}}
\newcommand{\so}{SO~\ensuremath{5_6-4_5}\xspace}
\newcommand{\SO}{SO~\ensuremath{1_2-1_1}\xspace}
\newcommand{\ammonia}{NH\ensuremath{_3}\xspace}
\newcommand{\twelveco}{\ensuremath{^{12}\textrm{CO}}\xspace}
\newcommand{\thirteenco}{\ensuremath{^{13}\textrm{CO}}\xspace}
\newcommand{\ceighteeno}{\ensuremath{\textrm{C}^{18}\textrm{O}}\xspace}
\def\ee#1{\ensuremath{\times10^{#1}}}
\newcommand{\degrees}{\ensuremath{^{\circ}}}
% can't have \degree because I'm getting a degree...
\newcommand{\lowirac}{800}
\newcommand{\highirac}{8000}
\newcommand{\lowmips}{600}
\newcommand{\highmips}{5000}
\newcommand{\perbeam}{\ensuremath{\textrm{beam}^{-1}}}
\newcommand{\ds}{\ensuremath{\textrm{d}s}}
\newcommand{\dnu}{\ensuremath{\textrm{d}\nu}}
\newcommand{\dv}{\ensuremath{\textrm{d}v}}
\def\secref#1{Section \ref{#1}}
\def\eqref#1{Equation \ref{#1}}
\def\facility#1{#1}
%\newcommand{\arcmin}{'}

\newcommand{\necluster}{Sh~2-233IR~NE}
\newcommand{\swcluster}{Sh~2-233IR~SW}
\newcommand{\region}{IRAS 05358}

\newcommand{\nwfive}{40}
\newcommand{\nouter}{15}

\newcommand{\vone}{{\rm v}1.0\xspace}
\newcommand{\vtwo}{{\rm v}2.0\xspace}
\newcommand\mjysr{\ensuremath{{\rm MJy~sr}^{-1}}}
\newcommand\jybm{\ensuremath{{\rm Jy~bm}^{-1}}}
\newcommand\nbolocat{8552\xspace}
\newcommand\nbolocatnew{548\xspace}
\newcommand\nbolocatnonew{8004\xspace} % = nbolocat-nbolocatnew
\renewcommand\arcdeg{\mbox{$^\circ$}\xspace} 
\renewcommand\arcmin{\mbox{$^\prime$}\xspace} 
\renewcommand\arcsec{\mbox{$^{\prime\prime}$}\xspace} 

\newcommand{\todo}[1]{\textcolor{red}{#1}}
\newcommand{\okinfinal}[1]{{#1}}
%% only needed if not aastex
%\newcommand{\keywords}[1]{}
%\newcommand{\email}[1]{}
%\newcommand{\affil}[1]{}


%aastex hack
%\newcommand\arcdeg{\mbox{$^\circ$}}%
%\newcommand\arcmin{\mbox{$^\prime$}\xspace}%
%\newcommand\arcsec{\mbox{$^{\prime\prime}$}\xspace}%

%\newcommand\epsscale[1]{\gdef\eps@scaling{#1}}
%
%\newcommand\plotone[1]{%
% \typeout{Plotone included the file #1}
% \centering
% \leavevmode
% \includegraphics[width={\eps@scaling\columnwidth}]{#1}%
%}%
%\newcommand\plottwo[2]{{%
% \typeout{Plottwo included the files #1 #2}
% \centering
% \leavevmode
% \columnwidth=.45\columnwidth
% \includegraphics[width={\eps@scaling\columnwidth}]{#1}%
% \hfil
% \includegraphics[width={\eps@scaling\columnwidth}]{#2}%
%}}%


%\newcommand\farcm{\mbox{$.\mkern-4mu^\prime$}}%
%\let\farcm\farcm
%\newcommand\farcs{\mbox{$.\!\!^{\prime\prime}$}}%
%\let\farcs\farcs
%\newcommand\fp{\mbox{$.\!\!^{\scriptscriptstyle\mathrm p}$}}%
%\newcommand\micron{\mbox{$\mu$m}}%
%\def\farcm{%
% \mbox{.\kern -0.7ex\raisebox{.9ex}{\scriptsize$\prime$}}%
%}%
%\def\farcs{%
% \mbox{%
%  \kern  0.13ex.%
%  \kern -0.95ex\raisebox{.9ex}{\scriptsize$\prime\prime$}%
%  \kern -0.1ex%
% }%
%}%

\def\Figure#1#2#3#4#5{
\begin{figure*}[!htp]
\includegraphics[scale=#4,width=#5]{#1}
\caption{#2}
\label{#3}
\end{figure*}
}

\def\WrapFigure#1#2#3#4#5#6{
\begin{wrapfigure}{#6}{0.5\textwidth}
\includegraphics[scale=#4,width=#5]{#1}
\caption{#2}
\label{#3}
\end{wrapfigure}
}

% % #1 - filename
% % #2 - caption
% % #3 - label
% % #4 - epsscale
% % #5 - R or L?
% \def\WrapFigure#1#2#3#4#5#6{
% \begin{wrapfigure}[#6]{#5}{0.45\textwidth}
% %  \centercaption
% %  \vspace{-14pt}
%   \epsscale{#4}
%   \includegraphics[scale=#4]{#1}
%   \caption{#2}
%   \label{#3}
% \end{wrapfigure}
% }

\def\RotFigure#1#2#3#4#5{
\begin{sidewaysfigure*}[!htp]
\includegraphics[scale=#4,width=#5]{#1}
\caption{#2}
\label{#3}
\end{sidewaysfigure*}
}

\def\FigureSVG#1#2#3#4{
\begin{figure*}[!htp]
    \def\svgwidth{#4}
    \input{#1}
    \caption{#2}
    \label{#3}
\end{figure*}
}

% originally intended to be included in a two-column paper
% this is in includegraphics: ,width=3in
% but, not for thesis
\def\OneColFigure#1#2#3#4#5{
\begin{figure}[!htpb]
\epsscale{#4}
\includegraphics[scale=#4,angle=#5]{#1}
\caption{#2}
\label{#3}
\end{figure}
}

\def\SubFigure#1#2#3#4#5{
\begin{figure*}[!htp]
\addtocounter{figure}{-1}
\epsscale{#4}
\includegraphics[angle=#5]{#1}
\caption{#2}
\label{#3}
\end{figure*}
}

%\def\FigureTwo#1#2#3#4#5{
%\begin{figure*}[!htp]
%\epsscale{#5}
%\plottwo{#1}{#2}
%\caption{#3}
%\label{#4}
%\end{figure*}
%}

\def\FigureTwo#1#2#3#4#5#6{
\begin{figure*}[!htp]
\subfigure[]{ \includegraphics[scale=#5,width=#6]{#1} }
\subfigure[]{ \includegraphics[scale=#5,width=#6]{#2} }
\caption{#3}
\label{#4}
\end{figure*}
}

\def\FigureTwoAA#1#2#3#4#5#6{
\begin{figure*}[!htp]
\subfigure[]{ \includegraphics[scale=#5,width=#6]{#1} }
\subfigure[]{ \includegraphics[scale=#5,width=#6]{#2} }
\caption{#3}
\label{#4}
\end{figure*}
}

\newenvironment{rotatepage}%
{}{}
   %{\pagebreak[4]\afterpage\global\pdfpageattr\expandafter{\the\pdfpageattr/Rotate 90}}%
   %{\pagebreak[4]\afterpage\global\pdfpageattr\expandafter{\the\pdfpageattr/Rotate 0}}%


\def\RotFigureTwoAA#1#2#3#4#5#6{
\begin{rotatepage}
\begin{sidewaysfigure*}[!htp]
\subfigure[]{ \includegraphics[scale=#5,width=#6]{#1} }
\\
\subfigure[]{ \includegraphics[scale=#5,width=#6]{#2} }
\caption{#3}
\label{#4}
\end{sidewaysfigure*}
\end{rotatepage}
}

\def\RotFigureThreeAA#1#2#3#4#5#6#7{
\begin{rotatepage}
\begin{sidewaysfigure*}[!htp]
\subfigure[]{ \includegraphics[scale=#6,width=#7]{#1} }
\\
\subfigure[]{ \includegraphics[scale=#6,width=#7]{#2} }
\\
\subfigure[]{ \includegraphics[scale=#6,width=#7]{#3} }
\caption{#4}
\label{#5}
\end{sidewaysfigure*}
\end{rotatepage}
\clearpage
}

\def\FigureThreeAA#1#2#3#4#5#6#7{
\begin{figure*}[!htp]
\subfigure[]{ \includegraphics[scale=#6,width=#7]{#1} }
\subfigure[]{ \includegraphics[scale=#6,width=#7]{#2} }
\subfigure[]{ \includegraphics[scale=#6,width=#7]{#3} }
\caption{#4}
\label{#5}
\end{figure*}
}



\def\SubFigureTwo#1#2#3#4#5{
\begin{figure*}[!htp]
\addtocounter{figure}{-1}
\epsscale{#5}
\plottwo{#1}{#2}
\caption{#3}
\label{#4}
\end{figure*}
}

\def\FigureFour#1#2#3#4#5#6{
\begin{figure*}[!htp]
\subfigure[]{ \includegraphics[width=3in]{#1} }
\subfigure[]{ \includegraphics[width=3in]{#2} }
\subfigure[]{ \includegraphics[width=3in]{#3} }
\subfigure[]{ \includegraphics[width=3in]{#4} }
\caption{#5}
\label{#6}
\end{figure*}
}

\def\FigureFourPDF#1#2#3#4#5#6{
\begin{figure*}[!htp]
\subfigure[]{ \includegraphics[width=3in,type=pdf,ext=.pdf,read=.pdf]{#1} }
\subfigure[]{ \includegraphics[width=3in,type=pdf,ext=.pdf,read=.pdf]{#2} }
\subfigure[]{ \includegraphics[width=3in,type=pdf,ext=.pdf,read=.pdf]{#3} }
\subfigure[]{ \includegraphics[width=3in,type=pdf,ext=.pdf,read=.pdf]{#4} }
\caption{#5}
\label{#6}
\end{figure*}
}

\def\FigureThreePDF#1#2#3#4#5{
\begin{figure*}[!htp]
\subfigure[]{ \includegraphics[width=3in,type=pdf,ext=.pdf,read=.pdf]{#1} }
\subfigure[]{ \includegraphics[width=3in,type=pdf,ext=.pdf,read=.pdf]{#2} }
\subfigure[]{ \includegraphics[width=3in,type=pdf,ext=.pdf,read=.pdf]{#3} }
\caption{#4}
\label{#5}
\end{figure*}
}

\def\Table#1#2#3#4#5{
%\renewcommand{\thefootnote}{\alph{footnote}}
\begin{table}
\caption{#2}
\label{#3}
    \begin{tabular}{#1}
        \hline\hline
        #4
        \hline
        #5
        \hline
    \end{tabular}
\end{table}
%\renewcommand{\thefootnote}{\arabic{footnote}}
}


%\def\Table#1#2#3#4#5#6{
%%\renewcommand{\thefootnote}{\alph{footnote}}
%\begin{deluxetable}{#1}
%\tablewidth{0pt}
%\tabletypesize{\footnotesize}
%\tablecaption{#2}
%\tablehead{#3}
%\startdata
%\label{#4}
%#5
%\enddata
%\bigskip
%#6
%\end{deluxetable}
%%\renewcommand{\thefootnote}{\arabic{footnote}}
%}

%\def\tablenotetext#1#2{
%\footnotetext[#1]{#2}
%}

\def\LongTable#1#2#3#4#5#6#7#8{
% required to get tablenotemark to work: http://www2.astro.psu.edu/users/stark/research/psuthesis/longtable.html
\renewcommand{\thefootnote}{\alph{footnote}}
\begin{longtable}{#1}
\caption[#2]{#2}
\label{#4} \\

 \\
\hline 
#3 \\
\hline
\endfirsthead

\hline
#3 \\
\hline
\endhead

\hline
\multicolumn{#8}{r}{{Continued on next page}} \\
\hline
\endfoot

\hline 
\endlastfoot
#7 \\

#5
\hline
#6 \\

\end{longtable}
\renewcommand{\thefootnote}{\arabic{footnote}}
}

\def\TallFigureTwo#1#2#3#4#5#6{
\begin{figure*}[htp]
\epsscale{#5}
\subfigure[]{ \includegraphics[width=#6]{#1} }
\subfigure[]{ \includegraphics[width=#6]{#2} }
\caption{#3}
\label{#4}
\end{figure*}
}

		% file containing author's macro definitions

\begin{document}
%\title{Cores and Outflows and Chemistry in the W51 Protoclusters}
\title{Thermal Feedback in the high-mass star and cluster forming region W51}

\begin{abstract}
    Massive stars form primarily in high-density regions ...
\end{abstract}

\section{Paper 1 to-do list}
\begin{enumerate}
    \item Make a photometry table - one from the hand-extracted, one from
        dendrogram-extracted?  Section \ref{sec:photometry}
    \item Is any discussion of the gas kinematics warranted?  That should
        perhaps be relegated to the (lack of?) disks paper
    \item create Table ref{tab:linelist} = \ref{tab:linelist} (Section
        \ref{sec:temperature})
    \item Show a `core' mass histogram in Section \ref{sec:temperature}
    \item Repeat Section \ref{sec:W51e2e} for e8 and north
    \item move Section \ref{sec:cheminterp} to discussion
    \item Add a section ``building on Ginsburg+ 2016''
    \item (maybe for the other paper): detailed SED study of e2w.  What is its
        luminosity?  spectral type?  What about e2e?  the latter is more appropriate
        for this paper
    \item for the large scales, where is most of the luminosity?  What regions
        generate most of the energy as perceived from $>$100pc distance?  I suspect
        it is the IRS1 region, which has no star formation and maybe no stars...

\end{enumerate}

\section{Introduction}
We introduce...

\section{Observations}
As part of ALMA Cycle 2 program 2013.1.00308.S, we observed a
$\sim2\arcmin\times1\arcmin$ region centered between W51 IRS2 and W51 e1/e2
with a 37-pointing mosaic.  Two configurations of the 12m array were used,
achieving a resolution of 0.2\arcsec.  Additionally, a 12-pointing mosaic was
performed using the 7m array, hypothetically probing scales up to
$\sim28$\arcsec.  The full UV coverage was from $\sim12$ to $\sim1500$ m
(figure \ref{fig:uvcov}).

% figure produced directly in CASA; not included in repository b/c
% it requries having access to the full data set
\Figure{figures/visibility_weight_vs_uvdist_linear.png}
{A weighted histogram of the visibility weights as a function of UV distance;
this approximately shows the amount of data received at each baseline length.}
{fig:uvcov}{1}{16cm}

\begin{table*}[htp]
\caption{Spectral Setup}
\begin{tabular}{llll}
\label{tab:spw}
SpwID & Minimum Frequency & Maximum Frequency & Channel Width \\
 & $\mathrm{GHz}$ & $\mathrm{GHz}$ & $\mathrm{kHz}$ \\
\hline
0 & 218.11930228 & 218.619301 & -122.07 \\
1 & 218.36288652 & 220.355073 & -488.281 \\
2 & 230.376575 & 232.36876148 & 488.281 \\
3 & 232.981075 & 234.97326148 & 488.281 \\
\hline
\end{tabular}
\end{table*}


\subsection{Data Reduction}
Data reduction was performed using CASA.  The QA2-produced data products were
combined using the standard inverse variance weighting.  Two sets of images
were produced for different aspects of the analysis, one including the 7m array
data and one including only 12m data.  Except where otherwise noted, the
12m-only data were used in order to focus on the compact structures.


\subsubsection{Continuum}
A continuum image combining all 4 spectral windows was produced using
\texttt{tclean}.  We phase self-calibrated the data on baselines longer than
100m to increase the dynamic range.  The final image was cleaned with 50000
iterations to a threshold of 5 mJy.  The lowest noise level in the image, away
from bright sources, is $\sim0.2$ mJy/beam, but near the bright sources e2 and
IRS2, the noise reached as high as $\sim2$ mJy/beam.  Deeper cleaning was
attempted, but lead to instabilities.


\subsubsection{Lines}
We produced spectral image cubes of the lines listed in Tables
\ref{tab:linesspw0},
\ref{tab:linesspw1},
\ref{tab:linesspw2}, and
\ref{tab:linesspw3}.

\begin{table*}[htp]
\caption{Spectral Lines in SPW 0}
\begin{tabular}{ll}
\label{tab:linesspw0}
Line Name & Frequency \\
 & $\mathrm{GHz}$ \\
\hline
H$_2$CO $3_{0,3}-2_{0,2}$ & 218.22219 \\
H$_2$CO $3_{2,2}-2_{2,1}$ & 218.47564 \\
E-CH$_3$OH $4_{2,2}-3_{1,2}$ & 218.44005 \\
CH$_3$OCHO $17_{3,14}-16_{3,13}$E & 218.28083 \\
CH$_3$OCHO $17_{3,14}-16_{3,13}$A & 218.29787 \\
CH$_3$CH$_2$CN $24_{3,21}-23_{3,20}$ & 218.39002 \\
Acetone $8_{7,1}-7_{4,4}$AE & 218.24017 \\
O$^{13}$CS 18-17 & 218.19898 \\
CH$_3$OCH$_3$ $23_{3,21}-23_{2,22}$AA & 218.49441 \\
CH$_3$OCH$_3$ $23_{3,21}-23_{2,22}$EE & 218.49192 \\
CH$_3$NCO $25_{1,24} - 24_{1,23}$ & 218.5418 \\
CH$_3$SH $23_2-23_1$ & 218.18612 \\
\hline
\end{tabular}

\end{table*}


\begin{table*}[htp]
\caption{Spectral Lines in SPW 1}
\begin{tabular}{ll}
\label{tab:linesspw1}
Line Name & Frequency \\
 & $\mathrm{GHz}$ \\
\hline
H$_2$CO $3_{2,1}-2_{2,0}$ & 218.76007 \\
HC$_3$N 24-23 & 218.32471 \\
HC$_3$Nv$_7$=1 24-23a & 219.17358 \\
HC$_3$Nv$_7$=1 24-23a & 218.86063 \\
HC$_3$Nv$_7$=2 24-23 & 219.67465 \\
OCS 18-17 & 218.90336 \\
SO $6_5-5_4$ & 219.94944 \\
HNCO $10_{1,10}-9_{1,9}$ & 218.98102 \\
HNCO $10_{2,8}-9_{2,7}$ & 219.73719 \\
HNCO $10_{0,10}-9_{0,9}$ & 219.79828 \\
HNCO $10_{5,5}-9_{5,4}$ & 219.39241 \\
HNCO $10_{4,6}-9_{4,5}$ & 219.54708 \\
HNCO $10_{3,8}-9_{3,7}$ & 219.65677 \\
CH$_3$OH $8_{0,8}-7_{1,6}$ & 220.07849 \\
CH$_3$OH $25_{3,22}-24_{4,20}$ & 219.98399 \\
CH$_3$OH $23_{5,19}-22_{6,17}$ & 219.99394 \\
C$^{18}$O 2-1 & 219.56036 \\
H$_2$CCO 11-10 & 220.17742 \\
HCOOH $4_{3,1}-5_{2,4}$ & 219.09858 \\
CH$_3$OCHO $17_{4,13}-16_{4,12}$A & 220.19027 \\
CH$_3$CH$_2$CN $24_{2,22}-23_{2,21}$ & 219.50559 \\
Acetone $21_{1,20}-20_{2,19}$AE & 219.21993 \\
Acetone $21_{1,20}-20_{1,19}$EE & 219.24214 \\
Acetone $12_{9,4}-11_{8,3}$EE & 218.63385 \\
H$_2$$^{13}$CO $3_{1,2}-2_{1,1}$ & 219.90849 \\
SO$_2$ $22_{7,15}-23_{6,18}$ & 219.27594 \\
SO$_2$ $v_2=1$ $20_{2,18}-19_{3,17}$ & 218.99583 \\
SO$_2$ $v_2=1$ $22_{2,20}-22_{1,21}$ & 219.46555 \\
SO$_2$ $v_2=1$ $16_{3,13}-16_{2,14}$ & 220.16524 \\
\hline
\end{tabular}
\par
The Categories column consists of three letter codes as described in Section \ref{sec:contsourcenature}.In column 1, \texttt{F} indicates a free-free dominated source,\texttt{f} indicates significant free-free contribution,and \texttt{-} means there is no detected cm continuum.In column 2, the peak brightness temperature is used toclassify the temperature category.\texttt{H} is `hot' ($T>50$ K), \texttt{C} is `cold' ($T<20$ K), and \texttt{-} is indeterminate (either $20<T<50$K or no measurement)In column 3, \texttt{c} indicates compact sources, and \texttt{-} indicates a diffuse source.
\end{table*}


\begin{table*}[htp]
\caption{Spectral Lines in SPW 2}
\begin{tabular}{ll}
\label{tab:linesspw2}
Line Name & Frequency \\
 & $\mathrm{GHz}$ \\
\hline
$^{12}$CO $2-1$ & 230.538 \\
OCS 19-18 & 231.06099 \\
HNCO $28_{1,28}-29_{0,29}$ & 231.873255 \\
CH$_3$OH $10_{2,9}-9_{3,6}$ & 231.28115 \\
$^{13}$CS 5-4 & 231.22069 \\
NH$_2$CHO $11_{2,10}-10_{2,9}$ & 232.27363 \\
H30$\alpha$ & 231.90093 \\
CH$_3$OCHO $12_{4,9}-11_{3,8}$E & 231.01908 \\
CH$_3$CH$_2$OH $5_{5,0}-5_{4,1}$ & 231.02517 \\
CH$_3$OCH$_3$ $13_{0,13}-12_{1,12}$AA & 231.98772 \\
N$_2$D+ 3-2 & 231.32183 \\
g-CH$_3$CH$_2$OH $13_{2,11}-12_{2,10}$ & 230.67255 \\
g-CH$_3$CH$_2$OH $6_{5,1}-5_{4,1}$ & 230.79351 \\
g-CH$_3$CH$_2$OH $16_{5,11}-16_{4,12}$ & 230.95379 \\
g-CH$_3$CH$_2$OH $14_{0,14}-13_{1,13}$ & 230.99138 \\
SO$_2$ $v_2=1$ $6_{4,2}-7_{3,5}$ & 232.21031 \\
CH$_3$SH $16_2-16_1$ & 231.75891 \\
CH$_3$SH $7_3-8_2$ & 230.64608 \\
\hline
\end{tabular}
\par
The Categories column consists of three letter codes as described in Section \ref{sec:contsourcenature}.In column 1, \texttt{F} indicates a free-free dominated source,\texttt{f} indicates significant free-free contribution,and \texttt{-} means there is no detected cm continuum.In column 2, the peak brightness temperature is used toclassify the temperature category.\texttt{H} is `hot' ($T>50$ K), \texttt{C} is `cold' ($T<20$ K), and \texttt{-} is indeterminate (either $20<T<50$K or no measurement)In column 3, \texttt{c} indicates compact sources, and \texttt{-} indicates a diffuse source.
\end{table*}


\begin{table*}[htp]
\caption{Spectral Lines in SPW 3}
\begin{tabular}{ll}
\label{tab:linesspw3}
Line Name & Frequency \\
 & $\mathrm{GHz}$ \\
\hline
A-CH$_3$OH $4_{2,3}-5_{1,4}$ & 234.68345 \\
E-CH$_3$OH $5_{-4,2}-6_{-3,4}$ & 234.69847 \\
A-CH$_3$OH $18_{3,15}-17_{4,14}$ & 233.7958 \\
$^{13}$CH$_3$OH $5_{1,5}-4_{1,4}$ & 234.01158 \\
PN $5-4$ & 234.93569 \\
NH$_2$CHO $11_{5,6}-10_{5,5}$ & 233.59451 \\
Acetone $12_{11,2}-11_{10,1}$AE & 234.86136 \\
SO$_2$ $16_{6,10}-17_{5,13}$ & 234.42159 \\
CH$_3$NCO $27_{2,26} - 26_{2,25}$ & 234.08812 \\
CH$_3$SH $15_2-15_1$ & 234.19145 \\
\hline
\end{tabular}

\end{table*}





% \subsection{Continuum Morphological Analysis}
% \label{sec:morphology}
% The largest detected structures include the W51 Main HII region bubble and the
% W51 IRS2 HII region, which are relatively uninteresting since their properties
% have been previously well-characterized using radio (JVLA) data.  More exciting
% are the bright dusty structures, especially the ``tail'' pointing south of W51
% e8, which can be described as a 0.25 pc by 0.05 pc filament. This structure has
% a very high surface brightness along its ridge, exceeding 40 mJy/beam in our TE
% maps (23 K or 3.7\ee{4} MJy sr$^{-1}$).  This high brightness implies a high
% intrinsic temperature, $T>30$ K (Section \ref{sec:temperature}).
% 
% This narrow filament is most prominent in the continuum. It is evident
% in some lines (\formaldehyde, $^{13}$CS, \ceighteeno), but not others (SO, ).
% It is surrounded by molecular emission that is only slightly fainter...
% ...in SO it's pretty uniform brightness...

% \subsection{Simulations}
% {\bf Unfortunately CASA's simulations produce reproducible incorrectness, in
% that I cannot get an image that matches the input image.  There seems to
% *always* be some flux scaling no matter what input unit is used.  Therefore, I
% don't trust any of the results of CASA simulations yet.  Perseus looks OK, but
% Aquila is just flat out wrong, and the scale-recovery simulations I attempted
% also failed.
% 
% Followup on the above paragraph: I've gotten the simulations worked out; CASA
% always treats data as if they are in Jy/beam even if Jy/pixel units are
% specified when using the \texttt{sm.predict} module.  However, I'm not
% convinced of their utility at this stage.}
% 
% The enhanced noise around bright sources is unavoidable.  We tested the noise
% properties of our data set using the CASA \texttt{simobserve} toolkit.  We
% obtained a Herschel Gould's Belt Survey image of the Perseus molecular cloud at
% 250 \um \citep[resolution 18\arcsec][]{} and scaled it down by $\sim40\times$
% to match the resolution of our ALMA data.  We used the \texttt{sm.predict} task
% to ``observe'' the Herschel data with our exact UV data set.  We then used
% \texttt{sm.corrupt} to make the noise properties approximately match those of
% our observations.
% 
% We used the Perseus data scaled from 250 \um to 1.3 mm assuming
% a relatively shallow $\beta=1.5$ and a constant temperature $T=20$ K
% (since temperature maps are not available).
% For the Aquila data, we used the column density and temperature maps to
% derive a synthetic 1.3 mm map assuming an opacity $\kappa_{505 \mathrm{GHz}} =
% 4$ g \percc.  Since Aquila is at a greater distance, the Herschel resolution is
% coarser (0.9 \arcsec at 5.4 kpc) than our best resolution of 0.2\arcsec, so it
% is best compared to lower-resolution tapered data.
% 
% 
% When imaging the Perseus data set, we put NGC 1333 at the image center.  At the
% noise levels in our data, only the central portions of NGC 1333 are detected,
% with three point sources recovered.  In this map, the noise properties are
% very uniform.  We are therefore unable to analyze the NGC
% 1333 data with the exact same parameters as were used on W51.  However, using
% similar parameters (but with a lower significance threshold), we detect only
% three sources.
% 
% In a second experiment, we scaled the peak flux density of the Perseus map to
% be $\sim100\times$ brighter than it should be, making it comparable to the flux
% density of W51e2 in the real ALMA observations.  In this map, even with deep
% cleaning, the noise around the bright sources remains very high and artifacts
% are evident.



\section{Results \& Analysis}
\subsection{Source Identification}
% dendrogramming.py
We used the \texttt{dendrogram} method described by \citet{Rosolowsky2008c} and
implemented in \texttt{astrodendro} to identify sources.  We used a minimum
value of 1 mJy/beam ($\sim5-\sigma$) and a minimum $\Delta=0.4$ mJy/beam
($\sim2-\sigma$) with minimum 10 pixels (each pixel is 0.05\arcsec).  This
cataloging yielded over 8000 candidate sources, of which the majority are noise
or artifacts around the brightest sources.  To filter out these bad sources,
we created a noise map taking the local RMS of the \texttt{tclean}-produced
residual map over a $\sigma=30$ pixel (1.5\arcsec) gaussian.  We then removed
all sources with peak S/N < 8, mean S/N per pixel $< 5$, and minimum S/N per
pixel $ < 1$.  We also only included the smallest sources in the dendrogram,
the ``leaves''.  These parameters were tuned by checking against ``real''
sources identified by eye and selected using \texttt{ds9}: most real sources are
recovered and few spurious sources ($<10$) are
included.  The resulting catalog includes 113 sources.

The `by-eye' core extraction approach, in which we placed ds9 regions on all
sources that look `real', produced a more reliable but less complete (and less
quantifiable) catalog containing 75 sources.  This catalog is more useful in
the regions around the bright sources e2 and north, since these regions are
affected by substantial uncleaned PSF sidelobe artifacts.  In particular, the
dendrogram catalog includes a number of sources around e2/e8 that, by eye,
appear to be parts of continuous extended emission rather than local peaks;
``streaking'' artifacts in the reduced data result in their identification
despite our threshold criteria.  The dendrogram extraction also identified
sources within the IRS 2 \hii region that are not dust sources.  Dendrogram
extraction missed a few clear sources in the low-noise regions away from
W51 Main and IRS 2 because the identification criteria were too conservative.

When extracting properties of the `by-eye' sources, we used variable sized
circular apertures, where the apertures were selected to include all of the
detectable symmetric emission around a central peak up to a maximum radius
$r\sim0.6$\arcsec.  This approach is necessary, as some of the sources are not
centrally peaked and are therefore likely to be spatially resolved starless
cores.
% can probably argue safely that the 'background' is filtered out as much
% as we would want...
%For spectral extraction, we also measured a background spectrum averaging
%over an annulus with twice the radius of the source extraction.

\subsection{The spatial distribution of cores}
\label{sec:corespatialdistribution}
The detected cores are not uniformly distributed across the observed region.
The most notable feature in the spatial distribution is their alignment: most cores
collect along approximately linear features.  This is especially evident
in W51 IRS2, where the core density is very high and there is virtually no
deviation from the line.  The e8 filament is also notably linear, though there
area few sources detected just off the filament. 

On a larger scale, the e8 filament points toward e2, apparently tracing a
slightly longer filamentary structure.  With some imagination, this might be
extended along the entire northeast ridge to eventually connect in a broad
half-circle with the IRS2 filament (Figure \ref{fig:corepositions}).  This
morphology hints at a possible sequential star formation event, where some
central bubble has swept gas into these filaments.  However, this ring has no
counterparts in ionized gas, and there is little reason to expect such circular
symmetry from a real cloud, so the star forming circle may be merely a figment.

Whether it is physical or not, there is a notable lack of cores within the
circle.  There is no lack of molecular gas, however, as both CO and \formaldehyde
emission fill the full field of view.

\Figure{figures/core_spatial_distribution.png}
{The spatial distribution of the hand-identified core sample.
The black outer contour shows the observed field of view.
The dashed circle shows a hypothetical ring of star formation.
}{fig:corepositions}{1}{16cm}

\subsection{Photometry}
\label{sec:photometry}
We created a catalog of the sources including their peak and mean
flux density, their centroid, and their geometric properties.  For each source,
we further extracted aperture photometry around the centroid in 6 apertures:
0.2, 0.4, 0.6, 0.8, 1.0, and 1.5\arcsec.  We performed the same aperture
photometry on the W51 Ku-band images from \citet{Ginsburg2016a} to estimate
the free-free contribution to the observed flux density measurements.  These
measurements are reported in Table {...}.

The source flux density distribution is shown in Figure
\ref{fig:fluxhistograms}.  The most common nearest-neighbor separation between
cataloged cores is $\sim0.3\arcsec$, which implies that the larger apertures
double-count some pixels.  The smallest separation is
0.26\arcsec, so the 0.2\arcsec aperture contains only unique pixels.


Except where noted below, the hand-selected sources are used for further
analysis as they are more reliable.  However, to encourage reproducible results
- which hand-extracted source positions defy - we also provide the
algorithmically-extracted and selected dendrogram catalog.

\Figure{figures/dendro_core_flux_histograms.png}
{Histograms of the core flux densities measured with circular apertures centered
on the dendrogram-extracted core centroids.  The aperture size is listed 
in the y-axis label.  Free-free-dominated sources are excluded.}
{fig:dendrofluxhistograms}
{1}{16cm}

\Figure{figures/core_flux_histogram_apertureradius.png}
{Histograms of the core flux densities measured with circular apertures centered
on the hand-extracted core positions.  The aperture size is listed 
in the y-axis label.  For the top plot, labeled `Peak', this is the peak
flux density in Jy/beam.  For the rest, it is the integrated flux density
in the specified aperture.  The unfilled data show all sources and the hashed
data are for starless core candidates (Section \ref{sec:contsourcenature}).}
{fig:fluxhistograms}
{1}{16cm}


\subsubsection{Distribution Functions}
\label{sec:distributionfunctions}
We fit power law distributions to each aperture's flux distribution using the
packages \texttt{plfit} and \texttt{PowerLaw}
\citep[https://github.com/keflavich/plfit,
https://github.com/jeffalstott/powerlaw;][]{Clauset2007a,Alstott2014a}.  The
powerlaws steepen slightly from $\alpha=2.0\pm0.12$ to $\alpha=2.2\pm0.16$ for
larger apertures.  The minimum flux density represented by a power law
increases from $\sim20$ mJy for the peak flux density distribution to 0.4 Jy
for the largest aperture (14-280 \msun at 20K).  These slopes are shallower
than the Salpeter-like slope for the mass function derived by
\citep{Konyves2015a} for their sample, though with only modest significance
($<3-\sigma$).  Of course, these measurements are of the continuum flux
density, not directly of the mass, and so a direct comparison may not be
appropriate.  We revisit this question after assessing the dust temperature in
Section \ref{sec:temperature}.


%\subsection{The source flux density distribution \& the core mass function}
%\todo{maybe}


%Not used any more: the fitting approach is better, it just took longer to
%implement
% \subsection{Spectral Lines \& Velocities}
% \label{sec:losvelo}
% To determine the line-of-sight velocity of each source, we extracted a spectrum
% from an 0.5\arcsec aperture centered on the source and from a 0.5-1.0\arcsec
% annulus around it.  We then searched each spectrum for the brightest pixel and
% associated it with the likeliest spectral line.  We repeated this in each of
% our 4 spectral windows, then averaged the 4 velocities to get an estimate of
% the source velocity.   This process also allowed us to identify the brightest
% lines in each window and the brightest overall line observed, which we use
% later for temperature estimation.


% \subsection{Dense Gas Kinematics}
% \label{sec:kinematics}
% We examine the gas kinematics throughout the cloud, but especially near the 
% massive cores.
% 
% The ambient cloud, which consists of gas that has not yet condensed into
% compact prestellar objects, is evident in absorption against the mm cores at
% 55-58 \kms (toward e2) and 57-60 \kms (toward e8).  Narrower velocity
% components related to the known high-velocity streams are detected around 68
% \kms toward both sources.  These absorption features are seen in all 
% \formaldehyde transitions, \methanol $4_{2,2}-3_{1,2}$, OCS 18-17,
% but it was less obvious or absent in HNCO $10_{1,10}-9_{1,9}$ and OCS 19-18.
% 
% In the material surrounding the e2 and e8 ``cores'', one particularly notable
% feature is that the cores themselves show a redshifted centroid velocity
% relative to their surroundings in nearly all of the bright lines (H2CO, OCS,
% SO, \ceighteeno).  The observed shift is up to $\lesssim2$ \kms.  The shift is
% a sign of infall.  Given the high continuum brightness, the cores are likely
% optically thick in the continuum (Section \ref{sec:} XXX), therefore obscuring
% all molecular emission behind them.  We are seeing only gas in the
% foreground, and this gas is clearly moving toward the cores.
% 
% The source ALMAmm14 shows a similar kinematic signature....


\subsection{Temperature estimation of the continuum sources}
\label{sec:temperature}
\label{sec:methanol}
The temperature is a critical ingredient for determining the total mass of each
continuum source or region. Since we do not have any means of directly
determining the dust temperature, as the SED peak is well into the THz regime
and inaccessible with any existing instruments at the requisite resolution, we
employ alternative indicators.  Above a density $n\gtrsim10^5-10^6$ \percc,
the gas and dust become strongly collisionally coupled, meaning the gas
temperature should accurately reflect the dust temperature.  Below this density,
the two may be decoupled.

The average dust temperature, as estimated from Herschel Hi-Gal SED fits
\citep{Molinari2016a,Wang2015a}, is 38 K when including the 70 \um data or 26 K
when excluding it.  This average is obtained over a $\sim45\arcsec$ ($\sim 1$
pc) beam and therefore is likely to be strongly biased toward the hottest dust
in the HII regions and around the massive cores, which have
temperatures reaching $>300$ K \citep{Goddi2016a}.  Despite these
uncertainties, this bulk measurement provides us with a reasonable range to
assume for the uncoupled, low-density dust, which (weakly) dominates the mass
(see Section \ref{sec:massbudget}).

One constraint on the dust temperature we can employ is the absolute surface
brightness.  For some regions, especially the e8 filament and the hot cores,
%noted in Section \ref{sec:morphology},
the surface brightness is substantially
brighter than is possible for a beam-filling, optically thick blackbody at 20
K, providing a lower limit on the dust temperature ranging from 30 K (40
mJy/beam) to 600 K (1 Jy/beam).  Toward most of this emission, optically-thick
free-free emission can be strongly ruled out as the driving mechanism using
existing data that limits the free-free contribution to be $<50\%$ if it is
optically thick, and negligible ($<<1\%$) if it is optically thin at radio wavelengths
\citep{Ginsburg2016b, Goddi2016a}.
%(Ginsburg et al, 2016; Goddi2016a).

To gain a more detailed measurement of the dust temperature in regions where it
is likely to be coupled to the gas, we use the peak brightness temperature
$T_{B,max}$ of spectral lines along the line of sight.  If the observed
molecule is in local thermal equilibrium, as is expected if the density is high
enough to be collisionally coupled to the dust, and it is optically thick, the
brightness temperature provides an approximate measurement of the local
temperature near the $\tau=1$ surface.  If any of these assumptions do not
hold, $T_{B,max}$ will set a lower limit on the true gas temperature.  Only
nonthermal (maser) emission would push $T_{B,max} > T_{gas}$.

One potential problem with this approach is if the gas becomes optically thick
before probing most of the dust.  Some transitions of more abundant molecules,
e.g., CO and \formaldehyde, are likely to be affected by this issue.  However,
many of the molecules included in the observations have lower abundances and
are likely to be optically thin along most of the line of sight.

Some sources have no detected line emission aside from the molecular cloud
species CO and \formaldehyde.  The minimum density requirement imposed by a
continuum detection at our limit of 1.6 mJy is $n>10^{7.5}$ \percc for a
spherical source.  At such high density, it is unlikely that the species are
undetected because they are subthermally excited.  More likely, the
line-nondetection sources have an underyling emission source that is very
compact, optically thick, and/or cold.

Figure \ref{fig:peaktbhist} shows the distribution of peak line brightnesses
for the continuum sources.  The spectra used to determine this brightness are
the mean spectra over the continuum photometry aperture.  To obtain the peak
line brightness, we fit Gaussian profiles to each identified line listed in
Table \ref{tab:linelist}, rejecting those with poor fits.  The line
brightnesses reported in the figure are the sum of the continuum-subtracted
peak line brightness and the continuum brightness.  Excepting CO and
\formaldehyde, which are excluded from the plot, \methanol is the brightest
line toward most sources. 

\Figure{figures/brightest_line_histogram.png}
{Histogram of the brightest line toward each continuum source.
The bars are colored by the molecular species associated with the brightest
line that is not associated with extended molecular cloud emission,
i.e., CO and its isotopologues and \formaldehyde are excluded.}
{fig:peaktbhist}{1}{10cm}

We use these peak line brightness temperatures to compute the `corrected'
masses of the continuum sources.  For sources with $T_{B,max} < 20$ K, we
assume $T_{dust} = 20$ K to avoid producing unreasonably high masses; in such
sources the lines are likely to be optically thin and/or subthermally excited.
The correction is illustrated in Figure \ref{fig:moftbvsm20k}.

\Figure{figures/aperture_mass20K_vs_massTB.png}
{The mass computed at the peak brightness temperature vs. that computed
assuming 20 K  for the aperture extracted continuum sources.
The faded sources in the top left of the diagram are those with $M(T_{B,max}) >
M(20\textrm{K})$; the circles along the dashed line show their $M(20\mathrm{K})$.
The dashed line shows $M(T_{B,max}) =
M(20\textrm{K})$ and the dashed line shows $M(T_{B,max}) = 0.1 
M(20\textrm{K})$ 
}{fig:moftbvsm20k}{1}{10cm}

%\Figure{figures/dendro_peakTB_vs_selfconsistentcontinuum.png}
%{The peak line brightness vs the continuum brightness of our target sources
%within $\sim1\arcsec$ beams.  The points are color-coded by the brightest
%observed line.  The dashed line shows where the brightness temperature of the
%continuum matches that of the lines; technically this means that it should be
%impossible for any point to be below the line.  
%%However, in this
%%iteration of the plot, the lines are extracted from broader apertures than the
%%continuum, so the continuum peak brightness is capable of being higher than the
%%line peak brightness.  TODO: replace this plot once spectra have been
%%appropriately extracted from the high-resolution spectral data cube.
%}
%{fig:peaktb}{1}{10cm}


\subsection{The nature of the continuum sources}
\label{sec:contsourcenature}
Millimeter continuum sources in star-forming regions are usually assumed to be
either protostars or starless cores.  However, in this high-mass star-forming
region, we have to consider both those possibilities and potential free-free
sources or high-luminosity main-sequence stars embedded in dust.

We fit each of up to $\sim50$ lines (see Table ...) with Gaussian profiles to
attempt to determine the relative line strengths toward each source.  Most
sources were detected in at least $\sim5-10$ lines, though some of these are
associated with interstellar rather than circumstellar material, i.e.,
\formaldehyde, CO, $^{13}$CS.  For sources with detections in non-interstellar
lines, we used the peak brightness temperature of the line as an estimated lower
limit on the core temperature.

In the continuum, we measured a `concentration parameter', which is the ratio
of the flux density in a 0.2\arcsec aperture to that in a 0.2\arcsec-0.4\arcsec
annulus divided by three to account for the annulus' larger area.  A uniform
source with $r>0.4\arcsec$ source would have a concentration $C=1$ by this
definition, while an unresolved point source would have a Gaussian profile
resulting in $C=14$.  Only one source approaches this extreme, the HII region
e5, while the rest have $C\leq7$.  We set the threshold for a `concentrated'
source to be $C>2$, which is arbitrary, but does a reasonable job of
distinguishing the sources with a clear central concentration from those that
have none.

Main sequence OB stars and their illuminated ionized nebulae are in principle
easily identified by their free-free emission.  Starless cores, protostellar
cores, and their variants are more difficult to identify, so we used a combination
of temperature and concentration parameter to classify them.

We classified each of the 75 hand-selected sources on the following parameters:
\begin{enumerate}
    \item Free-free dominated sources ($S_{15 GHz} > 0.5 S_{226 GHz}$) are \hii
        regions
    \item Free-free contaminated sources ($S_{15 GHz} > 0.1 S_{226 GHz}$) are
        likely to be dust-dominated but with \hii region contamination; these
        are either dusty sources superposed on or embedded in a large \hii
        region or they are compact, dusty \hii regions
    \item Starless core candidates were identified as those with cold peak
        brightness temperatures $T_B < 20$ K and with a high concentration
        parameter ($C>2$)
    \item Hot core candidates are those with peak $T_B>50$ K and $C>2$
    \item Extended cold core and hot core candidates are those with $T_B<20$ 
        and $T_B>50$ K and $C<2$
    \item The remaining sources with $S_{15 GHz} < 0.1 S_{226 GHz}$ and $50 >
        T_B > 20$ K were classed as uncertain compact ($C>2$) or uncertain extended
        $C<2$
\end{enumerate}

The classification is a broad guideline for further analysis.   



%*demonstrate this* Generally, the obviously interstellar features are narrower.

%idea: concentration parameter, compare to power-law cores of varying indices.
%Protostar / core ratio on this basis?  Evolutionary indicator of subregions?





\subsubsection{W51e2e mass and temperature estimates from continuum}
\label{sec:W51e2e}

% analysis done in total_mass_analysis and (originally) dust_properties
In a $0.21\arcsec\times0.19\arcsec$ beam ($1100\times1000$ au), the peak flux
density toward W51 e2e is 0.38 Jy, which corresponds to a brightness
temperature $T_B=225$ K.  This is a lower limit to the surface brightness of
the millimeter core, since an optical depth $\tau<1$ or a filling factor of the
emission $ff<1$ would both imply higher intrinsic temperatures.  The implied
luminosity, assuming a pure blackbody, is $L = 4\pi r^2 \sigma_{sb} T^4 =
2.3\ee{4} \lsun$.  Since any systematic uncertainties imply a higher
temperature, this estimate is a lower limit on the source luminosity.

If we assume that $T_{dust} = T_{peak}$ and that the dust is optically thin, we
derive a dust mass $M_{dust}\sim20$ \msun.  This  mass is not a strict limit in
either direction: if the dust is optically thick, there may be substantial
hidden or undetected gas, while if the filling factor is low, it may be much
hotter and therefore lower in mass.  However, simulations and models both
predict that the dust will become highly optically thick at radii
$r\lesssim1000$ au \citep{Forgan2016a,Klassen2016a}, so it is likely that
this measurement provides only a lower limit on the total gas mass surrounding
the protostar.  If we directly assume that the dust is optically thick
throughout our beam, and assume an opacity constant $\kappa=120$ g \persc,
the minimum mass per beam to achieve $\tau\geq1$ is $N=18$ \msun beam$^{-1}$.

For an independent measurement of the temperature that is not limited
to the optically thick regions, we use the \methanol
lines in band, calculating an LTE temperature that is $200 < T < 600$ K
out to $r<2$\arcsec ($r<10^4$ au; Section \ref{sec:methanol}).  As noted
in Section \ref{sec:methanol}, these temperatures may be overestimates
when the low-J lines of \methanol are optically thick, but for now they
are the best measurements we have available.

\subsubsection{W51 e8 and north mass and temperature from continuum}
\label{sec:w51e8andnorth}

The lower limit luminosities of W51 e8 and north in a single beam, assuming the
brightest detected beam is optically thick, are 1.6\ee{4} and 3.9\ee{4} \lsun,
respectively.




\subsection{Radial mass profiles around the most massive cores}
\label{sec:radialmass}
In Figure \ref{fig:hmradprof}, we show the radial profiles extracted from the
three high-mass protostellar cores in W51: W51 North, W51 e2e, and W51 e8.
The plot shows the enclosed mass out to $\sim1\arcsec$ (5400 AU).  On larger
spatial scales, the enclosed mass rises more shallowly, indicating the end of the
core.
% too hard to measure this (though our data are capable of recovering spatial
% scales up to XXXXX).

All three sources show similar radial profiles, containing up to 3000 \msun
within a very compact radius of 5400 AU (0.03 pc).  However, the temperature
structure within these sources is certainly not homogeneous, and very likely a
large fraction of the total flux comes from $T\gtrsim300$ K heated material
\citep[Section \ref{sec:temperature}; ][]{Goddi2016a}.  If the observed dust were
all at 600 K, the mass would be $\sim17\times$ lower, 100 \msun, which we treat
as a strict lower bound as it is unlikely that the dust more than $\gtrsim1000$
au is so warm.  Additionally, it is very likely that a substantial mass of cold
dust is also present but undetectable because it is hidden by the hotter dust.
%\todo{This paragraph is somewhat redundant with Section \ref{sec:W51e2e}, but
%still useful.  Perhaps blend these?}

%In the broader region surrounding W51e2e, out to a radius $\sim8000$ au, the
%total mass is $M\approx1400 (T / 100 \mathrm{K})^{-1}$ \msun  (Figure
%\ref{fig:hmradprof}).  On these larger scales, this moderately cooler ($T=100$
%K) assumed temperature and optically thin dust are more reasonable, though
%a factor of 2-5 increase in temperature may be warranted.  Nonetheless, the mass
%heated by e2e is enormous, constituting an entire star cluster's
%worth of material.

\FigureTwo
{figures/cumulative_radial_flux_massivecores.png}
{figures/cumulative_radial_mass_of_TCH3OH_massivecores.png}
{The cumulative flux density radial profiles centered on three massive
protostellar cores.  They share similar profiles and are likely dominated by
hot dust in their innermost regions, but they are more likely to be dominated
by cooler dust in their outer, more massive regions.  The cumulative mass
distribution may therefore be deceptive.  In (a), we use a constant
temperature.  In (b), we use the temperature map computed from \methanol in
Section \ref{sec:methanol}.}
{fig:hmradprof}{1}{8cm}

%probably not worth doing
% \subsection{The most massive prestellar cores in W51}
% % may be repeated from temperature section above
% To place limits on the most massive prestellar cores, we need to know the
% temperature of the dust in all of the bright ($>15$ mJy; 10 \msun at 20 K)
% sources.  We do not have any direct means of evaluating the dust temperature,
% but we can infer at
% least a lower limit on it by determining the peak brightness temperature of an
% optically thick line that is excited at densities $n\gtrsim10^5$ \percc, at
% which the dust and gas are coupled.
% 
% To accomplish this, we have found the brightest lines across the full $\sim6$
% GHz spectra and measured their peak brightness temperature.
% 
% \todo{TODO: use coarser resolution (2\arcsec = 0.1 pc?) data to extract ``cores''
% where no smaller (protostellar) cores are detected.  Try to estimate their masses.
% These are the best candidate ``prestellar" cores, though likely anything this
% large is likely to be a massive cluster...}

\subsection{The mass and light budget on different spatial scales}
\label{sec:massbudget}
An evolutionary indicator used for star-forming regions is the amount of mass
at a given density; a more evolved (more efficiently star-forming) region will
have more mass at high densities.  We cannot measure the dense gas fraction
directly, but the amount of flux density recovered by an interferometer
provides a reasonable approximation.

% total_mass_analysis
For the ``total'' flux density in the region, we use the Bolocam Galactic Plane
Survey observations \citep{Aguirre2011a,Ginsburg2013a}, which are the closest
in frequency single-dish millimeter data available.  We assume a spectral index
$\alpha=3.5$ to convert the BGPS flux density measurements at 271.4 GHz to the
mean ALMA frequency of 226.6 GHz.  The ALMA data (\todo{specifically, the
0.2\arcsec 12m-only data}) have a total flux 23.2 Jy above a very conservative
threshold of 10 mJy/beam in our
mosaic; in the same area the BGPS data have a flux of 144 Jy, which scales down to
76.5 Jy.  The recovery fraction is 30$\pm3$\%, where the error bar accounts
for a change in $\alpha\pm0.5$.  The threshold of 10 mJy/beam corresponds to a
column threshold $N>1.3\ee{25}$ \percc for 20 K dust. This threshold also
corresponds to an optical depth of $\tau\approx0.5$, implying that a large
fraction of the cloud is either approaching optically thick or warmer than 20
K.  For an unresolved spherical source in the $\sim0.2\arcsec$ beam, this
column density corresponds to a volume density $n>10^{8.1}$ \percc.

% total_mass_analysis also (end of it)
Even more impressive is the amount of the total flux density concentrated
into the three `massive cores', W51 e2e, e8, and north.  These three contain
12.3 Jy (within 1\arcsec or 5400 AU apertures) of the total 23.2 Jy in the
observed field - more than half of the total ALMA flux density, or 15\% of the
BGPS flux density.

% \todo{TODO: determine the largest angular scale in the ALMA images.  Requires
% using the simulations.}

\subsection{Chemically Distinct Regions}
\label{sec:chemistrymaps}
\subsubsection{Observations}
The ``hot cores'' in W51 (e2, e8, and North) are spatially well-resolved and
multi-layered.  These cores are detected in lines of many different species
spanning areas $\sim5\ee{3}-10^4$ au across.

Surrounding W51e2e, there are relatively sharp-edged uniform-brightness regions
in a few spectral lines over the range 51-60 \kms (Figure
\ref{fig:chemmapse2}).  Some of these features are elongated in the direction
of the outflow, but most have significant extent orthogonal to the outflow,
spanning $9500\times6600$ AU.  They are prominent in \methanol, OCS, and
\dimethylether, weak but present in \formaldehyde and SO, and absent in
\cyanoacetylene and HNCO.

Around e8, a similar feature is observed, but in this case \dimethylether is absent.
Toward W51 north, \methanol, \formaldehyde, and SO exhibit the sharp-edged
enhancement feature, while the other species do not.  The enhancement is from
50-60 \kms.

By contrast, along the south end of the e8 filament, no such enhanced features
are seen; only \formaldehyde and the lowest transition of \methanol are
evident.

The relative chemical structures of e2, e8, and IRS2 are relatively similar.
The same species are detected in all of the central cores.  However, in e2,
\dimethylether, \methylformate, \ethylcyanide, and Acetone (\acetone) are
significantly more extended in e2 than in the other sources.
\gaucheethanol is detected in W51 North, but is weak in e8 and almost absent
in e2 (Figures \ref{fig:chemmapse2}, \ref{fig:chemmapse8},
\ref{fig:chemmapsnorth}, \ref{fig:chemmapsALMAmm14}).

Different chemical groups exhibit different morphologies around e2.  Species that
are elongated in the NW/SE direction are associated primarily with the outflow
(\cyanoacetylene, \ethylcyanide).  Other species are associated primarily with
the extended core (\methylformate, \dimethylether, \acetone).  Some are only
seen in the compact core (\methyleneamidogen, HNCO, \formamide, and
vibrationally excited \cyanoacetylene).  Only \methanol and OCS are associated
with both the extended core and the outflow, but not the greater extended
emission.  \ketene seems to be associated with only the extended core, but not
the compact core. Finally, there are the species that trace the broader ISM in
addition to the cores and outflows (\formaldehyde, $^{13}$CS, OCS, C$^{18}$O
and SO).  Both HCOOH and N$_2$D+ are weak and associated only with the innermost
e2e core.

\Figure{figures/chemical_m0_slabs_e2.png}
{Moment 0 maps of the e2 region in 40 different lines over the range 51 to 60
\kms with continuum subtraction using the 30th percentile emission
over the ranges 25-40 and 75-90 \kms.  All images are on the same scale, and
the negative features show absorption against the continuum.  There
is a strong `halo' of emission seen in the CH$_3$Ox lines and OCS.  Extended
emission is also clearly seen in SO, $^{13}$CS, and \formaldehyde, though these
lines more smoothly blend into their surroundings.  HNCO and \formamide have
smaller but substantial regions of enhancement with a sharp contrast to their
surroundings.  HC$_3$N traces the e2 outflow.  The bright H30$\alpha$ emission
marks the position of e2w, the hypercompact HII region that dominates the
centimeter emission in e2.
}{fig:chemmapse2}{1}{18cm}

\Figure{figures/chemical_m0_slabs_e8.png}
{Moment 0 maps of the e8 region in 40 different lines over the range 52 to 63
\kms with continuum subtraction using the 30th percentile emission
over the ranges 25-40 and 75-90 \kms.  All images are on the same scale, and
the negative features show absorption against the continuum.  As in e2,
there is extended emission in the CH$_3$OH and OCS lines, but in contrast,
the othe CH$_3$Ox lines are more compact. SO is brighter than OCS in e8, 
whereas the opposite is true in e2.
}{fig:chemmapse8}{1}{18cm}

\Figure{figures/chemical_m0_slabs_north.png}
{Moment 0 maps of the W51 IRS2 region in 40 different lines over the range 54 to 64
\kms with continuum subtraction using the 30th percentile emission
over the ranges 25-40 and 75-90 \kms.  All images are on the same scale, and
the negative features show absorption against the continuum.  Qualitatively,
 the relative extents of species seem comparable to e8.  The H30$\alpha$ 
 point source is W51 d2.
}{fig:chemmapsnorth}{1}{18cm}

\Figure{figures/chemical_m0_slabs_ALMAmm14.png}
{Moment 0 maps of the ALMAmm14 region in 40 different lines over the range 58 to 67
\kms with continuum subtraction using the 30th percentile emission
over the ranges 25-40 and 75-90 \kms.  All images are on the same scale. 
ALMAmm14 is one of the brightest sources outside of e2/e8/IRS2, but
it is substantially fainter than those regions.  Still, it has a noticeably
rich chemistry.
}{fig:chemmapsALMAmm14}{1}{18cm}

\subsubsection{Chemical structure: Interpretation}
\label{sec:cheminterp}
\todo{Probably cut P1 here and move the whole thing to "Discussion"}
The enhancements in noted lines occur as factor of 3-10 increases in the peak
brightness of most of the lines shown in Figure \ref{fig:chemmapse2}.  These
enhancements could occur from increased total column density, increased
abundance of the molecules, or increased excitation.

In Section \ref{sec:ch3ohtem}, we examine the \methanol abundance
and excitation conditions. 
The detection of highly excited \methanol lines, including
$18_{3,15}-17_{4,14}$ ($E_U=447$ K) and $25_{3,22}-24_{4,20}$ ($E_U=802$ K),
suggests that excitation is part of the explanation for the brighter
molecular emission.  However, LTE modeling reveals that the \methanol abundance
increases by a factor of $\sim5-10$ from the protocluster gas inward toward the
e2e core.  This abundance gradient is likely present in other molecules as
well.  Given the enhanced abundance of \methanol, these sharp-edged bubbles
probably represent sublimation zones in which substantial quantities of
grain-processed materials are released into the gas phase.  The relatively
sharp edges likely reflect the particular point where the temperature exceeds
the sublimation temperature for each species.


\subsection{\methanol temperatures \& columns in the hot cores}
\label{sec:ch3ohtem}
The extreme chemical regions appear to be associated with regions of
elevated gas temperature.  We examine this directly by analyzing
the excitation of lines for which we have detected multiple transitions with
significant energy differences.  We do not use \formaldehyde for this analysis
because it is clearly optically thick (self-absorbed) in all lines in the
regions of greatest interest.

We produce rotational diagrams for each spatial pixel covering all \methanol lines
detected at high significance toward at least one position.  The detected lines
span a range $45 < E_U < 800$ K, allowing robust measurements of the temperature
assuming the lines are optically thin, in LTE, and the gas temperature is high
enough to excite the lines.  These conditions are likely to be satisfied in the
e2e, e8, and North cores, except for the optically thin requirement.  Luckily,
there are some lines in band that have much lower Einstein $A_{i,j}$ values
but comparable upper-state energy levels, allowing us to probe higher column
densities than would otherwise be possible.

\FigureTwo{figures/ch3oh_temperature_map_e2.png}{figures/ch3oh_column_map_e2.png}
{Methanol temperature and column density maps around e2.  The central regions around
the cores appear to have lower column densities because the lines become
optically thick and self-absorbed.}
{fig:ch3ohe2}{1}{10cm}

Sample fitted rotational diagrams are displayed in Figure \ref{fig:ch3ohe2rot}.
The line intensities are computed from moment maps integrating over the range
(51, 60) \kms in continuum-subtracted spectral cubes, where the continuum
was estimated as the median over the ranges (25-35,85-95) \kms, except
for the J=25 lines, which had a continuum estimated from the 10th percentile
over the same range to exclude contamination from the SO outflow line wings.

\Figure{figures/ch3oh_rotation_diagrams_e2.png}
{A sampling of fitted rotation diagrams of the detected \methanol transitions.
These are meant to provide validation of the temperatures and column densities
derived and shown in Figure \ref{fig:ch3ohe2}.  The lower-left corner shows
the position from which the data were extracted in that figure in units of
figure fraction.  The horizontal black lines show the detection threshold of each
of the transitions; points below these lines are ignored when fitting, and instead
the threshold itself is used.  The fitted temperature and
column are shown in the top right of each plot.
}{fig:ch3ohe2rot}{1}{18cm}

To validate some of the rotational diagram fits, we examined the modeled
spectra overlaid on the real (Figure \ref{fig:ch3ohe2epeaks}).  These generally
display significant discrepancies, especially at low J where self-absorption is
evident.  In Figure \ref{fig:ch3ohe2epeaks}, there is clearly a low-temperature
component slightly redshifted from the high-J peak that can be seen as a dip
within the line profile.  The presence of this unmodeled low-temperature
component renders our \methanol temperature measurements uncertain.



\FigureTwo
{figures/ch3oh_rotdiagram_fits_SelectedPixel1.png}
{figures/ch3oh_rotdiagram_fits_SelectedPixel2.png}
{Spectra of the \methanol lines toward a selected pixel just outside of the e2e
core.  The red curves show the LTE model fitted from a rotational diagram as
shown in Figure \ref{fig:ch3ohe2rot}.  The model is not a fit to the data
shown, but is instead a single-component LTE model fit to the integrated
intensity of the lines shown.  As such, the fit is not convincing, and it is
evident that a single-temperature, single-velocity model does not explain the
observed lines.  Nonetheless, a component with the modeled temperature is
likely to be present in addition to a cooler component responsible for the
self-absorption in the low-J lines.  (a) shows a pixel close to the center of
e2e, which is probably optically thick in most of the shown transitions, while
(b) shows a better case where the highest-$A_{ij}$ lines are overpredicted but
many of the others are well-fit.}
{fig:ch3ohe2epeaks}{1}{8cm}

Figure \ref{fig:ch3ohvscont} shows a comparison between the \methanol
$10_{2,9}-9_{3,6}$ line and the 225 GHz continuum.  While the brightest regions
in \methanol mostly have corresponding dust emission, the dust morphology
traces the \methanol morphology very poorly.  This difference suggests that the
enhanced brightness is not simply because of higher total column density.
We examine the dust-\methanol link more quantitatively in Figure
\ref{fig:ch3ohtemX}.

\FigureTwo{figures/continuum_contours_on_ch3oh1029.png}
          {figures/ch3oh1029_contours_on_continuum.png}
{Images showing \methanol $10_{2,9}-9_{3,6}$ and 225 GHz continuum emission,
with \methanol in grayscale and continuum in contours (left) and continuum in
grayscale, \methanol in contours (right).  The fainter (whiter) regions in the center
of the \methanol map correspond to the bright continuum cores and show where all lines
appear to be self-absorbed.}
{fig:ch3ohvscont}{1}{8cm}

Figure \ref{fig:methanolradialprofile} shows the observed brightness profiles
of \methanol line and dust continuum emission.
Figure \ref{fig:ch3ohtemX} shows a comparison of the \methanol temperature and
abundance.  The \methanol abundance is derived by comparing the rotational
diagram (RTD) fitted \methanol column density to the dust column density while
using the \methanol-derived temperature as the assumed dust temperature.  The
figure shows all pixels within a 3\arcsec (16200 AU) radius of e2e, with pixels
having low column density and high temperature (i.e., pixels with bad fits) and
those near e2w (which may be heated by a different source) excluded.  We used
moment-0 (integrated intensity) maps of the \methanol lines to perform these
RTD fits, which means we have ignored the line profile entirely and in some
cases underestimated the intensity of the optically thick lower-J lines: in the
regions of highest column, the column is  underestimated and the temperature is
overestimated, as can be seen in Figure \ref{fig:ch3ohe2epeaks}.


\Figure{figures/radialprofile_max_CH3OH_e2.png}
{Radial profiles of the peak surface brightness of five \methanol transitions
along with the profile of the continuum brightness.  The radial profiles were
constructed from images with 0.2\arcsec resolution including only 12m data.
The central dip shows where the lines go into absorption, though they are only
seen in absorption at $\sim55$ \kms.  The \methanol lines are
continuum-subtracted.}
{fig:methanolradialprofile}{1}{12cm}

% plot_code/overlay_contours_on_ch3oh.py
\FigureFour
{figures/e2_CH3OH_LTE_temperature_vs_abundance.png}
{figures/e2_CH3OH_LTE_temperature_radial_profile.png}
{figures/e2_CH3OH_LTE_abundance_radial_profile.png}
{figures/e2_CH3OH_LTE_vs_dust_column.png}
{Comparison of the \methanol temperature, column density, and abundance.
(a) The relation between temperature and abundance.  There is a weak correlation,
but most of the high abundance regions are at high temperatures.
(b) Temperature vs distance from e2e.  There is a clear trend toward higher
temperatures closer  to the central source
(c) Abundance vs distance from e2e.  The apparent dip at $r<1\arcsec$ is
somewhat artificial, as it is driven by a rising dust emissivity that
corresponds to an increasing optical depth in the dust.  The \methanol column
in this inner region is likely to be underestimated. 
(d) \methanol vs dust column density.  }
{fig:ch3ohtemX}

A few features illustrate the effects of thermal radiative feedback on the gas.
The temperature jump starting inwards of  $r\sim1.5\arcsec$ (8100 AU; Figure
\ref{fig:ch3ohtemX}b) is
substantial, though the 100-200 K floor at greater radii is likely artificial as
the low-J transitions are not consistent with a thermal distribution.
There is an abundance enhancement at the inner radii, but it appears to be a
radial bump rather than a pure increase.  The abundance enhancement is probably real,
and is approximately a factor of $\sim5-10$.  The inner abundance dip
is caused by two coincident effects: first, the \methanol column becomes underestimated
because the \methanol is \emph{self}-absorbed, and second, the dust becomes
optically thick, blocking additional \methanol emission, though this latter
effect is somewhat self-regulating since it also decreases the dust column (the
denominator in the abundance expression).


% presently, this section adds nothing to the paper-in-progress...
% \subsection{Temperatures derived from \formaldehyde}
% The original goal of this project was to measure the gas temperatures in the
% moderate-density ($n\sim10^4-10^5$ \percc) gas that may correspond to
% pre-stellar material.  We performed the same analysis as was done in \citet{Ginsburg2016a}
% to create a \formaldehyde temperature map, but found very high temperatures even
% in the regions expected to be cold, with a temperature floor around 50 K.
% 
% There are a few possible explanations for this high thermal floor.  First is
% purely observational: the maps do not include any zero-spacing information and
% may therefore have resolved out some emission.  It is possible that the
% \threeohthree line is preferentially filtered, as it should be the brightest
% and most widespread of the triplet.  This possibility can be examined when
% zero-spacing data become available.
% 
% Second, it is possible that a large fraction of the area of the cloud is
% optically thick in at least the \formaldehyde \threeohthree transition.  Such a
% high optical depth is not expected since the observed brightness temperatures
% typically peak at $\sim1$ K and max out at $\lesssim15$ K outside of the
% central protoclusters.  Such a low brightness temperature for optically thick
% gas would imply that the molecules are subthermally excited but highly
% abundant.
% 
% Third, the temperatures could be genuinely high.  A Galactic molecular cloud
% should not be in thermal equilibrium at 50 K, but should readily cool to 10-20
% K, so such a cloud would have to be subject to extraordinary heating
% conditions.  Indeed, W51 is subject to some fairly extreme conditions, with one
% of the Galaxy's most luminous HII regions encompassing most of the molecular
% material.  Even with such heating, though, the densest gas should be able to
% cool well below 50 K.  The UV photons ionizing the HII region cannot penetrate
% these densest regions and the infrared radiation field is not strong enough to
% maintain such a high floor (is it?).  Curiously, the CO 3-2 maps of
% \citet{Parsons2012a} exhibit peak brightness temperatures up to 50 K within the
% mapped region, so it is plausible that the moderate-density medium is much
% warmer than in a typical cloud.
% 
% To evaluate this third option, current data are insufficient.  Ruling out option (1),
% filtering, would be helpful but not definitive.  Additional observations of J=2 and
% J=4 \formaldehyde transitions or the $H_2$$^{13}$CO J=3 lines would be enough to rule
% out hypotheses (1) and (2).
% 
% \Figure{figures/W51e2_full_h2co_aplpy.pdf}
% {An RGB composite image of the peak intensity of the three \formaldehyde lines.
% The red channel, \threeohthree, has an upper-state energy level 23 K, so redder
% regions are qualitatively cooler than whiter or bluer regions.  However, the solid
% white zones around W51 e2, e8, and north are areas where all three lines become
% very optically thick, so the color no longer implies anything about the temperature.}
% {fig:h2corgb}{1}{12cm}

\subsection{Ionizing vs non-ionizing radiation}
\label{sec:nonionizingradiation}
The formed and forming protostars are producing a total $\gtrsim10^7$ \lsun of
far infrared illumination \citep{Ginsburg2016b}.  This radiation heats the
cloud's molecular gas, affecting the initial conditions of future star
formation.

The ionizing radiation in W51 was discussed in detail in \citep{Ginsburg2016b}.
Ionizing radiation affects much of the cloud, but little of the high-density
prestellar material.  There is no evidence of increased gas temperatures in the
vicinity of \hii regions.  While in Section \ref{sec:nonionizingradiation} we
identify chemically enhanced regions as those where radiative feedback has
heated the dust and released ices into the gas phase, no such regions are
observed surrounding the most luminous compact \hii regions.

This lack of molecular brightness enhancement is only an indirect indication
that the \hii regions do not affect the surrounding dense gas temperature.  A
direct proof that the ionizing sources are having no or minimal effect would
require temperature measurements on the outskirts of the \hii regions, but this
is not possible without prior detection of bright emission in thermometric
transitions.

The chemical maps shown in Section \ref{sec:chemistrymaps} show the volumes of
gas clearly affected by newly-forming high-luminosity stars.  The
\methanol-enhanced region around W51e2 extends 0.04 pc, or 8500 AU (see Section
\ref{sec:ch3ohtem}). Other locally enhanced species, especially the nitrogenic
molecules HNCO and \formamide, occupy a smaller and more asymmetric region
around e2e and e2w (Figure \ref{fig:e2methanolhnco}).  These chemically enhanced
regions are most prominent around the weakest radio sources or regions with
no radio detection; they are most likely heated by direct infrared radiation
from these sources.

\FigureTwo
{figures/W51e2_ch3oh_hnco_continuum_aplpy_kucontours.pdf}
{figures/W51e8fil_ch3oh_hnco_continuum_aplpy_kucontours.pdf}
{Image of \methanol $8_{0,8}-7_{1,6}$ (red), HCNO $10_{0,10}-9_{0,9}$ (green), and 225 GHz
continuum (blue) toward (a) W51e2 (b) W51e8.  The contours show Ku-band radio continuum
emission tracing the \hii regions (a) W51 e2w and (b) W51e1, e3, e4, e9, and
e10.  The \methanol emission is relatively symmetric around the high-mass
protostar W51 e2e and the weak radio source W51 e8, suggesting that these
forming stars are responsible for heating their surroundings.  By contrast, the
\hii regions do not exhibit any local molecular brightness enhancements (except
e8), indicating that the \hii regions are not heating their local dense
molecular gas.}
{fig:e2methanolhnco}{1}{8.5cm}
 

\Figure
{figures/W51north_ch3oh_hnco_continuum_aplpy_kucontours.pdf}
{Image of \methanol $8_{0,8}-7_{1,6}$ (red), HCNO $10_{0,10}-9_{0,9}$ (green),
and 225 GHz continuum (blue) toward  north, as in Figure
\ref{fig:e2methanolhnco}.  The contours show Ku-band radio continuum emission
tracing the diffuse IRS 2 \hii region.}
{fig:northmethanolhnco}{1}{8.5cm}


\subsection{Outflows}
\label{sec:outflows}
We detected many outflows, primarily in CO 2-1 and SO $6_5-5_4$.  The flows are
weakly detected in some other lines, e.g. \formaldehyde, but we defer
discussion of outflow chemistry to a future work.

In this section, we discuss some of the unique outflows and unique features of
outflows in the W51 region.  We show the most readily identified outflows in
Figures \ref{fig:outflowscontinuume2}, \ref{fig:outflowscontinuumnorth},
\ref{fig:e2ecooutflow}, and \ref{fig:e8cooutflow}.

\subsubsection{The Lacy jet}
\label{sec:lacyjet}
A high-velocity outflow was discovered within the W51 IRS2 region by
\citet{Lacy2007a}, and subsequently detected in H77$\alpha$ by
\citet{Ginsburg2016b}.  We have discovered the CO counterpart to this
outflow, which comes from near the continuum source ALMAmm31 (Figure
\ref{fig:lacyjet}).  Strangely, though, the outflow is not directly centered on
the continuum source, but is slightly offset.  The outflow shows red- and
blue-shifted flows that form the base of the ionized outflow reported by
\citet[][Figure \ref{fig:outflowscontinuumnorth}]{Lacy2007a}.

%\todo{Make a figure of this \& describe data reduction if it is to be included.}
%Additionally, we have reduced archival VLT SINFONI observations of the region
%and discovered a 2-micron \hh knot positioned directly between the cold
%molecular (CO) and the ionized components of the flow.  This \hh emission
%reveals the position at which the CO is breaking out of the cloud and into the
%\hii region.

\FigureTwo
{figures/rgb_CO_continuum_outflows_aplpy_wideLacy.png}
{figures/rgb_SO_continuum_outflows_aplpy_wideLacy.png}
{Outflows shown in red and blue for (a) CO 2-1 and (b) SO $6_5-5_4$ with
continuum in green.  This symmetric molecular outflow forms the base of the
\citet{Lacy2007a} ionized outflow detected further to the east.
The continuum source is offset from the line joining the red and blue outflow lobes.}
{fig:lacyjet}{1}{8cm}


\Figure{{figures/NACO_green_outflows_aplpy_CONTours_hires_h77acontour}.png}
{Outflows in the W51 IRS2 region.  The green emission is NACO K-band continuum
\citep{Barbosa2008a}, with ALMA 1.4 mm continuum contours in white and
H77$\alpha$ contours in blue.  The \citet{Lacy2007a} jet is prominent in
H77$\alpha$.}
{fig:outflowscontinuumnorth}{1}{12cm}

\subsubsection{north}
The outflow from W51 north is extended and complex.
A jet-like high-velocity feature appears directly to the north of W51 north in
both CO and SO (Figure \ref{fig:outflowscontinuumnorth}).  However, in SO, this feature begins to emit at $\sim47$ \kms
and continues to $\sim 100$ \kms.  The CO emission below $<70$ \kms is
completely absent, presumably obscured by foreground material.  The blueshifted
component, by contrast with the red, points to the southeast and is barely
detected in CO, but again cleanly in SO.  It is sharply truncated, extending
only $\sim1 \arcsec$ ($\sim5000$ AU).  Unlike the Lacy jet, there is no
evidence that this outflow transitions into an externally ionized state.

The northernmost point of the W51 North outflow may coincide with
the \citet{Hodapp2002a} \hh and [Fe II] outflow.  There is some CO 2-1
emission coincident with the southernmost point of the \hh features,
and these all lay approximately along the W51 North outflow vector.
However, the association is only circumstantial.


\subsubsection{The e2e outflow}
The dominant outflow in W51, which was previously detected by the SMA
\citep{Shi2010a,Shi2010b}, comes from the source e2e.  This outflow is
remarkable for its high velocity, extending nearly to the limit of our spectral
coverage.  The ends of the flow cover at least $-50 < v_{lsr} < 160$ \kms, or a
velocity $v\pm100$ \kms.  

The morphology is also notable.  Both ends of the outflow are sharply truncated
at $\sim2.5\arcsec$ (0.07 pc) from e2e (Figure \ref{fig:outflowscontinuume2}).
To the southeast, the high-velocity flow lies along a line that is consistent
with the extrapolation from the northwest flow, but at lower velocities ($10 <
v_{LSR} < 45$ \kms), it jogs toward a more north-south direction (Figure
\ref{fig:e2ecooutflow}).  In the
northwest, the redshifted part of this flow ($70 < v_{LSR} < 120$ \kms)
apparently collides with a \emph{blue}shifted flow from another source ($22 <
v_{LSR} < 45$ \kms), suggesting that these outflows intersect, though such a
scenario seems  implausible given their small volume filling factor.

% really?  important implications?  Maybe not. [edit: -important]
The extreme velocity and morphology carry a few implications for the
accretion process in W51.  The sharp symmetric truncation, combined with the
extraordinary velocity, suggests that the outflow is freshly carving a cavity
in the surrounding dense gas.  The observed velocities are high enough that
their bow shocks likely dissociated all molecules, so some ionized gas is
likely present at the endpoints; this ionized gas has not been detected in
radio images because of the nearby 100 mJy HCHII region e2w.  The dynamical age
of the outflow is $\sim600$ years at the peak observed velocity, which is a
lower limit on the true age of the outflow.


\Figure{{figures/Alma1.4mmcont_green_outflows_aplpy_CONTours}.png}
{Outflows in red and blue overlaid on mm continuum in green with cm continuum
contours in white.  The northern source is e2, the southern source at the tip
of the long continuum filament is e8.}
{fig:outflowscontinuume2}{1}{12cm}


\Figure{{figures/e2e_CO2-1_channelmaps}.png}
{Channel maps of the e2e outflow in CO 2-1.  The dashed line approximately
connects the northwest and southeast extrema of the flow.}
{fig:e2ecooutflow}{1}{18cm}

\subsubsection{e8}
There are at least four distinct outflows coming from the e8 filament.
The e8 core is launching a redshifted outflow to the northwest.  A blueshifted
outflow is coming from somewhere south of the e8 peak and pointing straight
east.  While these originate quite near each other, they seem not to have
a common source, since the red and blue streams are not parallel (Figures
\ref{fig:outflowscontinuume2} and \ref{fig:e8cooutflow}).  The e8 outflows are too
confused and asymmetric for simple interpretation.


\Figure{{figures/e8_CO2-1_channelmaps}.png}
{channel maps of the e8 outflow in CO}
{fig:e8cooutflow}{1}{18cm}



% \todo{What direct tests can be used to show that \hii regions aren't heating
% their surroundings?  \formaldehyde is good, in principle, but maybe not in
% practice because of the possible optical depth issues.  Radio \ammonia might be
% OK, but it might also be affected by imaging artifacts from the bright radio
% sources.}



%incorrect, and discussed in sec:ch3ohtem in detail
% The extraordinarily high column densities of \methanol make direct temperature
% estimation impossible; many or all of the observed \methanol lines are almost
% certainly optically thick.  Fitted rotational diagrams resulted in negative
% temperatures throughout the \methanol-enhanced region, implying that the levels
% are not thermally populated.  However, the total column densities from these
% rotational diagram fits are reasonable lower-limits on the column, and they
% exceed $N(\methanol)\gtrsim10^{20}$ \persc.
% (demonstrate this?)

% \todo{If we assume $T_{ex}=100$ K, can we use the $5_{-4,2}-6_{-4,3}$ line
% to measure the column density?}
% Out of the five clearly detected \methanol lines, the $5_{-4,2}-6_{-4,3}$
% line seems to be subthermally excited and/or optically thin...



% \todo{How does the \methanol brightness/column profile compare with the dust
% brightness/column?  Is it going up faster?  By how much?}
% answered in fig:ch3ohtemX


%Is there any evidence that the main-sequence stars that illuminate the \hii
%regions in W51 \citep{Ginsburg2016b} affect the pre-star-forming gas throughout
%W51?
%\todo{Our ALMA program was designed to answer this question by measuring the
%temperature in the dense prestellar \formaldehyde-rich gas.  Naively, the data
%say "yes, the temperatures are all ridiculously high, $T>100$ K", but that
%can't be.  The \formaldehyde temperatures suggest that temperature is correlated
%with density, which unfortunately suggests instead that the \formaldehyde
%line optical depth is correlated with density.  It is therefore not straightforward
%to systematically examine the thermal feedback effects from MYSOs.}
%
%\todo{Notes from chatting with Wing-Fai Thi:  \methanol has a similar condensation
%temperature to water, so the desorbed region is probably $\sim90-100$ K.  HNCO
%has a much \emph{lower} desorption temperature, so if it was coming from grain
%surfaces, it should be more widespread than \methanol.  Since it is not, the
%enhancement is most likely due to gas-phase chemistry.}
%
%\todo{
%However, Ewine van Dishoeck pointed out that HNCO and \formamide can be mixed
%into ices that evaporate at a much higher temperature, consistent with the
%structure we observe.
%}


\section{Discussion}
\subsection{The scales and types of feedback}
The most promiment features of our observations are the warm, chemically
enhanced regions surrounding the highest dust concentrations, and the
corresponding \emph{lack} of such features around the ionized nebulae.  This
difference implies that the immediate star formation process - that of gas
collapse and fragmentation from a molecular cloud - is primarily affected by
feedback from other forming stars, \emph{not} from previous generations of
now-exposed stellar photospheres.

On the scales relevant to the fragmentation process, i.e., the $\sim0.1$ pc
scales of prestellar cores, this decoupling can be explained simply.  Stellar
light is produced mostly in the UV, optical, and near-infrared.  As soon as a
star is exposed, either by consuming or destroying its natal core, that light
is able to stream to relatively large ($\gtrsim1$ pc) scales before being
absorbed.  At that point, the stellar radiation is poorly coupled to the scales
of direct star formation.  By contrast, stars embedded in their natal cores
will have all of their light reprocessed from UV/optical/NIR to the far-IR
within a $<0.1$ pc sphere, providing a far-infrared background light capable of
heating its surroundings.

The different effects of ionizing vs thermal radiation can be seen directly in
the three main massive star forming regions, e2, e8, and north.  Figures
\ref{fig:e2methanolhnco} and \ref{fig:northmethanolhnco} show both the
highly-excited warm molecular gas in color and the free-free emission from
ionized gas in contours.  As described in Section
\ref{sec:nonionizingradiation}, the spatial differences indicate that the
ionizing radiation sources - the exposed OB stars - have little effect on the
star-forming collapsing and fragmenting gas.

% subsubsection on the formation of the IMF?
The low impact of photospheric radiation on collapsing gas suggests that
second-generation star formation is relatively unaffected by its surroundings.
Instead, the stars of the same generation - those currently embedded and
accreting - have the dominant regulating effect on the gas.  The formation of
the stellar initial mass function \emph{within clusters} is therefore
predominantly self-regulated, with little external influence.

% something about implications for star formation on galactic scales


%\subsection{Question: Where does the radiation from the HII regions end up?}
%The \hii regions show no signs of heating around them.  However, we know that
%these must be $>10^4$ \lsun stars, and even the infrared radiation should be
%rising with luminosity (or temperature).  While most of the energy might go
%into ionizing the gas cloud, many of the photons must get reprocessed into the
%infrared at some point.  If optical/NIR photons were escaping, we should be
%able to see them unless the geometry is particularly unfavorable.

%\subsection{The continuum sources}
%
%We have detected 75 distinct compact `sources' and characterized some of
%their basic properties assuming they consist purely of gas and dust, but this
%interpretation is incomplete.  At the least, all of the sources with peak
%surface brightnesses $T_{B,max} > 30$ K are likely to contain central heating
%sources, i.e., stars or protostars.
%
%\subsection{Limits on accretion onto HII regions}
%\citet{Peters2010a} and \citet{Klaassen2012a} proposed that ultra- and
%hyper-compact \hii regions may be variably accreting.  When accretion is most
%active, the \hii region is confined and shrinks or may even be turned off.
%When accretion is slower or weaker, the \hii region expands, following
%approximately Str{\"o}mgren expansion \todo{make sure that's actually what they
%say...}.  
%
%The observed lack of warm molecular gas around compact \hii regions suggests
%that they have not recently been accreting...
%
%d2 provides a counterpoint, however, as it is a hypercompact \hii region that *does*
%exhibit enhanced molecular emission in its surroundings


\subsubsection{Outflows}
\label{sec:outflowdiscussion}
While the outflows described in Section \ref{sec:outflows} are impressive and
plentiful, they are obviously not the dominant form of feedback, as their area
filling factor is small compared to that of the various forms of radiative
feedback.  A low area filling factor implies a substantially smaller volume
filling factor and therefore a lower overall effect on the cloud.  However,
these outflows likely do punch holes through protostellar envelopes and the
surrounding cloud material, allowing radiation to escape.

The detection of widespread high-J \methanol emission around the highest-mass
protostars suggests that the use of \methanol as a bulk outflow tracer as
suggested by \citet{Kristensen2015a} is not viable in regions with forming
high-mass stars.  While mid-J \methanol emission is detected associated with
the outflow (e.g., the J=10-9 transition), it is completely dominated by the
general `extended hot core' emission described in Section
\ref{sec:chemistrymaps}.




\subsection{The accreting phase of high-mass star formation}
The strong outflows observed around the highest-mass forming stars, e2e, e8,
and north are clear indications of ongoing accretion onto these sources.
However, the bright \hii regions, including e2w, e1, and d2, all lack any sign
of an outflow or a surrounding rotating molecular structure.  Most of these
sources lack any surrounding molecular material at all.
%; e2w is an exception, being embedded in the e2e core, but it shows signs
% that it may be escaping from the core, rather than being its focus

Some models of high-mass star formation suggest that accretion continues
through the ionized (\hii region) phase \citep{Keto2007a}.  The lack of
molecular material around the majority of the compact \hii regions in W51
suggests instead that most of the accretion is done by the time an \hii region
ignites.

There is one counterexample in our sample.  The source d2 is a bright, compact
\hii region, but it is also surrounded by a molecular enhancement.  However, it
does not appear to drive an outflow, so there is again no direct evidence of
ongoing accretion.

\subsection{Fragmentation: Jeans analysis}
Fragmentation is one of the critical problems in high-mass star formation.
Assuming typical initial conditions for molecular clouds, with temperatures of
order 10 K, gas is expected to fragment into sub-solar mass cores, preventing
material from accreting onto single high-mass stars \citep{Krumholz2015a}.
Even after high-mass stars successfully form, further fragmentation could
halt the growth of these stars and limit their final mass \citep{Peters2010a}.

Thermal Jeans fragmentation can be limited or suppressed entirely if the gas is
warm enough.  The high observed gas temperatures, $T\sim100-600$ K over
$\sim10^4$ AU, around the high mass protostars indicate that their radiative
feedback in the infrared has a dramatic effect on the gas.  

We examined the temperature structure around the highest-mass cores in Section
\ref{sec:temperature} and the mass structure in Section \ref{sec:radialmass}.
We put these together to measure the Jeans mass, $M_J = (\pi / 6) c_s^3 G^{-3/2}
\rho^{-1/2}$, in Figure \ref{fig:mjeans}

\Figure{figures/azimuthalaverage_radial_mj_of_TCH3OH_massivecores.png}
{The azimuthally averaged Jeans mass surrounding the three most massive cores.
We used the \methanol temperature from \ref{sec:methanol} in both the Jeans
mass calculation and the dust-based mass determination.  The Jeans mass
decreases toward the center at least in part because of the density increase,
while the temperature only rises slowly.}
{fig:mjeans}{1}{10cm}

% Expanding HII regions are more likely due to stars leaving their dense gas
% regions than blowing away their surroundings...





% useful questions:
% how much of the 'core' mass will end up in a single star?  single system?

\section{Conclusions}
We conclude.



\ifstandalone
\bibliographystyle{apj_w_etal}  % or "siam", or "alpha", or "abbrv"
%\bibliography{thesis}      % bib database file refs.bib
\bibliography{bibdesk}      % bib database file refs.bib
\fi


%\todo{idea: dust is more concentrated than gas (compare PSDs).  Can we do this to
%sims and determine which lines are thick?}

\appendix

\section{A bubble around e5}
There is evidence of a bubble in the continuum around e5 with a radius of
6.2\arcsec (0.16 pc; Figure \ref{fig:e5bubble}).  The bubble is completely
absent in the centimeter continuum, so the observed emission is from dust.  The
bubble edge can be seen from 58 \kms to 63 \kms in \ceighteeno and
\formaldehyde, though it is not continuous in any single velocity channel.
There is a collection of compact sources (protostars or cores) along the
southeast edge of the bubble.

The presence of such a bubble in dense gas, but its absence in ionizing gas, is
surprising.  The most likely mechanism for blowing such a bubble is ionizing
radiative feedback, especially around a source that is currently a hypercompact
HII region, but since no free-free emission is evident within or on the edge of
the bubble, it is at least not presently driving the bubble.  A plausible
explanation for this discrepancy is that e5 was an exposed O-star within the
past Myr, but has since begun accreting heavily and therefore had its HII
region shrunk.  This model is marginally supported by the presence of a `pillar'
of dense material pointing from e5 toward the south.

The total flux in the north half of the `bubble', which shows no signs of
free-free contamination, is about 1.5 Jy.  The implied mass in just this
fragment of the bubble is about $M\sim350$ \msun for a relatively high assumed
temperature $T=50$ K.  The total mass of the bubble is closer to $M\sim1000$
\msun, though it may be lower ($\sim500$ \msun) if the southern half is
dominated by free-free emission.

With such a large mass, the implied density of the original cloud, assuming it
was uniformly distributed over a 0.2 pc sphere, is $n(\hh) \approx 2-5\ee{5}$
\percc.

% this analysis courtesy Jim Dale
To evaluate the plausibility of the \hii-region origin of the bubble, we compare
to classical equations for \hii regions.
The Str\"omgren radius is \\
\begin{eqnarray}
R_{\rm s}=\left(\frac{3Q_{\rm H}}{4\pi\alpha_{\rm B} n^{2}}\right)^{\frac{1}{3}}.
\end{eqnarray} 
For $Q_{\rm H}\sim10^{49}$ \pers, $\alpha_{\rm B}=3\times10^{-13}$\,cm$^{3}$\,s$^{-1}$, $R_{\rm s}\approx0.01$\,pc.\\
\\
The Spitzer solution for HII region expansion gives\\
\begin{eqnarray}
R_{\rm HII}(t)=R_{\rm s}\left(1+\frac{7}{4}\frac{c_{\rm II}t}{R_{\rm s}}\right)^{\frac{4}{7}}.
\end{eqnarray} 
With $c_{\rm II}=7.5$\,km\,s$^{-1}$ and $t=10^{4}$\,yr,
$R_{\rm HII}(t)\approx0.04$\,pc, while at $t=10^5$\,yr, it is $R_{\rm
HII}\approx0.16$\,pc, which is comparable to the observed radius
($r_{obs} \sim 0.13-0.19$ pc)\\
\\
Whitworth et al. 1994 give the fragmentation timescale as
\begin{eqnarray}
t_{\rm frag}\sim1.56\left(\frac{c_{\rm s}}{0.2{\rm km\,s}^{-1}}\right)^{\frac{7}{11}}\left(\frac{Q_{\rm H}}{10^{49}{\rm s}^{-1}}\right)^{-\frac{1}{11}}\left(\frac{n}{10^{3}{\rm cm}^{-3}}\right)^{-\frac{5}{11}}{\rm Myr}.
\end{eqnarray} 
Plugging in our numbers gives $t_{\rm frag}\approx1.0\times10^{5}$\,yr, or
$10\times$ longer than the expansion time.\\
\\
% The corresponding radius at which fragmentation occurs is\\
% \begin{eqnarray}
% R_{\rm frag}\sim5.8\left(\frac{c_{\rm s}}{0.2{\rm km\,s}^{-1}}\right)^{\frac{4}{11}}\left(\frac{Q_{\rm H}}{10^{49}{\rm s}^{-1}}\right)^{\frac{1}{11}}\left(\frac{n}{10^{3}{\rm cm}^{-3}}\right)^{-\frac{6}{11}}{\rm pc},
% \end{eqnarray}
% which gives us $R_{\rm frag}\approx0.17$\,pc.\\

These values are consistent with a late O-type star having been exposed,
driving an \hii region, for $\sim10^4-10^5$ year, after which a substantial
increase in the accretion rate quenched the ionizing radiation from the star,
trapping it into a hypercompact ($r<0.005$ pc) configuration.  The
recombination timescale is short enough that the ionized gas would disappear
almost immediately after the continuous ionizing radiation source was hidden.
This is essentially the scenario laid out in \citet{de-Pree2014a} as an
explanation for the compact \hii region lifetime problem.  In this case,
however, it also seems that the \hii region has effectively driven the
``collect'' phase of what will presumably end in a collect-and-collapse style
triggering event.

Technically, it is possible that e5 actually represents an optically thick
high-mass-loss-rate wind rather than an ultracompact HII region, but I think we
can rule this out on physical plausibility considerations if we compare to 
wind models.  Note that $\eta$ Car would have a flux of $\sim0.5$ Jy at 2 cm
and $\sim5$ Jy at 1 mm at the distance of W51.

\FigureTwo{figures/e5_bubble.png}{figures/e5_bubble_robust2.png}
{The bubble around source e5.  The bubble interior shows no sign of centimeter
emission, though the lower-left region of the shell - just south of the
``cores'' - coincides with part of the W51 Main ionized shell.  The source of
the ionization is not obvious.
({\it Left}): A robust -2.0 image with a small (0.2\arcsec) beam and poor
recovery of large angular scale emission.  This image highlights the presence
of protostellar cores on the left edge of the bubble and along a filament just
south of the central source.
({\it Right}): A robust +2.0 image with a larger (0.4\arcsec) beam and better
recovery of large angular scales.  The contours show radio continuum (14.5 GHz)
emission at 1.5, 3, and 6 mJy/beam.  While some of the detected 1.4 mm emission
in the south could be free-free emission, the eastern and northern parts of the
shell show no emission down to the 50 $\mu$Jy noise level of the Ku-band map,
confirming that they consist only of dust emission.
}{fig:e5bubble}{1}{8cm}



\end{document}
