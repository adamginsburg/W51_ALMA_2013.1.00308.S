\documentclass{aa}
%\documentclass[defaultstyle,11pt]{thesis}
%\documentclass[]{report}
%\documentclass[]{article}
%\usepackage{aastex_hack}
%\usepackage{deluxetable}
%\documentclass[preprint]{aastex}
%\documentclass{aa}
\newcommand\arcdeg{\mbox{$^\circ$}\xspace} 

\pdfminorversion=4


%%%%%%%%%%%%%%%%%%%%%%%%%%%%%%%%%%%%%%%%%%%%%%%%%%%%%%%%%%%%%%%%
%%%%%%%%%%%  see documentation for information about  %%%%%%%%%%
%%%%%%%%%%%  the options (11pt, defaultstyle, etc.)   %%%%%%%%%%
%%%%%%%  http://www.colorado.edu/its/docs/latex/thesis/  %%%%%%%
%%%%%%%%%%%%%%%%%%%%%%%%%%%%%%%%%%%%%%%%%%%%%%%%%%%%%%%%%%%%%%%%
%		\documentclass[typewriterstyle]{thesis}
% 		\documentclass[modernstyle]{thesis}
% 		\documentclass[modernstyle,11pt]{thesis}
%	 	\documentclass[modernstyle,12pt]{thesis}

%%%%%%%%%%%%%%%%%%%%%%%%%%%%%%%%%%%%%%%%%%%%%%%%%%%%%%%%%%%%%%%%
%%%%%%%%%%%    load any packages which are needed    %%%%%%%%%%%
%%%%%%%%%%%%%%%%%%%%%%%%%%%%%%%%%%%%%%%%%%%%%%%%%%%%%%%%%%%%%%%%
\usepackage{latexsym}		% to get LASY symbols
\usepackage{graphicx}		% to insert PostScript figures
%\usepackage{deluxetable}
\usepackage{rotating}		% for sideways tables/figures
\usepackage{natbib}  % Requires natbib.sty, available from http://ads.harvard.edu/pubs/bibtex/astronat/
\usepackage{savesym}
\usepackage{pdflscape}
\usepackage{amssymb}
\usepackage{morefloats}
%\savesymbol{singlespace}
\savesymbol{doublespace}
%\usepackage{wrapfig}
%\usepackage{setspace}
\usepackage{xspace}
\usepackage{color}
\usepackage{multicol}
\usepackage{mdframed}
\usepackage{url}
\usepackage{subfigure}
%\usepackage{emulateapj}
\usepackage{lscape}
\usepackage{grffile}
\usepackage{standalone}
\standalonetrue
\usepackage{import}
\usepackage[utf8]{inputenc}
\usepackage{longtable}
\usepackage{booktabs}
\usepackage[yyyymmdd,hhmmss]{datetime}
\usepackage{fancyhdr}
\usepackage[colorlinks=true,citecolor=blue,linkcolor=cyan]{hyperref}
\usepackage{ifpdf}






%\renewcommand\ion[2]{#1$\;${%
%\ifx\@currsize\normalsize\small \else
%\ifx\@currsize\small\footnotesize \else
%\ifx\@currsize\footnotesize\scriptsize \else
%\ifx\@currsize\scriptsize\tiny \else
%\ifx\@currsize\large\normalsize \else
%\ifx\@currsize\Large\large
%\fi\fi\fi\fi\fi\fi
%\rmfamily\@Roman{#2}}\relax}% 

\newcommand{\paa}{Pa\ensuremath{\alpha}}
\newcommand{\brg}{Br\ensuremath{\gamma}}
\newcommand{\msun}{\ensuremath{M_{\odot}}\xspace}			%  Msun
\newcommand{\mdot}{\ensuremath{\dot{M}}\xspace}
\newcommand{\lsun}{\ensuremath{L_{\odot}}\xspace}			%  Lsun
\newcommand{\rsun}{\ensuremath{R_{\odot}}\xspace}			%  Rsun
\newcommand{\lbol}{\ensuremath{L_{\mathrm{bol}}\xspace}}	%  Lbol
\newcommand{\ks}{K\ensuremath{_{\mathrm{s}}}}		%  Ks
\newcommand{\hh}{\ensuremath{\textrm{H}_{2}}\xspace}			%  H2
\newcommand{\dens}{\ensuremath{n(\hh) [\percc]}\xspace}
\newcommand{\formaldehyde}{\ensuremath{\textrm{H}_2\textrm{CO}}\xspace}
\newcommand{\formamide}{\ensuremath{\textrm{NH}_2\textrm{CHO}}\xspace}
\newcommand{\formaldehydeIso}{\ensuremath{\textrm{H}_2~^{13}\textrm{CO}}\xspace}
\newcommand{\methanol}{\ensuremath{\textrm{CH}_3\textrm{OH}}\xspace}
\newcommand{\ortho}{\ensuremath{\textrm{o-H}_2\textrm{CO}}\xspace}
\newcommand{\para}{\ensuremath{\textrm{p-H}_2\textrm{CO}}\xspace}
\newcommand{\oneone}{\ensuremath{1_{1,0}-1_{1,1}}\xspace}
\newcommand{\twotwo}{\ensuremath{2_{1,1}-2_{1,2}}\xspace}
\newcommand{\threethree}{\ensuremath{3_{1,2}-3_{1,3}}\xspace}
\newcommand{\threeohthree}{\ensuremath{3_{0,3}-2_{0,2}}\xspace}
\newcommand{\threetwotwo}{\ensuremath{3_{2,2}-2_{2,1}}\xspace}
\newcommand{\threetwoone}{\ensuremath{3_{2,1}-2_{2,0}}\xspace}
\newcommand{\fourtwotwo}{\ensuremath{4_{2,2}-3_{1,2}}\xspace} % CH3OH 218.4 GHz
\newcommand{\methylcyanide}{\ensuremath{\textrm{CH}_{3}\textrm{CN}}\xspace}
\newcommand{\ketene}{\ensuremath{\textrm{H}_{2}\textrm{CCO}}\xspace}
\newcommand{\ethylcyanide}{\ensuremath{\textrm{CH}_3\textrm{CH}_2\textrm{CN}}\xspace}
\newcommand{\cyanoacetylene}{\ensuremath{\textrm{HC}_{3}\textrm{N}}\xspace}
\newcommand{\methylformate}{\ensuremath{\textrm{CH}_{3}\textrm{OCHO}}\xspace}
\newcommand{\dimethylether}{\ensuremath{\textrm{CH}_{3}\textrm{OCH}_{3}}\xspace}
\newcommand{\gaucheethanol}{\ensuremath{\textrm{g-CH}_3\textrm{CH}_2\textrm{OH}}\xspace}
\newcommand{\acetone}{\ensuremath{\left[\textrm{CH}_{3}\right]_2\textrm{CO}}\xspace}
\newcommand{\methyleneamidogen}{\ensuremath{\textrm{H}_{2}\textrm{CN}}\xspace}
\newcommand{\Rone}{\ensuremath{\para~S_{\nu}(\threetwoone) / S_{\nu}(\threeohthree)}\xspace}
\newcommand{\Rtwo}{\ensuremath{\para~S_{\nu}(\threetwotwo) / S_{\nu}(\threetwoone)}\xspace}
\newcommand{\JKaKc}{\ensuremath{J_{K_a K_c}}}
\newcommand{\water}{H$_{2}$O\xspace}		%  H2O
\newcommand{\feii}{\ion{Fe}{2}}		%  FeII
\newcommand{\uchii}{\ion{UCH}{ii}\xspace}
\newcommand{\UCHII}{\ion{UCH}{ii}\xspace}
\newcommand{\hchii}{\ion{HCH}{ii}\xspace}
\newcommand{\HCHII}{\ion{HCH}{ii}\xspace}
\newcommand{\hii}{\ion{H}{ii}\xspace}
\newcommand{\hi}{H~{\sc i}\xspace}
\newcommand{\Hii}{\hii}
\newcommand{\HII}{\hii}
\newcommand{\Xform}{\ensuremath{X_{\formaldehyde}}}
\newcommand{\kms}{\textrm{km~s}\ensuremath{^{-1}}\xspace}	%  km s-1
\newcommand{\nsample}{456\xspace}
\newcommand{\CFR}{5\xspace} % nMPC / 0.25 / 2 (6 for W51 once, 8 for W51 twice) REFEDIT: With f_observed=0.3, becomes 3/2./0.3 = 5
\newcommand{\permyr}{\ensuremath{\mathrm{Myr}^{-1}}\xspace}
\newcommand{\pers}{\ensuremath{\mathrm{s}^{-1}}\xspace}
\newcommand{\tsuplim}{0.5\xspace} % upper limit on starless timescale
\newcommand{\ncandidates}{18\xspace}
\newcommand{\mindist}{8.7\xspace}
\newcommand{\rcluster}{2.5\xspace}
\newcommand{\ncomplete}{13\xspace}
\newcommand{\middistcut}{13.0\xspace}
\newcommand{\nMPC}{3\xspace} % only count W51 once.  W51, W49, G010
\newcommand{\obsfrac}{30}
\newcommand{\nMPCtot}{10\xspace} % = nmpc / obsfrac
\newcommand{\nMPCtoterr}{6\xspace} % = sqrt(nmpc) / obsfrac
\newcommand{\plaw}{2.1\xspace}
\newcommand{\plawerr}{0.3\xspace}
\newcommand{\mmin}{\ensuremath{10^4~\msun}\xspace}
%\newcommand{\perkmspc}{\textrm{per~km~s}\ensuremath{^{-1}}\textrm{pc}\ensuremath{^{-1}}\xspace}	%  km s-1 pc-1
\newcommand{\kmspc}{\textrm{km~s}\ensuremath{^{-1}}\textrm{pc}\ensuremath{^{-1}}\xspace}	%  km s-1 pc-1
\newcommand{\sqcm}{cm$^{2}$\xspace}		%  cm^2
\newcommand{\percc}{\ensuremath{\textrm{cm}^{-3}}\xspace}
\newcommand{\perpc}{\ensuremath{\textrm{pc}^{-1}}\xspace}
\newcommand{\persc}{\ensuremath{\textrm{cm}^{-2}}\xspace}
\newcommand{\persr}{\ensuremath{\textrm{sr}^{-1}}\xspace}
\newcommand{\peryr}{\ensuremath{\textrm{yr}^{-1}}\xspace}
\newcommand{\perkmspc}{\textrm{km~s}\ensuremath{^{-1}}\textrm{pc}\ensuremath{^{-1}}\xspace}	%  km s-1 pc-1
\newcommand{\perkms}{\textrm{per~km~s}\ensuremath{^{-1}}\xspace}	%  km s-1 
\newcommand{\um}{\ensuremath{\mu \textrm{m}}\xspace}    % micron
\newcommand{\microjy}{\ensuremath{\mu\textrm{Jy}}\xspace}    % micron
\newcommand{\mum}{\um}
\newcommand{\htwo}{\ensuremath{\textrm{H}_2}}
\newcommand{\Htwo}{\ensuremath{\textrm{H}_2}}
\newcommand{\HtwoO}{\ensuremath{\textrm{H}_2\textrm{O}}}
\newcommand{\htwoo}{\ensuremath{\textrm{H}_2\textrm{O}}}
\newcommand{\ha}{\ensuremath{\textrm{H}\alpha}}
\newcommand{\hb}{\ensuremath{\textrm{H}\beta}}
\newcommand{\so}{SO~\ensuremath{5_6-4_5}\xspace}
\newcommand{\SO}{SO~\ensuremath{1_2-1_1}\xspace}
\newcommand{\ammonia}{NH\ensuremath{_3}\xspace}
\newcommand{\twelveco}{\ensuremath{^{12}\textrm{CO}}\xspace}
\newcommand{\thirteenco}{\ensuremath{^{13}\textrm{CO}}\xspace}
\newcommand{\ceighteeno}{\ensuremath{\textrm{C}^{18}\textrm{O}}\xspace}
\def\ee#1{\ensuremath{\times10^{#1}}}
\newcommand{\degrees}{\ensuremath{^{\circ}}}
% can't have \degree because I'm getting a degree...
\newcommand{\lowirac}{800}
\newcommand{\highirac}{8000}
\newcommand{\lowmips}{600}
\newcommand{\highmips}{5000}
\newcommand{\perbeam}{\ensuremath{\textrm{beam}^{-1}}}
\newcommand{\ds}{\ensuremath{\textrm{d}s}}
\newcommand{\dnu}{\ensuremath{\textrm{d}\nu}}
\newcommand{\dv}{\ensuremath{\textrm{d}v}}
\def\secref#1{Section \ref{#1}}
\def\eqref#1{Equation \ref{#1}}
\def\facility#1{#1}
%\newcommand{\arcmin}{'}

\newcommand{\necluster}{Sh~2-233IR~NE}
\newcommand{\swcluster}{Sh~2-233IR~SW}
\newcommand{\region}{IRAS 05358}

\newcommand{\nwfive}{40}
\newcommand{\nouter}{15}

\newcommand{\vone}{{\rm v}1.0\xspace}
\newcommand{\vtwo}{{\rm v}2.0\xspace}
\newcommand\mjysr{\ensuremath{{\rm MJy~sr}^{-1}}}
\newcommand\jybm{\ensuremath{{\rm Jy~bm}^{-1}}}
\newcommand\nbolocat{8552\xspace}
\newcommand\nbolocatnew{548\xspace}
\newcommand\nbolocatnonew{8004\xspace} % = nbolocat-nbolocatnew
\renewcommand\arcdeg{\mbox{$^\circ$}\xspace} 
\renewcommand\arcmin{\mbox{$^\prime$}\xspace} 
\renewcommand\arcsec{\mbox{$^{\prime\prime}$}\xspace} 

\newcommand{\todo}[1]{\textcolor{red}{#1}}
\newcommand{\okinfinal}[1]{{#1}}
%% only needed if not aastex
%\newcommand{\keywords}[1]{}
%\newcommand{\email}[1]{}
%\newcommand{\affil}[1]{}


%aastex hack
%\newcommand\arcdeg{\mbox{$^\circ$}}%
%\newcommand\arcmin{\mbox{$^\prime$}\xspace}%
%\newcommand\arcsec{\mbox{$^{\prime\prime}$}\xspace}%

%\newcommand\epsscale[1]{\gdef\eps@scaling{#1}}
%
%\newcommand\plotone[1]{%
% \typeout{Plotone included the file #1}
% \centering
% \leavevmode
% \includegraphics[width={\eps@scaling\columnwidth}]{#1}%
%}%
%\newcommand\plottwo[2]{{%
% \typeout{Plottwo included the files #1 #2}
% \centering
% \leavevmode
% \columnwidth=.45\columnwidth
% \includegraphics[width={\eps@scaling\columnwidth}]{#1}%
% \hfil
% \includegraphics[width={\eps@scaling\columnwidth}]{#2}%
%}}%


%\newcommand\farcm{\mbox{$.\mkern-4mu^\prime$}}%
%\let\farcm\farcm
%\newcommand\farcs{\mbox{$.\!\!^{\prime\prime}$}}%
%\let\farcs\farcs
%\newcommand\fp{\mbox{$.\!\!^{\scriptscriptstyle\mathrm p}$}}%
%\newcommand\micron{\mbox{$\mu$m}}%
%\def\farcm{%
% \mbox{.\kern -0.7ex\raisebox{.9ex}{\scriptsize$\prime$}}%
%}%
%\def\farcs{%
% \mbox{%
%  \kern  0.13ex.%
%  \kern -0.95ex\raisebox{.9ex}{\scriptsize$\prime\prime$}%
%  \kern -0.1ex%
% }%
%}%

\def\Figure#1#2#3#4#5{
\begin{figure*}[!htp]
\includegraphics[scale=#4,width=#5]{#1}
\caption{#2}
\label{#3}
\end{figure*}
}

\def\WrapFigure#1#2#3#4#5#6{
\begin{wrapfigure}{#6}{0.5\textwidth}
\includegraphics[scale=#4,width=#5]{#1}
\caption{#2}
\label{#3}
\end{wrapfigure}
}

% % #1 - filename
% % #2 - caption
% % #3 - label
% % #4 - epsscale
% % #5 - R or L?
% \def\WrapFigure#1#2#3#4#5#6{
% \begin{wrapfigure}[#6]{#5}{0.45\textwidth}
% %  \centercaption
% %  \vspace{-14pt}
%   \epsscale{#4}
%   \includegraphics[scale=#4]{#1}
%   \caption{#2}
%   \label{#3}
% \end{wrapfigure}
% }

\def\RotFigure#1#2#3#4#5{
\begin{sidewaysfigure*}[!htp]
\includegraphics[scale=#4,width=#5]{#1}
\caption{#2}
\label{#3}
\end{sidewaysfigure*}
}

\def\FigureSVG#1#2#3#4{
\begin{figure*}[!htp]
    \def\svgwidth{#4}
    \input{#1}
    \caption{#2}
    \label{#3}
\end{figure*}
}

% originally intended to be included in a two-column paper
% this is in includegraphics: ,width=3in
% but, not for thesis
\def\OneColFigure#1#2#3#4#5{
\begin{figure}[!htpb]
\epsscale{#4}
\includegraphics[scale=#4,angle=#5]{#1}
\caption{#2}
\label{#3}
\end{figure}
}

\def\SubFigure#1#2#3#4#5{
\begin{figure*}[!htp]
\addtocounter{figure}{-1}
\epsscale{#4}
\includegraphics[angle=#5]{#1}
\caption{#2}
\label{#3}
\end{figure*}
}

%\def\FigureTwo#1#2#3#4#5{
%\begin{figure*}[!htp]
%\epsscale{#5}
%\plottwo{#1}{#2}
%\caption{#3}
%\label{#4}
%\end{figure*}
%}

\def\FigureTwo#1#2#3#4#5#6{
\begin{figure*}[!htp]
\subfigure[]{ \includegraphics[scale=#5,width=#6]{#1} }
\subfigure[]{ \includegraphics[scale=#5,width=#6]{#2} }
\caption{#3}
\label{#4}
\end{figure*}
}

\def\FigureTwoAA#1#2#3#4#5#6{
\begin{figure*}[!htp]
\subfigure[]{ \includegraphics[scale=#5,width=#6]{#1} }
\subfigure[]{ \includegraphics[scale=#5,width=#6]{#2} }
\caption{#3}
\label{#4}
\end{figure*}
}

\newenvironment{rotatepage}%
{}{}
   %{\pagebreak[4]\afterpage\global\pdfpageattr\expandafter{\the\pdfpageattr/Rotate 90}}%
   %{\pagebreak[4]\afterpage\global\pdfpageattr\expandafter{\the\pdfpageattr/Rotate 0}}%


\def\RotFigureTwoAA#1#2#3#4#5#6{
\begin{rotatepage}
\begin{sidewaysfigure*}[!htp]
\subfigure[]{ \includegraphics[scale=#5,width=#6]{#1} }
\\
\subfigure[]{ \includegraphics[scale=#5,width=#6]{#2} }
\caption{#3}
\label{#4}
\end{sidewaysfigure*}
\end{rotatepage}
}

\def\RotFigureThreeAA#1#2#3#4#5#6#7{
\begin{rotatepage}
\begin{sidewaysfigure*}[!htp]
\subfigure[]{ \includegraphics[scale=#6,width=#7]{#1} }
\\
\subfigure[]{ \includegraphics[scale=#6,width=#7]{#2} }
\\
\subfigure[]{ \includegraphics[scale=#6,width=#7]{#3} }
\caption{#4}
\label{#5}
\end{sidewaysfigure*}
\end{rotatepage}
\clearpage
}

\def\FigureThreeAA#1#2#3#4#5#6#7{
\begin{figure*}[!htp]
\subfigure[]{ \includegraphics[scale=#6,width=#7]{#1} }
\subfigure[]{ \includegraphics[scale=#6,width=#7]{#2} }
\subfigure[]{ \includegraphics[scale=#6,width=#7]{#3} }
\caption{#4}
\label{#5}
\end{figure*}
}



\def\SubFigureTwo#1#2#3#4#5{
\begin{figure*}[!htp]
\addtocounter{figure}{-1}
\epsscale{#5}
\plottwo{#1}{#2}
\caption{#3}
\label{#4}
\end{figure*}
}

\def\FigureFour#1#2#3#4#5#6{
\begin{figure*}[!htp]
\subfigure[]{ \includegraphics[width=3in]{#1} }
\subfigure[]{ \includegraphics[width=3in]{#2} }
\subfigure[]{ \includegraphics[width=3in]{#3} }
\subfigure[]{ \includegraphics[width=3in]{#4} }
\caption{#5}
\label{#6}
\end{figure*}
}

\def\FigureFourPDF#1#2#3#4#5#6{
\begin{figure*}[!htp]
\subfigure[]{ \includegraphics[width=3in,type=pdf,ext=.pdf,read=.pdf]{#1} }
\subfigure[]{ \includegraphics[width=3in,type=pdf,ext=.pdf,read=.pdf]{#2} }
\subfigure[]{ \includegraphics[width=3in,type=pdf,ext=.pdf,read=.pdf]{#3} }
\subfigure[]{ \includegraphics[width=3in,type=pdf,ext=.pdf,read=.pdf]{#4} }
\caption{#5}
\label{#6}
\end{figure*}
}

\def\FigureThreePDF#1#2#3#4#5{
\begin{figure*}[!htp]
\subfigure[]{ \includegraphics[width=3in,type=pdf,ext=.pdf,read=.pdf]{#1} }
\subfigure[]{ \includegraphics[width=3in,type=pdf,ext=.pdf,read=.pdf]{#2} }
\subfigure[]{ \includegraphics[width=3in,type=pdf,ext=.pdf,read=.pdf]{#3} }
\caption{#4}
\label{#5}
\end{figure*}
}

\def\Table#1#2#3#4#5{
%\renewcommand{\thefootnote}{\alph{footnote}}
\begin{table}
\caption{#2}
\label{#3}
    \begin{tabular}{#1}
        \hline\hline
        #4
        \hline
        #5
        \hline
    \end{tabular}
\end{table}
%\renewcommand{\thefootnote}{\arabic{footnote}}
}


%\def\Table#1#2#3#4#5#6{
%%\renewcommand{\thefootnote}{\alph{footnote}}
%\begin{deluxetable}{#1}
%\tablewidth{0pt}
%\tabletypesize{\footnotesize}
%\tablecaption{#2}
%\tablehead{#3}
%\startdata
%\label{#4}
%#5
%\enddata
%\bigskip
%#6
%\end{deluxetable}
%%\renewcommand{\thefootnote}{\arabic{footnote}}
%}

%\def\tablenotetext#1#2{
%\footnotetext[#1]{#2}
%}

\def\LongTable#1#2#3#4#5#6#7#8{
% required to get tablenotemark to work: http://www2.astro.psu.edu/users/stark/research/psuthesis/longtable.html
\renewcommand{\thefootnote}{\alph{footnote}}
\begin{longtable}{#1}
\caption[#2]{#2}
\label{#4} \\

 \\
\hline 
#3 \\
\hline
\endfirsthead

\hline
#3 \\
\hline
\endhead

\hline
\multicolumn{#8}{r}{{Continued on next page}} \\
\hline
\endfoot

\hline 
\endlastfoot
#7 \\

#5
\hline
#6 \\

\end{longtable}
\renewcommand{\thefootnote}{\arabic{footnote}}
}

\def\TallFigureTwo#1#2#3#4#5#6{
\begin{figure*}[htp]
\epsscale{#5}
\subfigure[]{ \includegraphics[width=#6]{#1} }
\subfigure[]{ \includegraphics[width=#6]{#2} }
\caption{#3}
\label{#4}
\end{figure*}
}

		% file containing author's macro definitions

\begin{document}
\title{Cores and Outflows in the W51 Protoclusters}
"A multi-phase outflow from a high-mass protostar"


\section{Observations}
As part of ALMA Cycle 2 program 2013.1.00308.S, we observed a
$\sim2\arcmin\times1\arcmin$ region centered between W51 IRS2 and W51 e1/e2
with a 37-pointing mosaic.  Two configurations of the 12m array were used,
achieving a resolution of 0.2\arcsec.  Additionally, a 12-pointing mosaic was
performed using the 7m array, probing scales up to XXX\arcsec.  The full UV
coverage was from XX to YY m.

Data reduction was performed using CASA.  The QA2-produced data products were
combined using the standard inverse variance weighting (check this).
The visibilities were imaged into full spectral cubes at coarse (0.5\arcsec)
resolution in order to get a first look at all 15630 spectral channels.  The
resulting cube was used to identify bright lines in the spectrum extracted
from source e8.  To produce continuum images, frequency channels including
bright lines were excluded.

The spectral setups were ...

A continuum image combining all 4 spectral windows was produced using
\texttt{tclean}.  We phase self-calibrated the image on baselines longer than
100m to increase the dynamic range.  The final image was cleaned with 50000
iterations to a threshold of 5 mJy.  The lowest noise level in the image, away
from bright sources, is $\sim0.2$ mJy/beam, but near the bright sources e2 and
IRS2, the noise reached as high as $\sim2$ mJy/beam.  Deeper cleaning was
attempted, but lead to instabilities.

\subsection{Simulations}
The enhanced noise around bright sources is unavoidable.  We tested the noise
properties of our data set using the CASA \texttt{simobserve} toolkit.  We
obtained a Herschel Gould's Belt Survey image of the Perseus molecular cloud at
250 \um \citep[resolution 18\arcsec][]{} and scaled it down by $\sim40\times$
to match the resolution of our ALMA data.  We used the \texttt{sm.predict} task
to ``observe'' the Herschel data with our exact UV data set.  We then used
\texttt{sm.corrupt} to make the noise properties approximately match those of
our observations.

We performed two simulations to test the observability of the Perseus data.
First, we scaled the surface brightness at 250 \um to 1100 \um by assuming a
blackbody with $T=20$ K and opacity index $\beta=1.5$.  This resulted in only
weak detections of the brightest 3 sources in Perseus, and the noise properties
of the map were excellent and uniform.

Second, we scaled the peak flux density of the Perseus map to be
$\sim100\times$ brighter, or comparable to the flux density of W51e2 in the
real ALMA observations.  In this map, even with deep cleaning, the noise around
the bright sources remains very high.



\section{Analysis}
\subsection{Source Identification}
% dendrogramming.py
We used the \texttt{dendrogram} method described by \citet{Rosolowsky2008a} and
implemented in \texttt{astrodendro} to identify sources.  We used a minimum
value of 1 mJy/beam ($\sim5-\sigma$) and a minimum $\Delta=0.4$ mJy/beam
($\sim2-\sigma$) with minimum 10 pixels (each pixel is 0.05\arcsec).  This
cataloging yielded over 8000 candidate sources, of which the majority are noise
or artifacts around the brightest sources.  To filter out these bad sources,
we created a noise map taking the local RMS of the \texttt{tclean}-produced
residual map over a $\sigma=30$ pixel (1.5\arcsec) gaussian.  We then removed
all sources with peak S/N < 8, mean S/N per pixel $< 5$, and minimum S/N per
pixel $ < 1$.  We also only included the smallest sources in the dendrogram,
the ``leaves''.  These parameters were tuned by checking against ``real''
sources identified by eye and selected using \texttt{ds9}: most real sources are
recovered (but not all; see Section ...) and few spurious sources ($<10$) are
included.  The resulting catalog includes 113 sources.

\subsection{Photometry}
We created a catalog of the dendrogram sources including their peak and mean
flux density, their centroid, and their geometric properties.  For each source,
we further extracted aperture photometry around the centroid in 6 apertures:
0.2, 0.4, 0.6, 0.8, 1.0, and 1.5\arcsec.  We performed the same aperture
photometry on the W51 Ku-band images from \citet{Ginsburg2016a}.  These
observations are reported in Table {...}.

The source flux density distribution is shown in Figure
\ref{fig:fluxhistograms}.  The most common nearest-neighbor separation between
cataloged cores is $\sim0.3\arcsec$, which implies that the larger apertures
double-count some pixels.  The smallest separation is
0.26\arcsec, so the 0.2\arcsec aperture contains only unique pixels.

\Figure{figures/core_flux_histogram_apertureradius.png}
{Histograms of the core fluxes measured with circular apertures centered
on the dendrogram-extracted core centroids.  The aperture size is listed 
in the y-axis label.  Free-free-dominated sources are excluded.}
{fig:fluxhistograms}
{1}{10cm}

\subsection{Spectral Lines \& Velocities}
To determine the line-of-sight velocity of each source, we extracted a spectrum
from an 0.5\arcsec aperture centered on the source and from a 0.5-1.0\arcsec
annulus around it.  We then searched each spectrum for the brightest pixel and
associated it with the likeliest spectral line.  We repeated this in each of
our 4 spectral windows, then averaged the 4 velocities to get an estimate of
the source velocity.   This process also allowed us to identify the brightest
lines in each window and the brightest overall line observed, which we use
later for temperature estimation.

\subsection{The most massive protostellar cores in W51}
In Figure \ref{fig:hmradprof}, we show the radial profiles extracted from the
three high-mass protostellar cores in W51: W51 North, W51 e2e, and W51 e8.
The plot shows the enclosed mass out to $\sim1\arcsec$, which is ???the limit
of our recovered spatial scales??? / ???the limit of where a `core' might reasonably
be defined???  

All three sources show similar radial profiles, containing up to 4000 \msun
within a very compact radius of 5400 AU (0.03 pc).  However, the temperature
structure within these sources is certainly not homogeneous, and very likely a
large fraction of the total flux comes from $T\sim300$ K heated material
\citep{Goddi2016a}.  If the observed dust were all at 300 K, the mass would
be $\sim20\times$ lower, 300 \msun, which we treat as a strict lower bound
as it is unlikely that the dust more than $\gtrsim1000$ au is so warm.

\Figure{figures/cumulative_radial_flux_massivecores.png}
{The cumulative flux density radial profiles centered on three
massive protostellar cores.  They share similar profiles and
are likely dominated by hot dust in their innermost regions,
but they are more likely to be dominated by cooler dust in their
outer, more massive regions.}
{fig:hmradprof}{1}{10cm}

\subsection{The most massive prestellar cores in W51}
To place limits on the most massive prestellar cores, we need to know the
temperature of the dust in all of the bright ($>15$ mJy) sources.  We do not
have any direct means of evaluating the dust temperature, but we can infer at
least a lower limit on it by determining the peak brightness temperature of an
optically thick line that is excited at densities $n\gtrsim10^5$ \percc, at
which the dust and gas are coupled.

To accomplish this, we have found the brightest lines across the full $\sim6
GHz$ spectra and measured their peak brightness temperature.


\subsection{The mass budget on different spatial scales}
An important evolutionary indicator is the amount of mass at a given density; a
more evolved (more efficiently star-forming) region will have more mass at high
densities.  We cannot measure the dense gas fraction directly, but the amount of
flux density recovered by an interferometer provides a reasonable approximation.

% total_mass_analysis
For the ``total'' flux density in the region, we use the Bolocam Galactic Plane
Survey observations \citep{Aguirre2011a,Ginsburg2013a}, which are the closest
in frequency single-dish millimeter data available.  We assume a spectral index
$\alpha=3.5$ to convert the BGPS flux density measurements at 271.4 GHz to the
mean ALMA frequency of 226.6 GHz.  The ALMA data have a total flux 23.2 Jy above
a very conservative threshold of 10 mJy/beam in our
mosaic; in the same area the BGPS data have a flux of 144 Jy, which scales down to
76.5 Jy.  The recovery fraction is 30$\pm3$\%, where the error bar accounts
for a change in $\alpha\pm0.5$.  The threshold of 10 mJy/beam corresponds to
a column threshold $N>10^{25}$ \percc for 20 K dust.  For an unresolved spherical source,
this corresponds to a volume density $n>10^{8.1}$ \percc.

- largest angular scale in ALMA image

\end{document}
