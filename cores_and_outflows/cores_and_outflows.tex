\documentclass{aa}
%\documentclass[defaultstyle,11pt]{thesis}
%\documentclass[]{report}
%\documentclass[]{article}
%\usepackage{aastex_hack}
%\usepackage{deluxetable}
%\documentclass[preprint]{aastex}
%\documentclass{aa}
\newcommand\arcdeg{\mbox{$^\circ$}\xspace} 

\pdfminorversion=4


%%%%%%%%%%%%%%%%%%%%%%%%%%%%%%%%%%%%%%%%%%%%%%%%%%%%%%%%%%%%%%%%
%%%%%%%%%%%  see documentation for information about  %%%%%%%%%%
%%%%%%%%%%%  the options (11pt, defaultstyle, etc.)   %%%%%%%%%%
%%%%%%%  http://www.colorado.edu/its/docs/latex/thesis/  %%%%%%%
%%%%%%%%%%%%%%%%%%%%%%%%%%%%%%%%%%%%%%%%%%%%%%%%%%%%%%%%%%%%%%%%
%		\documentclass[typewriterstyle]{thesis}
% 		\documentclass[modernstyle]{thesis}
% 		\documentclass[modernstyle,11pt]{thesis}
%	 	\documentclass[modernstyle,12pt]{thesis}

%%%%%%%%%%%%%%%%%%%%%%%%%%%%%%%%%%%%%%%%%%%%%%%%%%%%%%%%%%%%%%%%
%%%%%%%%%%%    load any packages which are needed    %%%%%%%%%%%
%%%%%%%%%%%%%%%%%%%%%%%%%%%%%%%%%%%%%%%%%%%%%%%%%%%%%%%%%%%%%%%%
\usepackage{latexsym}		% to get LASY symbols
\usepackage{graphicx}		% to insert PostScript figures
%\usepackage{deluxetable}
\usepackage{rotating}		% for sideways tables/figures
\usepackage{natbib}  % Requires natbib.sty, available from http://ads.harvard.edu/pubs/bibtex/astronat/
\usepackage{savesym}
\usepackage{pdflscape}
\usepackage{amssymb}
\usepackage{morefloats}
%\savesymbol{singlespace}
\savesymbol{doublespace}
%\usepackage{wrapfig}
%\usepackage{setspace}
\usepackage{xspace}
\usepackage{color}
\usepackage{multicol}
\usepackage{mdframed}
\usepackage{url}
\usepackage{subfigure}
%\usepackage{emulateapj}
\usepackage{lscape}
\usepackage{grffile}
\usepackage{standalone}
\standalonetrue
\usepackage{import}
\usepackage[utf8]{inputenc}
\usepackage{longtable}
\usepackage{booktabs}
\usepackage[yyyymmdd,hhmmss]{datetime}
\usepackage{fancyhdr}
\usepackage[colorlinks=true,citecolor=blue,linkcolor=cyan]{hyperref}
\usepackage{ifpdf}






%\renewcommand\ion[2]{#1$\;${%
%\ifx\@currsize\normalsize\small \else
%\ifx\@currsize\small\footnotesize \else
%\ifx\@currsize\footnotesize\scriptsize \else
%\ifx\@currsize\scriptsize\tiny \else
%\ifx\@currsize\large\normalsize \else
%\ifx\@currsize\Large\large
%\fi\fi\fi\fi\fi\fi
%\rmfamily\@Roman{#2}}\relax}% 

\newcommand{\paa}{Pa\ensuremath{\alpha}}
\newcommand{\brg}{Br\ensuremath{\gamma}}
\newcommand{\msun}{\ensuremath{M_{\odot}}\xspace}			%  Msun
\newcommand{\mdot}{\ensuremath{\dot{M}}\xspace}
\newcommand{\lsun}{\ensuremath{L_{\odot}}\xspace}			%  Lsun
\newcommand{\rsun}{\ensuremath{R_{\odot}}\xspace}			%  Rsun
\newcommand{\lbol}{\ensuremath{L_{\mathrm{bol}}\xspace}}	%  Lbol
\newcommand{\ks}{K\ensuremath{_{\mathrm{s}}}}		%  Ks
\newcommand{\hh}{\ensuremath{\textrm{H}_{2}}\xspace}			%  H2
\newcommand{\dens}{\ensuremath{n(\hh) [\percc]}\xspace}
\newcommand{\formaldehyde}{\ensuremath{\textrm{H}_2\textrm{CO}}\xspace}
\newcommand{\formamide}{\ensuremath{\textrm{NH}_2\textrm{CHO}}\xspace}
\newcommand{\formaldehydeIso}{\ensuremath{\textrm{H}_2~^{13}\textrm{CO}}\xspace}
\newcommand{\methanol}{\ensuremath{\textrm{CH}_3\textrm{OH}}\xspace}
\newcommand{\ortho}{\ensuremath{\textrm{o-H}_2\textrm{CO}}\xspace}
\newcommand{\para}{\ensuremath{\textrm{p-H}_2\textrm{CO}}\xspace}
\newcommand{\oneone}{\ensuremath{1_{1,0}-1_{1,1}}\xspace}
\newcommand{\twotwo}{\ensuremath{2_{1,1}-2_{1,2}}\xspace}
\newcommand{\threethree}{\ensuremath{3_{1,2}-3_{1,3}}\xspace}
\newcommand{\threeohthree}{\ensuremath{3_{0,3}-2_{0,2}}\xspace}
\newcommand{\threetwotwo}{\ensuremath{3_{2,2}-2_{2,1}}\xspace}
\newcommand{\threetwoone}{\ensuremath{3_{2,1}-2_{2,0}}\xspace}
\newcommand{\fourtwotwo}{\ensuremath{4_{2,2}-3_{1,2}}\xspace} % CH3OH 218.4 GHz
\newcommand{\methylcyanide}{\ensuremath{\textrm{CH}_{3}\textrm{CN}}\xspace}
\newcommand{\ketene}{\ensuremath{\textrm{H}_{2}\textrm{CCO}}\xspace}
\newcommand{\ethylcyanide}{\ensuremath{\textrm{CH}_3\textrm{CH}_2\textrm{CN}}\xspace}
\newcommand{\cyanoacetylene}{\ensuremath{\textrm{HC}_{3}\textrm{N}}\xspace}
\newcommand{\methylformate}{\ensuremath{\textrm{CH}_{3}\textrm{OCHO}}\xspace}
\newcommand{\dimethylether}{\ensuremath{\textrm{CH}_{3}\textrm{OCH}_{3}}\xspace}
\newcommand{\gaucheethanol}{\ensuremath{\textrm{g-CH}_3\textrm{CH}_2\textrm{OH}}\xspace}
\newcommand{\acetone}{\ensuremath{\left[\textrm{CH}_{3}\right]_2\textrm{CO}}\xspace}
\newcommand{\methyleneamidogen}{\ensuremath{\textrm{H}_{2}\textrm{CN}}\xspace}
\newcommand{\Rone}{\ensuremath{\para~S_{\nu}(\threetwoone) / S_{\nu}(\threeohthree)}\xspace}
\newcommand{\Rtwo}{\ensuremath{\para~S_{\nu}(\threetwotwo) / S_{\nu}(\threetwoone)}\xspace}
\newcommand{\JKaKc}{\ensuremath{J_{K_a K_c}}}
\newcommand{\water}{H$_{2}$O\xspace}		%  H2O
\newcommand{\feii}{\ion{Fe}{2}}		%  FeII
\newcommand{\uchii}{\ion{UCH}{ii}\xspace}
\newcommand{\UCHII}{\ion{UCH}{ii}\xspace}
\newcommand{\hchii}{\ion{HCH}{ii}\xspace}
\newcommand{\HCHII}{\ion{HCH}{ii}\xspace}
\newcommand{\hii}{\ion{H}{ii}\xspace}
\newcommand{\hi}{H~{\sc i}\xspace}
\newcommand{\Hii}{\hii}
\newcommand{\HII}{\hii}
\newcommand{\Xform}{\ensuremath{X_{\formaldehyde}}}
\newcommand{\kms}{\textrm{km~s}\ensuremath{^{-1}}\xspace}	%  km s-1
\newcommand{\nsample}{456\xspace}
\newcommand{\CFR}{5\xspace} % nMPC / 0.25 / 2 (6 for W51 once, 8 for W51 twice) REFEDIT: With f_observed=0.3, becomes 3/2./0.3 = 5
\newcommand{\permyr}{\ensuremath{\mathrm{Myr}^{-1}}\xspace}
\newcommand{\pers}{\ensuremath{\mathrm{s}^{-1}}\xspace}
\newcommand{\tsuplim}{0.5\xspace} % upper limit on starless timescale
\newcommand{\ncandidates}{18\xspace}
\newcommand{\mindist}{8.7\xspace}
\newcommand{\rcluster}{2.5\xspace}
\newcommand{\ncomplete}{13\xspace}
\newcommand{\middistcut}{13.0\xspace}
\newcommand{\nMPC}{3\xspace} % only count W51 once.  W51, W49, G010
\newcommand{\obsfrac}{30}
\newcommand{\nMPCtot}{10\xspace} % = nmpc / obsfrac
\newcommand{\nMPCtoterr}{6\xspace} % = sqrt(nmpc) / obsfrac
\newcommand{\plaw}{2.1\xspace}
\newcommand{\plawerr}{0.3\xspace}
\newcommand{\mmin}{\ensuremath{10^4~\msun}\xspace}
%\newcommand{\perkmspc}{\textrm{per~km~s}\ensuremath{^{-1}}\textrm{pc}\ensuremath{^{-1}}\xspace}	%  km s-1 pc-1
\newcommand{\kmspc}{\textrm{km~s}\ensuremath{^{-1}}\textrm{pc}\ensuremath{^{-1}}\xspace}	%  km s-1 pc-1
\newcommand{\sqcm}{cm$^{2}$\xspace}		%  cm^2
\newcommand{\percc}{\ensuremath{\textrm{cm}^{-3}}\xspace}
\newcommand{\perpc}{\ensuremath{\textrm{pc}^{-1}}\xspace}
\newcommand{\persc}{\ensuremath{\textrm{cm}^{-2}}\xspace}
\newcommand{\persr}{\ensuremath{\textrm{sr}^{-1}}\xspace}
\newcommand{\peryr}{\ensuremath{\textrm{yr}^{-1}}\xspace}
\newcommand{\perkmspc}{\textrm{km~s}\ensuremath{^{-1}}\textrm{pc}\ensuremath{^{-1}}\xspace}	%  km s-1 pc-1
\newcommand{\perkms}{\textrm{per~km~s}\ensuremath{^{-1}}\xspace}	%  km s-1 
\newcommand{\um}{\ensuremath{\mu \textrm{m}}\xspace}    % micron
\newcommand{\microjy}{\ensuremath{\mu\textrm{Jy}}\xspace}    % micron
\newcommand{\mum}{\um}
\newcommand{\htwo}{\ensuremath{\textrm{H}_2}}
\newcommand{\Htwo}{\ensuremath{\textrm{H}_2}}
\newcommand{\HtwoO}{\ensuremath{\textrm{H}_2\textrm{O}}}
\newcommand{\htwoo}{\ensuremath{\textrm{H}_2\textrm{O}}}
\newcommand{\ha}{\ensuremath{\textrm{H}\alpha}}
\newcommand{\hb}{\ensuremath{\textrm{H}\beta}}
\newcommand{\so}{SO~\ensuremath{5_6-4_5}\xspace}
\newcommand{\SO}{SO~\ensuremath{1_2-1_1}\xspace}
\newcommand{\ammonia}{NH\ensuremath{_3}\xspace}
\newcommand{\twelveco}{\ensuremath{^{12}\textrm{CO}}\xspace}
\newcommand{\thirteenco}{\ensuremath{^{13}\textrm{CO}}\xspace}
\newcommand{\ceighteeno}{\ensuremath{\textrm{C}^{18}\textrm{O}}\xspace}
\def\ee#1{\ensuremath{\times10^{#1}}}
\newcommand{\degrees}{\ensuremath{^{\circ}}}
% can't have \degree because I'm getting a degree...
\newcommand{\lowirac}{800}
\newcommand{\highirac}{8000}
\newcommand{\lowmips}{600}
\newcommand{\highmips}{5000}
\newcommand{\perbeam}{\ensuremath{\textrm{beam}^{-1}}}
\newcommand{\ds}{\ensuremath{\textrm{d}s}}
\newcommand{\dnu}{\ensuremath{\textrm{d}\nu}}
\newcommand{\dv}{\ensuremath{\textrm{d}v}}
\def\secref#1{Section \ref{#1}}
\def\eqref#1{Equation \ref{#1}}
\def\facility#1{#1}
%\newcommand{\arcmin}{'}

\newcommand{\necluster}{Sh~2-233IR~NE}
\newcommand{\swcluster}{Sh~2-233IR~SW}
\newcommand{\region}{IRAS 05358}

\newcommand{\nwfive}{40}
\newcommand{\nouter}{15}

\newcommand{\vone}{{\rm v}1.0\xspace}
\newcommand{\vtwo}{{\rm v}2.0\xspace}
\newcommand\mjysr{\ensuremath{{\rm MJy~sr}^{-1}}}
\newcommand\jybm{\ensuremath{{\rm Jy~bm}^{-1}}}
\newcommand\nbolocat{8552\xspace}
\newcommand\nbolocatnew{548\xspace}
\newcommand\nbolocatnonew{8004\xspace} % = nbolocat-nbolocatnew
\renewcommand\arcdeg{\mbox{$^\circ$}\xspace} 
\renewcommand\arcmin{\mbox{$^\prime$}\xspace} 
\renewcommand\arcsec{\mbox{$^{\prime\prime}$}\xspace} 

\newcommand{\todo}[1]{\textcolor{red}{#1}}
\newcommand{\okinfinal}[1]{{#1}}
%% only needed if not aastex
%\newcommand{\keywords}[1]{}
%\newcommand{\email}[1]{}
%\newcommand{\affil}[1]{}


%aastex hack
%\newcommand\arcdeg{\mbox{$^\circ$}}%
%\newcommand\arcmin{\mbox{$^\prime$}\xspace}%
%\newcommand\arcsec{\mbox{$^{\prime\prime}$}\xspace}%

%\newcommand\epsscale[1]{\gdef\eps@scaling{#1}}
%
%\newcommand\plotone[1]{%
% \typeout{Plotone included the file #1}
% \centering
% \leavevmode
% \includegraphics[width={\eps@scaling\columnwidth}]{#1}%
%}%
%\newcommand\plottwo[2]{{%
% \typeout{Plottwo included the files #1 #2}
% \centering
% \leavevmode
% \columnwidth=.45\columnwidth
% \includegraphics[width={\eps@scaling\columnwidth}]{#1}%
% \hfil
% \includegraphics[width={\eps@scaling\columnwidth}]{#2}%
%}}%


%\newcommand\farcm{\mbox{$.\mkern-4mu^\prime$}}%
%\let\farcm\farcm
%\newcommand\farcs{\mbox{$.\!\!^{\prime\prime}$}}%
%\let\farcs\farcs
%\newcommand\fp{\mbox{$.\!\!^{\scriptscriptstyle\mathrm p}$}}%
%\newcommand\micron{\mbox{$\mu$m}}%
%\def\farcm{%
% \mbox{.\kern -0.7ex\raisebox{.9ex}{\scriptsize$\prime$}}%
%}%
%\def\farcs{%
% \mbox{%
%  \kern  0.13ex.%
%  \kern -0.95ex\raisebox{.9ex}{\scriptsize$\prime\prime$}%
%  \kern -0.1ex%
% }%
%}%

\def\Figure#1#2#3#4#5{
\begin{figure*}[!htp]
\includegraphics[scale=#4,width=#5]{#1}
\caption{#2}
\label{#3}
\end{figure*}
}

\def\WrapFigure#1#2#3#4#5#6{
\begin{wrapfigure}{#6}{0.5\textwidth}
\includegraphics[scale=#4,width=#5]{#1}
\caption{#2}
\label{#3}
\end{wrapfigure}
}

% % #1 - filename
% % #2 - caption
% % #3 - label
% % #4 - epsscale
% % #5 - R or L?
% \def\WrapFigure#1#2#3#4#5#6{
% \begin{wrapfigure}[#6]{#5}{0.45\textwidth}
% %  \centercaption
% %  \vspace{-14pt}
%   \epsscale{#4}
%   \includegraphics[scale=#4]{#1}
%   \caption{#2}
%   \label{#3}
% \end{wrapfigure}
% }

\def\RotFigure#1#2#3#4#5{
\begin{sidewaysfigure*}[!htp]
\includegraphics[scale=#4,width=#5]{#1}
\caption{#2}
\label{#3}
\end{sidewaysfigure*}
}

\def\FigureSVG#1#2#3#4{
\begin{figure*}[!htp]
    \def\svgwidth{#4}
    \input{#1}
    \caption{#2}
    \label{#3}
\end{figure*}
}

% originally intended to be included in a two-column paper
% this is in includegraphics: ,width=3in
% but, not for thesis
\def\OneColFigure#1#2#3#4#5{
\begin{figure}[!htpb]
\epsscale{#4}
\includegraphics[scale=#4,angle=#5]{#1}
\caption{#2}
\label{#3}
\end{figure}
}

\def\SubFigure#1#2#3#4#5{
\begin{figure*}[!htp]
\addtocounter{figure}{-1}
\epsscale{#4}
\includegraphics[angle=#5]{#1}
\caption{#2}
\label{#3}
\end{figure*}
}

%\def\FigureTwo#1#2#3#4#5{
%\begin{figure*}[!htp]
%\epsscale{#5}
%\plottwo{#1}{#2}
%\caption{#3}
%\label{#4}
%\end{figure*}
%}

\def\FigureTwo#1#2#3#4#5#6{
\begin{figure*}[!htp]
\subfigure[]{ \includegraphics[scale=#5,width=#6]{#1} }
\subfigure[]{ \includegraphics[scale=#5,width=#6]{#2} }
\caption{#3}
\label{#4}
\end{figure*}
}

\def\FigureTwoAA#1#2#3#4#5#6{
\begin{figure*}[!htp]
\subfigure[]{ \includegraphics[scale=#5,width=#6]{#1} }
\subfigure[]{ \includegraphics[scale=#5,width=#6]{#2} }
\caption{#3}
\label{#4}
\end{figure*}
}

\newenvironment{rotatepage}%
{}{}
   %{\pagebreak[4]\afterpage\global\pdfpageattr\expandafter{\the\pdfpageattr/Rotate 90}}%
   %{\pagebreak[4]\afterpage\global\pdfpageattr\expandafter{\the\pdfpageattr/Rotate 0}}%


\def\RotFigureTwoAA#1#2#3#4#5#6{
\begin{rotatepage}
\begin{sidewaysfigure*}[!htp]
\subfigure[]{ \includegraphics[scale=#5,width=#6]{#1} }
\\
\subfigure[]{ \includegraphics[scale=#5,width=#6]{#2} }
\caption{#3}
\label{#4}
\end{sidewaysfigure*}
\end{rotatepage}
}

\def\RotFigureThreeAA#1#2#3#4#5#6#7{
\begin{rotatepage}
\begin{sidewaysfigure*}[!htp]
\subfigure[]{ \includegraphics[scale=#6,width=#7]{#1} }
\\
\subfigure[]{ \includegraphics[scale=#6,width=#7]{#2} }
\\
\subfigure[]{ \includegraphics[scale=#6,width=#7]{#3} }
\caption{#4}
\label{#5}
\end{sidewaysfigure*}
\end{rotatepage}
\clearpage
}

\def\FigureThreeAA#1#2#3#4#5#6#7{
\begin{figure*}[!htp]
\subfigure[]{ \includegraphics[scale=#6,width=#7]{#1} }
\subfigure[]{ \includegraphics[scale=#6,width=#7]{#2} }
\subfigure[]{ \includegraphics[scale=#6,width=#7]{#3} }
\caption{#4}
\label{#5}
\end{figure*}
}



\def\SubFigureTwo#1#2#3#4#5{
\begin{figure*}[!htp]
\addtocounter{figure}{-1}
\epsscale{#5}
\plottwo{#1}{#2}
\caption{#3}
\label{#4}
\end{figure*}
}

\def\FigureFour#1#2#3#4#5#6{
\begin{figure*}[!htp]
\subfigure[]{ \includegraphics[width=3in]{#1} }
\subfigure[]{ \includegraphics[width=3in]{#2} }
\subfigure[]{ \includegraphics[width=3in]{#3} }
\subfigure[]{ \includegraphics[width=3in]{#4} }
\caption{#5}
\label{#6}
\end{figure*}
}

\def\FigureFourPDF#1#2#3#4#5#6{
\begin{figure*}[!htp]
\subfigure[]{ \includegraphics[width=3in,type=pdf,ext=.pdf,read=.pdf]{#1} }
\subfigure[]{ \includegraphics[width=3in,type=pdf,ext=.pdf,read=.pdf]{#2} }
\subfigure[]{ \includegraphics[width=3in,type=pdf,ext=.pdf,read=.pdf]{#3} }
\subfigure[]{ \includegraphics[width=3in,type=pdf,ext=.pdf,read=.pdf]{#4} }
\caption{#5}
\label{#6}
\end{figure*}
}

\def\FigureThreePDF#1#2#3#4#5{
\begin{figure*}[!htp]
\subfigure[]{ \includegraphics[width=3in,type=pdf,ext=.pdf,read=.pdf]{#1} }
\subfigure[]{ \includegraphics[width=3in,type=pdf,ext=.pdf,read=.pdf]{#2} }
\subfigure[]{ \includegraphics[width=3in,type=pdf,ext=.pdf,read=.pdf]{#3} }
\caption{#4}
\label{#5}
\end{figure*}
}

\def\Table#1#2#3#4#5{
%\renewcommand{\thefootnote}{\alph{footnote}}
\begin{table}
\caption{#2}
\label{#3}
    \begin{tabular}{#1}
        \hline\hline
        #4
        \hline
        #5
        \hline
    \end{tabular}
\end{table}
%\renewcommand{\thefootnote}{\arabic{footnote}}
}


%\def\Table#1#2#3#4#5#6{
%%\renewcommand{\thefootnote}{\alph{footnote}}
%\begin{deluxetable}{#1}
%\tablewidth{0pt}
%\tabletypesize{\footnotesize}
%\tablecaption{#2}
%\tablehead{#3}
%\startdata
%\label{#4}
%#5
%\enddata
%\bigskip
%#6
%\end{deluxetable}
%%\renewcommand{\thefootnote}{\arabic{footnote}}
%}

%\def\tablenotetext#1#2{
%\footnotetext[#1]{#2}
%}

\def\LongTable#1#2#3#4#5#6#7#8{
% required to get tablenotemark to work: http://www2.astro.psu.edu/users/stark/research/psuthesis/longtable.html
\renewcommand{\thefootnote}{\alph{footnote}}
\begin{longtable}{#1}
\caption[#2]{#2}
\label{#4} \\

 \\
\hline 
#3 \\
\hline
\endfirsthead

\hline
#3 \\
\hline
\endhead

\hline
\multicolumn{#8}{r}{{Continued on next page}} \\
\hline
\endfoot

\hline 
\endlastfoot
#7 \\

#5
\hline
#6 \\

\end{longtable}
\renewcommand{\thefootnote}{\arabic{footnote}}
}

\def\TallFigureTwo#1#2#3#4#5#6{
\begin{figure*}[htp]
\epsscale{#5}
\subfigure[]{ \includegraphics[width=#6]{#1} }
\subfigure[]{ \includegraphics[width=#6]{#2} }
\caption{#3}
\label{#4}
\end{figure*}
}

		% file containing author's macro definitions

\begin{document}
\title{Cores and Outflows and Chemistry in the W51 Protoclusters}

This is not yet a paper.  It is a collection of paper ideas and some 
preliminary analysis that will make it in to one or more papers.

Data summary: the 7m and 12m data have all been delivered, but the UV coverage
is non-optimal.  Importantly and unfortunately, no single-dish (total power)
data were ever acquired.  I have requested APEX observations using the new
PI230 instrument to recover the spectral line zero spacings, but that data may
not be received for quite some time.  The original proposed project connecting
cloud scales to core scales will not be possible until the single dish data are
acquired.


Subprojects:
\begin{enumerate}
    \item Core identification and mass estimation and maybe even
        core mass function constraints.  Lead: Ginsburg.  This document
        is a quasi-draft of that work.
    \item ``A multi-phase outflow from a high-mass protostar"
        The Lacy jet is detected in both CO and RRLs.  This tells us something
        about its proximity to the HII region IRS2, but what else can we learn
        from it?  Lead: maybe Ginsburg.  Maybe this gets incorporated
        into other papers.
    \item Comparitive chemistry of the W51 cores.  Lead: Victor Rivilla \&
        Maite Beltr{\'a}n, possibly with participation from Alvaro
        Sanchez-Monge.
        This project will involve identifying all of the lines in the W51 cores
        and examining how the chemistry relates to physical parameters.
        We will use LTE modeling tools (XCLASS, MADCUBA) for many species to
        identify lines and determine abundances and temperatures
    \item \formaldehyde and CO turbulent cloud modeling / simulation comparison
        (Loughnane).  A difficulty here is that combination of the 7m and 12m
        data has not worked out very well yet, either for the lines or the
        continuum, and that may render cloud-scale analysis quite difficult.
    \item A study of the CO outflows.  Lead: Maybe Luke Maud?  Ciriaco Goddi
        will be involved.  The $^{12}$CO and SO outflows are spectacular and
        plentiful.  It is not obvious whether they can or should be
        incorporated into this paper; at least, I am not presently prepared to
        put in the effort to quantify the outflows properly.  Ciriaco is PI of
        a long-baseline program that has resolved the e2/e8 and North cores
        with $5\times$ better resolution than this program, and that data set
        may therefore be better suited to a core-outflow association work.
    \item Relative kinematics of ionized and molecular gas.  Lead: Galv{\'a}n-Madrid.
        Kinematics similar to \citet{Keto2008a} using RRLs+molecular lines to determine
        outflow vs infall
        
\end{enumerate}


\section{Overview}
\todo{This section is not part of any proposed paper text.}
The work below will be incorporated into one or more papers as it proves useful.
The figures \& text represent an initial exploration of the data.  There are some important notes
that are not included in the text:
\begin{itemize}
    \item The data reduction has some substantial problems because of strange UV coverage
        and stranger behavior of clean (for which I have opened many tickets).  It appears
        impossible to get the noise below $\sim1 $ mJy around the bright sources, and detections
        are unreliable below $\sim10$ mJy in these area (which I've tried to account for when doing
        source extraction)
    \item The 7m data does not combine well with the 12m data.  It seems to
        cause major large-angular-scale artifacts.  I think this is a weighting
        issue in which the 7m data have lower noise than the 12m data and are
        therefore too heavily weighted, causing large "halos" where the large
        angular scale dominates over the small, providing spuriously strong
        detections.  In principle, this can be solved by downweighting the 7m data,
        but it turns out CASA does not have that implemented (\texttt{innertaper}
        is the relevant keyword)

\end{itemize}


\section{Observations}
As part of ALMA Cycle 2 program 2013.1.00308.S, we observed a
$\sim2\arcmin\times1\arcmin$ region centered between W51 IRS2 and W51 e1/e2
with a 37-pointing mosaic.  Two configurations of the 12m array were used,
achieving a resolution of 0.2\arcsec.  Additionally, a 12-pointing mosaic was
performed using the 7m array, probing scales up to XXX\arcsec.  The full UV
coverage was from XX to YY m.

Data reduction was performed using CASA.  The QA2-produced data products were
combined using the standard inverse variance weighting (check this).
The visibilities were imaged into full spectral cubes at coarse (0.5\arcsec)
resolution in order to get a first look at all 15630 spectral channels.  The
resulting cube was used to identify bright lines in the spectrum extracted
from source e8.  To produce continuum images, frequency channels including
bright lines were excluded.

The spectral setups were ...

\subsection{Data Reduction}
\subsubsection{Continuum}
A continuum image combining all 4 spectral windows was produced using
\texttt{tclean}.  We phase self-calibrated the image on baselines longer than
100m to increase the dynamic range.  The final image was cleaned with 50000
iterations to a threshold of 5 mJy.  The lowest noise level in the image, away
from bright sources, is $\sim0.2$ mJy/beam, but near the bright sources e2 and
IRS2, the noise reached as high as $\sim2$ mJy/beam.  Deeper cleaning was
attempted, but lead to instabilities.


\subsubsection{Lines}
We produced spectral image cubes of the lines listed in Table ....
The cubes use only 12m data, as the 12m + 7m combination was unsuccessful.


\subsubsection{UV coverage}
\Figure{figures/visibility_weight_vs_uvdist_linear.png}
{A weighted histogram of the visibility weights as a function of UV distance;
this approximately shows the amount of data received at each baseline length.}
{fig:uvcov}{1}{16cm}

\subsection{Continuum Morphological Analysis}
\label{sec:morphology}
The largest detected structures include the W51 Main HII region bubble and the
W51 IRS2 HII region, which are relatively uninteresting since their properties
have been previously well-characterized using radio (JVLA) data.  More exciting
are the bright dusty structures, especially the ``tail'' pointing south of W51
e8, which can be described as a 0.25 pc by 0.05 pc filament. This structure has
a very high surface brightness along its ridge, exceeding 40 mJy/beam in our TE
maps (23 K or 3.7\ee{4} MJy sr$^{-1}$).  This high brightness implies a high
intrinsic temperature, $T>30$ K (Section \ref{sec:temperature}).

This narrow filament is most prominent in the continuum. It is evident
in some lines (\formaldehyde, $^{13}$CS, \ceighteeno), but not others (SO, ).
It is surrounded by molecular emission that is only slightly fainter...
...in SO it's pretty uniform brightness...

\subsubsection{A bubble around e5}
There is evidence of a bubble in the continuum around e5 with a radius of
6.2\arcsec (0.16 pc; Figure \ref{fig:e5bubble}).  The bubble is completely
absent in the centimeter continuum, so the observed emission is from dust.  The
bubble edge can be seen from 58 \kms to 63 \kms in \ceighteeno and
\formaldehyde, though it is not continuous in any single velocity channel.
There is a collection of compact sources (protostars or cores) along the
southeast edge of the bubble.

The presence of such a bubble in dense gas, but its absence in ionizing gas, is
surprising.  The most likely mechanism for blowing such a bubble is ionizing
radiative feedback, especially around a source that is currently a hypercompact
HII region, but since no free-free emission is evident within or on the edge of
the bubble, it is at least not presently driving the bubble.  A plausible
explanation for this discrepancy is that e5 was an exposed O-star within the
past Myr, but has since begun accreting heavily and therefore had its HII
region shrunk.  This model is marginally supported by the presence of a `pillar'
of dense material pointing from e5 toward the south.

The total flux in the north half of the `bubble', which shows no signs of
free-free contamination, is about 1.5 Jy.  The implied mass in just this
fragment of the bubble is about $M\sim350$ \msun for a relatively high assumed
temperature $T=50$ K.  The total mass of the bubble is closer to $M\sim1000$
\msun, though it may be lower ($\sim500$ \msun) if the southern half is
dominated by free-free emission.

With such a large mass, the implied density of the original cloud, assuming it
was uniformly distributed over a 0.2 pc sphere, is $n(\hh) \approx 2-5\ee{5}$
\percc.

% this analysis courtesy Jim Dale
To evaluate the plausibility of the \hii-region origin of the bubble, we compare
to classical equations for \hii regions.
The Str\"omgren radius is \\
\begin{eqnarray}
R_{\rm s}=\left(\frac{3Q_{\rm H}}{4\pi\alpha_{\rm B} n^{2}}\right)^{\frac{1}{3}}.
\end{eqnarray} 
For $Q_{\rm H}\sim10^{49}$ \pers, $\alpha_{\rm B}=3\times10^{-13}$\,cm$^{3}$\,s$^{-1}$, $R_{\rm s}\approx0.01$\,pc.\\
\\
The Spitzer solution for HII region expansion gives\\
\begin{eqnarray}
R_{\rm HII}(t)=R_{\rm s}\left(1+\frac{7}{4}\frac{c_{\rm II}t}{R_{\rm s}}\right)^{\frac{4}{7}}.
\end{eqnarray} 
With $c_{\rm II}=7.5$\,km\,s$^{-1}$ and $t=10^{4}$\,yr,
$R_{\rm HII}(t)\approx0.04$\,pc, while at $t=10^5$\,yr, it is $R_{\rm
HII}\approx0.16$\,pc, which is comparable to the observed radius
($r_{obs} \sim 0.13-0.19$ pc)\\
\\
Whitworth et al. 1994 give the fragmentation timescale as
\begin{eqnarray}
t_{\rm frag}\sim1.56\left(\frac{c_{\rm s}}{0.2{\rm km\,s}^{-1}}\right)^{\frac{7}{11}}\left(\frac{Q_{\rm H}}{10^{49}{\rm s}^{-1}}\right)^{-\frac{1}{11}}\left(\frac{n}{10^{3}{\rm cm}^{-3}}\right)^{-\frac{5}{11}}{\rm Myr}.
\end{eqnarray} 
Plugging in our numbers gives $t_{\rm frag}\approx1.0\times10^{5}$\,yr, or
$10\times$ longer than the expansion time.\\
\\
% The corresponding radius at which fragmentation occurs is\\
% \begin{eqnarray}
% R_{\rm frag}\sim5.8\left(\frac{c_{\rm s}}{0.2{\rm km\,s}^{-1}}\right)^{\frac{4}{11}}\left(\frac{Q_{\rm H}}{10^{49}{\rm s}^{-1}}\right)^{\frac{1}{11}}\left(\frac{n}{10^{3}{\rm cm}^{-3}}\right)^{-\frac{6}{11}}{\rm pc},
% \end{eqnarray}
% which gives us $R_{\rm frag}\approx0.17$\,pc.\\

These values are consistent with a late O-type star having been exposed,
driving an \hii region, for $\sim10^4-10^5$ year, after which a substantial
increase in the accretion rate quenched the ionizing radiation from the star,
trapping it into a hypercompact ($r<0.005$ pc) configuration.  The
recombination timescale is short enough that the ionized gas would disappear
almost immediately after the continuous ionizing radiation source was hidden.
This is essentially the scenario laid out in \citet{de-Pree2014a} as an
explanation for the compact \hii region lifetime problem.  In this case,
however, it also seems that the \hii region has effectively driven the
``collect'' phase of what will presumably end in a collect-and-collapse style
triggering event.

Technically, it is possible that e5 actually represents an optically thick
high-mass-loss-rate wind rather than an ultracompact HII region, but I think we
can rule this out on physical plausibility considerations if we compare to 
wind models.  Note that $\eta$ Car would have a flux of $\sim0.5$ Jy at 2 cm
and $\sim5$ Jy at 1 mm at the distance of W51.

\FigureTwo{figures/e5_bubble.png}{figures/e5_bubble_robust2.png}
{The bubble around source e5.  The bubble interior shows no sign of centimeter
emission, though the lower-left region of the shell - just south of the
``cores'' - coincides with part of the W51 Main ionized shell.  The source of
the ionization is not obvious.
({\it Left}): A robust -2.0 image with a small (0.2\arcsec) beam and poor
recovery of large angular scale emission.  This image highlights the presence
of protostellar cores on the left edge of the bubble and along a filament just
south of the central source.
({\it Right}): A robust +2.0 image with a larger (0.4\arcsec) beam and better
recovery of large angular scales.  The contours show radio continuum (14.5 GHz)
emission at 1.5, 3, and 6 mJy/beam.  While some of the detected 1.4 mm emission
in the south could be free-free emission, the eastern and northern parts of the
shell show no emission down to the 50 $\mu$Jy noise level of the Ku-band map,
confirming that they consist only of dust emission.
}{fig:e5bubble}{1}{8cm}




\subsection{Dense Gas Kinematics}
\label{sec:kinematics}
We examine the gas kinematics throughout the cloud, but especially near the 
massive cores.

The ambient cloud, which consists of gas that has not yet condensed into
compact prestellar objects, is evident in absorption against the mm cores at
55-58 \kms (toward e2) and 57-60 \kms (toward e8).  Narrower velocity
components related to the known high-velocity streams are detected around 68
\kms toward both sources.  These absorption features are seen in all 
\formaldehyde transitions, \methanol $4_{2,2}-3_{1,2}$, OCS 18-17,
but it was less obvious or absent in HNCO $10_{1,10}-9_{1,9}$ and OCS 19-18.

In the material surrounding the e2 and e8 ``cores'', one particularly notable
feature is that the cores themselves show a redshifted centroid velocity
relative to their surroundings in nearly all of the bright lines (H2CO, OCS,
SO, \ceighteeno).  The observed shift is up to $\lesssim2$ \kms.  The shift is
a sign of infall.  Given the high continuum brightness, the cores are likely
optically thick in the continuum (Section \ref{sec:} XXX), therefore obscuring
all molecular emission behind them.  We are seeing only gas in the
foreground, and this gas is clearly moving toward the cores.

The source ALMAmm14 shows a similar kinematic signature....

% \subsection{Simulations}
% {\bf Unfortunately CASA's simulations produce reproducible incorrectness, in
% that I cannot get an image that matches the input image.  There seems to
% *always* be some flux scaling no matter what input unit is used.  Therefore, I
% don't trust any of the results of CASA simulations yet.  Perseus looks OK, but
% Aquila is just flat out wrong, and the scale-recovery simulations I attempted
% also failed.
% 
% Followup on the above paragraph: I've gotten the simulations worked out; CASA
% always treats data as if they are in Jy/beam even if Jy/pixel units are
% specified when using the \texttt{sm.predict} module.  However, I'm not
% convinced of their utility at this stage.}
% 
% The enhanced noise around bright sources is unavoidable.  We tested the noise
% properties of our data set using the CASA \texttt{simobserve} toolkit.  We
% obtained a Herschel Gould's Belt Survey image of the Perseus molecular cloud at
% 250 \um \citep[resolution 18\arcsec][]{} and scaled it down by $\sim40\times$
% to match the resolution of our ALMA data.  We used the \texttt{sm.predict} task
% to ``observe'' the Herschel data with our exact UV data set.  We then used
% \texttt{sm.corrupt} to make the noise properties approximately match those of
% our observations.
% 
% We used the Perseus data scaled from 250 \um to 1.3 mm assuming
% a relatively shallow $\beta=1.5$ and a constant temperature $T=20$ K
% (since temperature maps are not available).
% For the Aquila data, we used the column density and temperature maps to
% derive a synthetic 1.3 mm map assuming an opacity $\kappa_{505 \mathrm{GHz}} =
% 4$ g \percc.  Since Aquila is at a greater distance, the Herschel resolution is
% coarser (0.9 \arcsec at 5.4 kpc) than our best resolution of 0.2\arcsec, so it
% is best compared to lower-resolution tapered data.
% 
% 
% When imaging the Perseus data set, we put NGC 1333 at the image center.  At the
% noise levels in our data, only the central portions of NGC 1333 are detected,
% with three point sources recovered.  In this map, the noise properties are
% very uniform.  We are therefore unable to analyze the NGC
% 1333 data with the exact same parameters as were used on W51.  However, using
% similar parameters (but with a lower significance threshold), we detect only
% three sources.
% 
% In a second experiment, we scaled the peak flux density of the Perseus map to
% be $\sim100\times$ brighter than it should be, making it comparable to the flux
% density of W51e2 in the real ALMA observations.  In this map, even with deep
% cleaning, the noise around the bright sources remains very high and artifacts
% are evident.



\section{Analysis}
\subsection{Source Identification}
% dendrogramming.py
We used the \texttt{dendrogram} method described by \citet{Rosolowsky2008c} and
implemented in \texttt{astrodendro} to identify sources.  We used a minimum
value of 1 mJy/beam ($\sim5-\sigma$) and a minimum $\Delta=0.4$ mJy/beam
($\sim2-\sigma$) with minimum 10 pixels (each pixel is 0.05\arcsec).  This
cataloging yielded over 8000 candidate sources, of which the majority are noise
or artifacts around the brightest sources.  To filter out these bad sources,
we created a noise map taking the local RMS of the \texttt{tclean}-produced
residual map over a $\sigma=30$ pixel (1.5\arcsec) gaussian.  We then removed
all sources with peak S/N < 8, mean S/N per pixel $< 5$, and minimum S/N per
pixel $ < 1$.  We also only included the smallest sources in the dendrogram,
the ``leaves''.  These parameters were tuned by checking against ``real''
sources identified by eye and selected using \texttt{ds9}: most real sources are
recovered (but not all; see Section ...) and few spurious sources ($<10$) are
included.  The resulting catalog includes 113 sources.

The `by-eye' core extraction approach, in which we placed ds9 regions on all
sources that look `real', produced a more reliable but less complete (and less
quantifiable) catalog.  This catalog is more useful in the regions around the
bright sources e2 and north, since these regions are affected by substantial
uncleaned PSF sidelobe artifacts.  In particular, the dendrogram catalog includes
a number of sources around e2/e8 that, by eye, appear to be parts of continuous
extended emission rather than local peaks; ``streaking'' artifacts in the
reduced data result in their identification despite our threshold criteria.

\subsection{The spatial distribution of cores}
The detected cores are not uniformly distributed across the observed region.
The most notable feature in the spatial distribution is their alignment: most
of the cores are detected along common lines.  This is especially evident
in W51 IRS2, where the core density is very high and there is virtually no
deviation from the line.  The e8 filament is also notably linear, though there
area few sources detected just off the filament. 

On a larger scale, the e8 filament points toward e2, apparently tracing a
slightly longer filamentary structure.  With some imagination, this might be
extended along the entire northeast ridge to eventually connect in a broad
half-circle with the IRS2 filament (Figure \ref{fig:corepositions}).  This
morphology hints at a possible sequential star formation event, where some
central bubble has swept gas into these filaments.  However, this ring has no
counterparts in ionized gas, and there is little reason to expect such circular
symmetry from a real cloud, so the star forming circle may be merely a figment.

Whether it is physical or not, there is a notable lack of cores within the
circle.  There is no lack of molecular gas, however, as both CO and \formaldehyde
emission fill the full field of view.

\Figure{figures/core_spatial_distribution.png}
{The spatial distribution of the hand-identified core sample.
The black outer contour shows the observed field of view.
The dashed circle shows a hypothetical ring of star formation.
}{fig:corepositions}{1}{16cm}

\subsection{Photometry}
We created a catalog of the dendrogram sources including their peak and mean
flux density, their centroid, and their geometric properties.  For each source,
we further extracted aperture photometry around the centroid in 6 apertures:
0.2, 0.4, 0.6, 0.8, 1.0, and 1.5\arcsec.  We performed the same aperture
photometry on the W51 Ku-band images from \citet{Ginsburg2016a}.  These
observations are reported in Table {...}.

The source flux density distribution is shown in Figure
\ref{fig:fluxhistograms}.  The most common nearest-neighbor separation between
cataloged cores is $\sim0.3\arcsec$, which implies that the larger apertures
double-count some pixels.  The smallest separation is
0.26\arcsec, so the 0.2\arcsec aperture contains only unique pixels.

\Figure{figures/core_flux_histogram_apertureradius.png}
{Histograms of the core fluxes measured with circular apertures centered
on the hand-extracted core positions.  The aperture size is listed 
in the y-axis label.}
{fig:fluxhistograms}
{1}{16cm}

\Figure{figures/dendro_core_flux_histograms.png}
{Histograms of the core fluxes measured with circular apertures centered
on the dendrogram-extracted core centroids.  The aperture size is listed 
in the y-axis label.  Free-free-dominated sources are excluded.}
{fig:dendrofluxhistograms}
{1}{16cm}

\subsubsection{Distribution Functions}
We fit power law distributions to each aperture's flux distribution.  The
powerlaws steepen slightly from $\alpha=2.0\pm0.12$ to $\alpha=2.2\pm0.16$ for
larger apertures.  The minimum flux represented by a power law increases from
$\sim20$ mJy to 0.4 Jy (14-280 \msun at 20K).  These slopes are shallower than 
the Salpeter-like slope for the mass function derived by \citep{Konyves2015a}
for their sample, though with only modest significance ($<3-\sigma$).  Of
course, these measurements are of the continuum flux density, not directly of
the mass, and so a direct comparison may not be appropriate.  We revisit this
question after assessing the dust temperature in Section \ref{sec:temperature}.

\subsection{Spectral Lines \& Velocities}
To determine the line-of-sight velocity of each source, we extracted a spectrum
from an 0.5\arcsec aperture centered on the source and from a 0.5-1.0\arcsec
annulus around it.  We then searched each spectrum for the brightest pixel and
associated it with the likeliest spectral line.  We repeated this in each of
our 4 spectral windows, then averaged the 4 velocities to get an estimate of
the source velocity.   This process also allowed us to identify the brightest
lines in each window and the brightest overall line observed, which we use
later for temperature estimation.

\subsection{Temperature estimation of the continuum cores}
\label{sec:temperature}
The temperature is a critical ingredient for determining the total mass of each
continuum source or region. Since we do not have any means of directly
determining the dust temperature, as the SED peak is well into the THz regime
and inaccessible with any existing instruments at the requisite resolution, we
employ alternative indicators.  Above a density $n\gtrsim10^5-10^6$ \percc,
the gas and dust become strongly collisionally coupled, meaning the gas
temperature should accurately reflect the dust temperature.  Below this density,
the two may be decoupled.

The average dust temperature, as estimated from Herschel Hi-Gal SED fits
\citep{Molinari2016a,Wang2015a}, is 38 K when including the 70 \um data or 26 K
when excluding it.  This average is obtained over a $\sim45\arcsec$ ($\sim 1$
pc) beam and therefore is likely to be strongly biased toward the hottest dust
in the HII regions and around the massive cores, which are known to have
temperatures reaching $>300$ K \citep{Goddi2016a}.  Despite these
uncertainties, this bulk measurement provides us with a reasonable range to
assume for the uncoupled, low-density dust, which (weakly) dominates the mass
(see Section \ref{sec:massbudget}).

One constraint on the dust temperature we can employ is the absolute surface
brightness.  For some regions, especially the ``filament'' and the hot cores
noted in Section \ref{sec:morphology}, the surface brightness is substantially
brighter than is possible for a beam-filling, optically thick blackbody at 20
K, providing a lower limit on the dust temperature ranging from 30 K (40
mJy/beam) to 600 K (1 Jy/beam).  Toward most of this emission, optically-thick
free-free emission can be strongly ruled out as the driving mechanism using
existing data that limits the free-free contribution to be $<50\%$ if it is
optically thick, and negligible ($<<1\%$) if it is optically thin (Ginsburg et
al, 2016; Goddi2016a).

To gain a more detailed measurement of the dust temperature in regions where it
is likely to be coupled to the gas, we use the peak brightness temperature
$T_{B,max}$ of lines along the line of sight.  If the observed molecule is in
local thermal equilibrium, as is expected if the density is high enough to be
coupled to the dust for many molecules, and it is optically thick, the
brightness temperature provides an approximate measurement of the true physical
temperature near the $\tau=1$ surface.  If any of these assumptions do not
hold, $T_{B,max}$ will set a lower limit on the true gas temperature.  Only
nonthermal (maser) emission would push $T_{B,max} > T_{gas}$.

One potential problem with this approach is if the gas becomes optically thick
before probing most of the dust.  Some transitions, e.g. CO and \formaldehyde, 
are likely to be affected by this issue.


\Figure{figures/dendro_peakTB_vs_selfconsistentcontinuum.png}
{The peak line brightness vs the continuum brightness of our target sources
within $\sim1\arcsec$ beams.  The points are color-coded by the brightest
observed line.  The dashed line shows where the brightness temperature of the
continuum matches that of the lines; technically this means that it should be
impossible for any point to be below the line.  
%However, in this
%iteration of the plot, the lines are extracted from broader apertures than the
%continuum, so the continuum peak brightness is capable of being higher than the
%line peak brightness.  TODO: replace this plot once spectra have been
%appropriately extracted from the high-resolution spectral data cube.
}
{fig:peaktb}{1}{10cm}

\subsubsection{W51e2e Mass and Temperature Estimates}
W51e2e

In a $0.21\arcsec\times0.19\arcsec$ beam ($1100\times1000$ au), the peak flux
density is 0.38 Jy, which corresponds to a brightness temperature $T=228$ K.
This is a lower limit to the surface brightness of the millimeter core, since
an optical depth $\tau<1$ or a filling factor of the emission $ff<1$ would both
imply higher intrinsic temperatures.  The implied luminosity, assuming a pure
blackbody, is $L = 4\pi r^2 \sigma_{sb} T^4 = 2.4\ee{4} \lsun$.  Since any systematic
uncertainties imply a higher temperature, this estimate is a lower limit on the source
luminosity.

If we assume that $T_{dust} = T_{peak}$ and that the dust is optically thin, we
derive a dust mass $M_{dust}\sim20$ \msun.  However, this final mass is not a
strict limit in either direction: if the dust is optically thick, there may be
substantial hidden or undetected gas, while if the filling factor is low, it
may be much hotter and therefore lower in mass.


\subsection{The most massive protostellar cores in W51}
In Figure \ref{fig:hmradprof}, we show the radial profiles extracted from the
three high-mass protostellar cores in W51: W51 North, W51 e2e, and W51 e8.
The plot shows the enclosed mass out to $\sim1\arcsec$ (5400 AU).  On larger
spatial scales, the enclosed mass rises more shallowly, indicating the end of the
core (though our data are capable of recovering spatial scales up to XXXXX).

All three sources show similar radial profiles, containing up to 3000 \msun
within a very compact radius of 5400 AU (0.03 pc).  However, the temperature
structure within these sources is certainly not homogeneous, and very likely a
large fraction of the total flux comes from $T\gtrsim300$ K heated material
\citep[Section \ref{sec:temperature}][]{Goddi2016a}.  If the observed dust were
all at 600 K, the mass would be $\sim17\times$ lower, 100 \msun, which we treat
as a strict lower bound as it is unlikely that the dust more than $\gtrsim1000$
au is so warm.  Additionally, it is very likely that a substantial mass of cold
dust is also present but undetectable because it is hidden by the hotter dust.

\Figure{figures/cumulative_radial_flux_massivecores.png}
{The cumulative flux density radial profiles centered on three
massive protostellar cores.  They share similar profiles and
are likely dominated by hot dust in their innermost regions,
but they are more likely to be dominated by cooler dust in their
outer, more massive regions.  The cumulative mass distribution may
therefore be deceptive.}
{fig:hmradprof}{1}{14cm}

\subsection{The most massive prestellar cores in W51}
% may be repeated from temperature section above
To place limits on the most massive prestellar cores, we need to know the
temperature of the dust in all of the bright ($>15$ mJy; 10 \msun at 20 K)
sources.  We do not have any direct means of evaluating the dust temperature,
but we can infer at
least a lower limit on it by determining the peak brightness temperature of an
optically thick line that is excited at densities $n\gtrsim10^5$ \percc, at
which the dust and gas are coupled.

To accomplish this, we have found the brightest lines across the full $\sim6
GHz$ spectra and measured their peak brightness temperature.

\todo{TODO: use coarser resolution (2\arcsec = 0.1 pc?) data to extract ``cores''
where no smaller (protostellar) cores are detected.  Try to estimate their masses.
These are the best candidate ``prestellar" cores, though likely anything this
large is likely to be a massive cluster...}

\subsection{The mass budget on different spatial scales}
\label{sec:massbudget}
An important evolutionary indicator is the amount of mass at a given density; a
more evolved (more efficiently star-forming) region will have more mass at high
densities.  We cannot measure the dense gas fraction directly, but the amount of
flux density recovered by an interferometer provides a reasonable approximation.

% total_mass_analysis
For the ``total'' flux density in the region, we use the Bolocam Galactic Plane
Survey observations \citep{Aguirre2011a,Ginsburg2013a}, which are the closest
in frequency single-dish millimeter data available.  We assume a spectral index
$\alpha=3.5$ to convert the BGPS flux density measurements at 271.4 GHz to the
mean ALMA frequency of 226.6 GHz.  The ALMA data (\todo{specifically, the
0.2\arcsec 12m-only data}) have a total flux 23.2 Jy above a very conservative
threshold of 10 mJy/beam in our
mosaic; in the same area the BGPS data have a flux of 144 Jy, which scales down to
76.5 Jy.  The recovery fraction is 30$\pm3$\%, where the error bar accounts
for a change in $\alpha\pm0.5$.  The threshold of 10 mJy/beam corresponds to
a column threshold $N>10^{25}$ \percc for 20 K dust.  For an unresolved spherical source,
this corresponds to a volume density $n>10^{8.1}$ \percc.

\todo{TODO: determine the largest angular scale in the ALMA images.  Requires
using the simulations.}

\subsection{Chemically Distinct Regions}
\label{sec:chemistrymaps}
The ``hot cores'' in W51 are resolved and multi-layered.

Surrounding W51e2e, we noted the presence of sharp-edged uniform-brightness
regions in a few spectral lines over the range 51-60 \kms (Figure
\ref{fig:chemmapse2}).  These features are elongated in the direction of the
outflow, but have significant extend orthogonal to the outflow, spanning
$9500\times6600$ AU.  They are prominent in \methanol, OCS, and \dimethylether, weak
but present in \formaldehyde and SO, and absent in \cyanoacetylene and HNCO.

Around e8, a similar feature is observed, but this time \dimethylether is absent.

Along the south end of the e8 filament, no such enhanced features are seen; only
\formaldehyde and the lowest transition of \methanol are evident.

Toward W51 north, \methanol, \formaldehyde, and SO exhibit the sharp-edged enhancement feature,
while the other species do not.  The enhancement is from 50-60 \kms.

Given the enhanced abundance of \methanol in particular, it seems likely that these
sharp-edged bubbles represent sublimation zones in which substantial quantities
of grain-processed materials are released into the gas phase.  The sharp edges
likely reflect the particular point where the temperature exceeds the sublimation
temperature for each species.

The relative chemical structures of e2, e8, and IRS2 are relatively similar.
The same species are detected in all of the central cores.  However, in e2,
\dimethylether, \methylformate, \ethylcyanide, and Acetone (\acetone) are
significantly more extended in e2 than in the other sources.
\gaucheethanol is detected in W51 North, but is weak in e8 and almost absent
in e2.

The substructure of different species in e2 is also revealing.  Species that
are elongated in the NW/SE direction are associated primarily with the outflow
(\cyanoacetylene, \ethylcyanide).  Other species are associated primarily with
the extended core (\methylformate, \dimethylether, \acetone).  Some are only
seen in the compact core (\methyleneamidogen, HNCO, \formamide, and
vibrationally excited \cyanoacetylene).  Only \methanol and OCS are associated
with both the extended core and the outflow, but not the greater extended
emission.  \ketene seems to be associated with only the extended core, but not
the compact core. Finally, there are the species that trace the broader ISM in
addition to the cores and outflows (\formaldehyde, $^{13}$CS, OCS, C$^{18}$O
and SO).  Both HCOOH and N$_2$D+ are weak and associated only with the innermost
e2e core.

\Figure{figures/chemical_m0_slabs_e2.png}
{Moment 0 maps of the e2 region in 40 different lines over the range 51 to 60
\kms with continuum subtraction using the 30th percentile emission
over the ranges 25-40 and 75-90 \kms.  All images are on the same scale, and
the negative features show absorption against the continuum.  There
is a strong `halo' of emission seen in the CH$_3$Ox lines and OCS.  Extended
emission is also clearly seen in SO, $^{13}$CS, and \formaldehyde, though these
lines more smoothly blend into their surroundings.  HNCO and \formamide have
smaller but substantial regions of enhancement with a sharp contrast to their
surroundings.  HC$_3$N traces the e2 outflow.  The bright H30$\alpha$ emission
marks the position of e2w, the hypercompact HII region that dominates the
centimeter emission in e2.
}{fig:chemmapse2}{1}{18cm}

\Figure{figures/chemical_m0_slabs_e8.png}
{Moment 0 maps of the e8 region in 40 different lines over the range 52 to 63
\kms with continuum subtraction using the 30th percentile emission
over the ranges 25-40 and 75-90 \kms.  All images are on the same scale, and
the negative features show absorption against the continuum.  As in e2,
there is extended emission in the CH$_3$OH and OCS lines, but in contrast,
the othe CH$_3$Ox lines are more compact. SO is brighter than OCS in e8, 
whereas the opposite is true in e2.
}{fig:chemmapse8}{1}{18cm}

\Figure{figures/chemical_m0_slabs_north.png}
{Moment 0 maps of the W51 IRS2 region in 40 different lines over the range 54 to 64
\kms with continuum subtraction using the 30th percentile emission
over the ranges 25-40 and 75-90 \kms.  All images are on the same scale, and
the negative features show absorption against the continuum.  Qualitatively,
 the relative extents of species seem comparable to e8.  The H30$\alpha$ 
 point source is W51 d2.
}{fig:chemmapsnorth}{1}{18cm}

\Figure{figures/chemical_m0_slabs_ALMAmm14.png}
{Moment 0 maps of the ALMAmm14 region in 40 different lines over the range 58 to 67
\kms with continuum subtraction using the 30th percentile emission
over the ranges 25-40 and 75-90 \kms.  All images are on the same scale. 
ALMAmm14 is one of the brightest sources outside of e2/e8/IRS2, but
it is substantially fainter than those regions.  Still, it has a noticeably
rich chemistry.
}{fig:ALMAmm14}{1}{18cm}

\subsection{Temperatures \& Columns associated with Chemically Extreme Regions}
The extreme chemical regions appear to be associated with regions of
dramatically elevated gas temperature.  We examine this directly by analyzing
the excitation of lines for which we have detected multiple transitions with
significant energy differences.  We do not use \formaldehyde for this analysis
because it is clearly optically thick (self-absorbed) in the regions of
greatest interest.

We produce rotational diagrams for each spatial pixel covering all \methanol lines
detected at high significance toward at least one position.

\FigureTwo{figures/ch3oh_temperature_map_e2.png}{figures/ch3oh_column_map_e2.png}
{Methanol temperature and column density maps around e2.  The temperature goes
negative toward the center, where the column densities are highest, strongly
indicating that the lines have become optically thick in these regions.}
{fig:ch3ohe2}{1}{10cm}

\Figure{figures/ch3oh_rotation_diagrams_e2.png}
{A sampling of fitted rotation diagrams of the detected \methanol transitions.
These are meant to provide validation of the temperatures and column densities
derived and shown in Figure \ref{fig:ch3ohe2}.  The lower-left corner shows
the position from which the data were extracted in that figure in units of
figure fraction.  The horizontal black lines show the detection threshold of each
of the transitions; points below these lines are ignored when fitting, and instead
the threshold itself is used \todo{CONFIRM / DO THIS}.  The fitted temperature and
column are shown in the top right of each plot.
}{fig:ch3ohe2rot}{1}{18cm}
 
\clearpage

\subsection{Temperatures derived from \formaldehyde}
The original goal of this project was to measure the gas temperatures in the
moderate-density ($n\sim10^4-10^5$ \percc) gas that may correspond to
pre-stellar material.  However, it appears that a large fraction of the area
of the cloud is optically thick in at least the \formaldehyde \threeohthree
transition.  Such a high optical depth is surprising, since the observed
brightness temperatures typically peak at $\sim1$ K and max out at $\lesssim15$
K outside of the central protoclusters.  Such a low brightness temperature for 
optically thick gas implies that the molecules are subthermally excited
but highly abundant.

\section{Outflows}
\label{sec:outflows}
We detected many outflows, primarily in CO 2-1 and SO $6_5-5_4$.  The flows are weakly detected in some
other lines, e.g. \formaldehyde, but we defer discussion of outflow chemistry to....

In this section, we discuss some of the unique outflows and unique features of
outflows in the W51 region.  

\subsection{The Lacy jet}
A high-velocity outflow was discovered within the W51 IRS2 region by
\citet{Lacy2007a}, and subsequently detected in H77$\alpha$ by
\citet{Ginsburg2016b}.  We have now discovered the CO counterpart to this
outflow, which comes from source XXX.   The outflow shows red- and blue-shifted
flows that form the base of the ionized outflow reported by \citet{Lacy2007a}.

Additionally, we have reduced archival VLT SINFONI observations of the region
and discovered a 2-micron \hh knot positioned directly between the cold
molecular (CO) and the ionized components of the flow.  This \hh emission
reveals the position at which the CO is breaking out of the cloud and into the
\hii region.

\subsection{The e2e outflow}
The dominant outflow in W51, which was previously detected by the SMA \citep{},
comes from the source e2e.  This outflow is remarkable for its high velocity,
extending nearly to the limit of our spectral coverage.  The ends of the flow
cover at least $-50 < v_{lsr} < 160$ \kms, or a velocity $v\pm100$ \kms.  

The morphology is also notable.  Both ends of the outflow are sharply truncated
at $\sim2.5\arcsec$ (0.07 pc) from e2e. To the southeast, the high-velocity
flow lies along a line that is consistent with the extrapolation from the
northwest flow,
but at lower velocities ($10 < v_{LSR} < 45$ \kms), it jogs toward a more
north-south direction.  In the northwest, the redshifted part of this flow ($70
< v_{LSR} < 120$ \kms) apparently collides with a \emph{blue}shifted flow from
another source ($22 < v_{LSR} < 45$ \kms), suggesting that these outflows
intersect, though such a scenario seems probabilistically implausible given
their small volume filling factor.

% really?  important implications?  Maybe not.
The extreme velocity and morphology carry a few important implications for the
accretion process in W51.  The sharp symmetric truncation, combined with the
extraordinary velocity, suggests that the outflow is freshly carving a cavity
in the surrounding dense gas.  The observed velocities are high enough that
their bow shocks likely dissociated all molecules, so some ionized gas is
likely present at the endpoints; this ionized gas has not been detected in
radio images because of the nearby 100 mJy HCHII region e2w.  The dynamical age
of the outflow is $\sim600$ years at the peak observed velocity, which is a
lower limit on the true age of the outflow.

\subsection{e8}
There are at least four distinct outflows coming from the e8 filament.
The e8 core is launching a redshifted outflow to the northwest.  A blueshifted
outflow is coming from somewhere south of the e8 peak and pointing straight
east.  While these originate quite near each other, they seem not to have
a common source, since the red and blue streams are not parallel.

\subsection{north}
The outflow from W51 north is extended and complex.

A jet-like high-velocity feature appears directly to the north of W51 north in
both CO and SO.  However, in SO, this feature begins to emit at $\sim47$ \kms
and continues to $\sim 100$ \kms.  The CO emission below $<70$ \kms is
completely absent, presumably obscured by foreground material.  The blueshifted
component, by contrast with the red, points to the southeast and is barely
detected in CO, but again cleanly in SO.  It is sharply truncated, extending
only $\sim1 \arcsec$ ($\sim5000$ AU).  Unlike the Lacy jet, there is no
evidence that this outflow transitions into an externally ionized state.

The northernmost point of the W51 North outflow may coincide with
the \citet{Hodapp2002a} \hh and [Fe II] outflow.  There is some CO 2-1
emission coincident with the southernmost point of the \hh features,
and these all lay approximately along the W51 North outflow vector.
However, the association is only circumstantial.


\Figure{{figures/Alma1.4mmcont_green_outflows_aplpy_CONTours}.png}
{Outflows in red and blue overlaid on mm continuum in green with cm continuum contours in white.}
{fig:outflowscontinuume2}{1}{12cm}

\Figure{{figures/NACO_green_outflows_aplpy_CONTours_hires_h77acontour}.png}
{Outflows in the W51 IRS2 region.  The green emission is NACO K-band continuum,
with ALMA 1.4 mm continuum contours in white and H77$\alpha$ contours in blue.
The \citet{Lacy2007a} jet is prominent in H77$\alpha$.}
{fig:outflowscontinuumnorth}{1}{12cm}

\Figure{{figures/e2e_CO2-1_channelmaps}.png}
{Channel maps of the e2e outflow in CO 2-1.  The dashed line approximately
connects the northwest and southeast extrema of the flow.}
{fig:e2ecooutflow}{1}{18cm}

\Figure{{figures/e8_CO2-1_channelmaps}.png}
{channel maps of the e8 outflow in CO}
{fig:e8cooutflow}{1}{18cm}

\section{Feedback in W51}
\subsection{Ionizing Radiation}
This was covered in \citep{Ginsburg2016b}.  Ionizing radiation affects much
of the cloud, but little of the prestellar material.  There is no evidence
of increased gas temperatures in the vicinity of \hii regions.  While in
Section \ref{sec:nonionizingradiation} we identify chemically enhanced
regions as those where radiative feedback has heated the dust and released
ices into the gas phase, no such regions are observed surrounding the most
luminous compact \hii regions.

\todo{What direct tests can be used to show that \hii regions aren't heating
their surroundings?  \formaldehyde is good, in principle, but maybe not in
practice because of the possible optical depth issues.  Radio \ammonia might be
OK, but it might also be affected by imaging artifacts from the bright radio
sources.}

\subsection{Outflows}
While the outflows described in Section \ref{sec:outflows} are impressive and
plentiful, they are obviously not the dominant form of feedback, as their area
filling factor is small compared to that of the various forms of radiative
feedback.  A low area filling factor implies a substantially smaller volume
filling factor and therefore a lower overall effect on the cloud.  However,
these outflows likely do punch holes through protostellar envelopes and the
surrounding cloud material, allowing radiation to escape.

\subsection{Non-ionizing Radiation}
\label{sec:nonionizingradiation}
The formed and forming protostars are producing a total $\gtrsim10^7$ \lsun of
far infrared illumination \citep{Ginsburg2016b}.  This radiation heats the
cloud's molecular gas, affecting the initial conditions of future star
formation.

The chemical maps shown in Section \ref{sec:chemistrymaps} show the volumes of
gas clearly affected by newly-forming high-luminosity stars.  The
\methanol-enhanced region around W51e2 extends 0.04 pc, or 8500 AU. Other
locally enhanced species, especially the nitrogenic molecules HNCO and
\formamide, occupy a smaller and more asymmetric region around e2e and e2w 
(Figure \ref{fig:e2methanolhnco}).

The extraordinarily high column densities of \methanol make direct temperature
estimation impossible; many or all of the observed \methanol lines are almost
certainly optically thick.  Fitted rotational diagrams resulted in negative
temperatures throughout the \methanol-enhanced region, implying that the levels
are not thermally populated.  However, the total column densities from these
rotational diagram fits are reasonable lower-limits on the column, and they
exceed $N(\methanol)\gtrsim10^{20}$ \persc.
(demonstrate this?)

\todo{If we assume $T_{ex}=100$ K, can we use the $5_{-4,2}-6_{-4,3}$ line
to measure the column density?}
Out of the five clearly detected \methanol lines, the $5_{-4,2}-6_{-4,3}$
line

\Figure{figures/radialprofile_max_CH3OH_e2.png}
{Radial profiles of the peak surface brightness of five \methanol transitions
along with the profile of the continuum brightness.  The radial profiles were
constructed from images with 0.2\arcsec resolution including only 12m data.
The central dip shows where the lines go into absorption, though they are only
seen in absorption at $\sim55$ \kms.  The \methanol lines are
continuum-subtracted.}
{fig:methanolradialprofile}{1}{12cm}



\todo{How does the \methanol brightness/column profile compare with the dust
brightness/column?  Is it going up faster?  By how much?}

\Figure{figures/W51e2_ch3oh_hnco_continuum_aplpy_kucontours.pdf}
{Image of \methanol $8_{0,8}-7_{1,6}$ (red), HCNO 10-... (green), and 225 GHz
continuum (blue) toward W51e2.  The contours show Ku-band radio continuum
emission tracing the \hii region W51 e2w.  The \methanol emission is
symmetric around the high-mass protostar W51 e2e, suggesting that
this forming star is responsible for heating its surroundings.}
{fig:e2methanolhnco}{1}{12cm}

Is there any evidence that the main-sequence stars that illuminate the \hii
regions in W51 \citep{Ginsburg2016b} affect the pre-star-forming gas throughout
W51?
\todo{Our ALMA program was designed to answer this question by measuring the
temperature in the dense prestellar \formaldehyde-rich gas.  Naively, the data
say "yes, the temperatures are all ridiculously high, $T>100$ K", but that
can't be.  The \formaldehyde temperatures suggest that temperature is correlated
with density, which unfortunately suggests instead that the \formaldehyde
line optical depth is correlated with density.  It is therefore not straightforward
to systematically examine the thermal feedback effects from MYSOs.}

\todo{Notes from chatting with Wing-Fai Thi:  \methanol has a similar condensation
temperature to water, so the desorbed region is probably $\sim90-100$ K.  HNCO
has a much \emph{lower} desorption temperature, so if it was coming from grain
surfaces, it should be more widespread than \methanol.  Since it is not, the
enhancement is most likely due to gas-phase chemistry.}

\todo{
However, Ewine van Dishoeck pointed out that HNCO and \formamide can be mixed
into ices that evaporate at a much higher temperature, consistent with the
structure we observe.
}


\section{Discussion}
\subsection{Limits on accretion onto HII regions}
\citet{Peters2010a} and \citet{Klaassen2012a} proposed that ultra- and
hyper-compact \hii regions may be variably accreting.  When accretion is most
active, the \hii region is confined and shrinks or may even be turned off.
When accretion is slower or weaker, the \hii region expands, following
approximately Str{\"o}mgren expansion \todo{make sure that's actually what they
say...}.  

The observed lack of warm molecular gas around compact \hii regions suggests
that they have not recently been accreting...

d2 provides a counterpoint, however, as it is a hypercompact \hii region that *does*
exhibit enhanced molecular emission in its surroundings


\subsection{Question: Where does the radiation from the HII regions end up?}
The \hii regions show no signs of heating around them.  However, we know that
these must be $>10^4$ \lsun stars, and even the infrared radiation should be
rising with luminosity (or temperature).  While most of the energy might go
into ionizing the gas cloud, many of the photons must get reprocessed into the
infrared at some point.  If optical/NIR photons were escaping, we should be
able to see them unless the geometry is particularly unfavorable.

\ifstandalone
\bibliographystyle{apj_w_etal}  % or "siam", or "alpha", or "abbrv"
%\bibliography{thesis}      % bib database file refs.bib
\bibliography{bibdesk}      % bib database file refs.bib
\fi


\end{document}
