PAPER DRAFT

Cores and Outflows in the W51 Protoclusters



"A multi-phase outflow from a high-mass protostar"


\section{Observations}
As part of ALMA Cycle 2 program 2013.1.00308.S, we observed a
$\sim2\arcmin\times1\arcmin$ region centered between W51 IRS2 and W51 e1/e2
with a 37-pointing mosaic.  Two configurations of the 12m array were used,
achieving a resolution of 0.2\arcsec.  Additionally, a 12-pointing mosaic was
performed using the 7m array, probing scales up to XXX\arcsec.  The full UV
coverage was from XX to YY m.

Data reduction was performed using CASA.  The QA2-produced data products were
combined using the standard inverse variance weighting (check this).
The visibilities were imaged into full spectral cubes at coarse (0.5\arcsec)
resolution in order to get a first look at all 15630 spectral channels.  The
resulting cube was used to identify bright lines in the spectrum extracted
from source e8.  To produce continuum images, frequency channels including
bright lines were excluded.

The spectral setups were ...

A continuum image combining all 4 spectral windows was produced using
\texttt{tclean}.  We phase self-calibrated the image on baselines longer than
100m to increase the dynamic range.  The final image was cleaned with 50000
iterations to a threshold of 5 mJy.  The lowest noise level in the image, away
from bright sources, is $\sim0.2$ mJy/beam, but near the bright sources e2 and
IRS2, the noise reached as high as $\sim2$ mJy/beam.  Deeper cleaning was
attempted, but lead to instabilities.

\subsection{Simulations}
The enhanced noise around bright sources is unavoidable.  We tested the noise
properties of our data set using the CASA \texttt{simobserve} toolkit.  We
obtained a Herschel Gould's Belt Survey image of the Perseus molecular cloud at
250 \um \citep[resolution 18\arcsec][]{} and scaled it down by $\sim40\times$
to match the resolution of our ALMA data.  We used the \texttt{sm.predict} task
to ``observe'' the Herschel data with our exact UV data set.  We then used
\texttt{sm.corrupt} to make the noise properties approximately match those of
our observations.

We performed two simulations to test the observability of the Perseus data.
First, we scaled the surface brightness at 250 \um to 1100 \um by assuming a
blackbody with $T=20$ K and opacity index $\beta=1.5$.  This resulted in only
weak detections of the brightest 3 sources in Perseus, and the noise properties
of the map were excellent and uniform.

Second, we scaled the peak flux density of the Perseus map to be
$\sim100\times$ brighter, or comparable to the flux density of W51e2 in the
real ALMA observations.  In this map, even with deep cleaning, the noise around
the bright sources remains very high.



\section{Analysis}
\subsection{Source Identification}
% dendrogramming.py
We used the \texttt{dendrogram} method described by \citet{Rosolowsky2008a} and
implemented in \texttt{astrodendro} to identify sources.  We used a minimum
value of 1 mJy/beam ($\sim5-\sigma$) and a minimum $\Delta=0.4$ mJy/beam
($\sim2-\sigma$) with minimum 10 pixels (each pixel is 0.05\arcsec).  This
cataloging yielded over 8000 candidate sources, of which the majority are noise
or artifacts around the brightest sources.  To filter out these bad sources,
we created a noise map taking the local RMS of the \texttt{tclean}-produced
residual map over a $\sigma=30$ pixel (1.5\arcsec) gaussian.  We then removed
all sources with peak S/N < 8, mean S/N per pixel $< 5$, and minimum S/N per
pixel $ < 1$.  We also only included the smallest sources in the dendrogram,
the ``leaves''.  These parameters were tuned by checking against ``real''
sources identified by eye and selected using \texttt{ds9}: most real sources are
recovered (but not all; see Section ...) and few spurious sources ($<10$) are
included.  The resulting catalog includes 113 sources.

\subsection{Photometry}
We created a catalog of the dendrogram sources including their peak and mean flux density,
their centroid, and their geometric properties.  For each source, we further extracted aperture
photometry around the centroid in 6 apertures: 0.2, 0.4, 0.6, 0.8, 1.0, and 1.5\arcsec.
We performed the same aperture photometry on the W51 Ku-band images from \citet{Ginsburg2016a}.
These observations are reported in Table {...}.

\subsection{Spectral Lines \& Velocities}
To determine the line-of-sight velocity of each source, we extracted a spectrum
from an 0.5\arcsec aperture centered on the source and from a 0.5-1.0\arcsec
annulus around it.  We then searched each spectrum for the brightest pixel and
associated it with the likeliest spectral line.  We repeated this in each of
our 4 spectral windows, then averaged the 4 velocities to get an estimate of
the source velocity.   This process also allowed us to identify the brightest
lines in each window and the brightest overall line observed, which we use
later for temperature estimation.

