\begin{table*}[htp]
\caption{Spectral Lines in SPW 0}
\begin{tabular}{ll}
\label{tab:linesspw0}
Line Name & Frequency \\
 & $\mathrm{GHz}$ \\
\hline
H$_2$CO $3_{0,3}-2_{0,2}$ & 218.22219 \\
H$_2$CO $3_{2,2}-2_{2,1}$ & 218.47564 \\
CH$_3$OH $4_{2,2}-3_{1,2}$ & 218.44005 \\
CH$_3$OCHO $17_{3,14}-16_{3,13}$E & 218.28083 \\
CH$_3$OCHO $17_{3,14}-16_{3,13}$A & 218.29787 \\
CH$_3$CH$_2$CN $24_{3,21}-23_{3,20}$ & 218.39002 \\
Acetone $8_{7,1}-7_{4,4}$AE & 218.24017 \\
O$^{13}$CS 18-17 & 218.19898 \\
CH$_3$OCH$_3$ $23_{3,21}-23_{2,22}$AA & 218.49441 \\
CH$_3$OCH$_3$ $23_{3,21}-23_{2,22}$EE & 218.49192 \\
CH$_3$NCO $25_{1,24} - 24_{1,23}$ & 218.5418 \\
CH$_3$SH $23_2-23_1$ & 218.18612 \\
\hline
\end{tabular}
\par
The Categories column consists of three letter codes as described in Section \ref{sec:contsourcenature}.In column 1, \texttt{F} indicates a free-free dominated source,\texttt{f} indicates significant free-free contribution,and \texttt{-} means there is no detected cm continuum.In column 2, the peak brightness temperature is used toclassify the temperature category.\texttt{H} is `hot' ($T>50$ K), \texttt{C} is `cold' ($T<20$ K), and \texttt{-} is indeterminate (either $20<T<50$K or no measurement)In column 3, \texttt{c} indicates compact sources, and \texttt{-} indicates a diffuse source.
\end{table*}
