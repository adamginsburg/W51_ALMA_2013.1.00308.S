\begin{table*}[htp]
\caption{Spectral Lines in SPW 3}
\begin{tabular}{ll}
\label{tab:linesspw3}
Line Name & Frequency \\
 & $\mathrm{GHz}$ \\
\hline
CH$_3$OH $4_{2,3}-5_{1,4}$ & 234.68345 \\
CH$_3$OH $5_{-4,2}-6_{-4,3}$ & 234.69847 \\
CH$_3$OH $18_{3,15}-17_{4,14}$ & 233.79575 \\
$^{13}$CH$_3$OH $5_{1,5}-4_{1,4}$ & 234.01158 \\
PN $5-4$ & 234.93569 \\
NH$_2$CHO $11_{5,6}-10_{5,5}$ & 233.59451 \\
Acetone $12_{11,2}-11_{10,1}$AE & 234.86136 \\
SO$_2$ $16_{6,10}-17_{5,13}$ & 234.42159 \\
CH$_3$NCO $27_{2,26} - 26_{2,25}$ & 234.08812 \\
CH$_3$SH $15_2-15_1$ & 234.19145 \\
\hline
\end{tabular}
\par
The Categories column consists of three letter codes as described in Section \ref{sec:contsourcenature}.In column 1, \texttt{F} indicates a free-free dominated source,\texttt{f} indicates significant free-free contribution,and \texttt{-} means there is no detected cm continuum.In column 2, the peak brightness temperature is used toclassify the temperature category.\texttt{H} is `hot' ($T>50$ K), \texttt{C} is `cold' ($T<20$ K), and \texttt{-} is indeterminate (either $20<T<50$K or no measurement)In column 3, \texttt{c} indicates compact sources, and \texttt{-} indicates a diffuse source.
\end{table*}
