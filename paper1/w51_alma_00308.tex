\documentclass{emulateapj}
%\documentclass[defaultstyle,11pt]{thesis}
%\documentclass[]{report}
%\documentclass[]{article}
%\usepackage{aastex_hack}
%\usepackage{deluxetable}
%\documentclass[preprint]{aastex}
%\documentclass{aa}
\newcommand\arcdeg{\mbox{$^\circ$}\xspace} 

\pdfminorversion=4


%%%%%%%%%%%%%%%%%%%%%%%%%%%%%%%%%%%%%%%%%%%%%%%%%%%%%%%%%%%%%%%%
%%%%%%%%%%%  see documentation for information about  %%%%%%%%%%
%%%%%%%%%%%  the options (11pt, defaultstyle, etc.)   %%%%%%%%%%
%%%%%%%  http://www.colorado.edu/its/docs/latex/thesis/  %%%%%%%
%%%%%%%%%%%%%%%%%%%%%%%%%%%%%%%%%%%%%%%%%%%%%%%%%%%%%%%%%%%%%%%%
%		\documentclass[typewriterstyle]{thesis}
% 		\documentclass[modernstyle]{thesis}
% 		\documentclass[modernstyle,11pt]{thesis}
%	 	\documentclass[modernstyle,12pt]{thesis}

%%%%%%%%%%%%%%%%%%%%%%%%%%%%%%%%%%%%%%%%%%%%%%%%%%%%%%%%%%%%%%%%
%%%%%%%%%%%    load any packages which are needed    %%%%%%%%%%%
%%%%%%%%%%%%%%%%%%%%%%%%%%%%%%%%%%%%%%%%%%%%%%%%%%%%%%%%%%%%%%%%
\usepackage{latexsym}		% to get LASY symbols
\usepackage{graphicx}		% to insert PostScript figures
%\usepackage{deluxetable}
\usepackage{rotating}		% for sideways tables/figures
\usepackage{natbib}  % Requires natbib.sty, available from http://ads.harvard.edu/pubs/bibtex/astronat/
\usepackage{savesym}
\usepackage{pdflscape}
\usepackage{amssymb}
\usepackage{morefloats}
%\savesymbol{singlespace}
\savesymbol{doublespace}
%\usepackage{wrapfig}
%\usepackage{setspace}
\usepackage{xspace}
\usepackage{color}
\usepackage{multicol}
\usepackage{mdframed}
\usepackage{url}
\usepackage{subfigure}
%\usepackage{emulateapj}
\usepackage{lscape}
\usepackage{grffile}
\usepackage{standalone}
\standalonetrue
\usepackage{import}
\usepackage[utf8]{inputenc}
\usepackage{longtable}
\usepackage{booktabs}
\usepackage[yyyymmdd,hhmmss]{datetime}
\usepackage{fancyhdr}
\usepackage[colorlinks=true,citecolor=blue,linkcolor=cyan]{hyperref}
\usepackage{ifpdf}






%\renewcommand\ion[2]{#1$\;${%
%\ifx\@currsize\normalsize\small \else
%\ifx\@currsize\small\footnotesize \else
%\ifx\@currsize\footnotesize\scriptsize \else
%\ifx\@currsize\scriptsize\tiny \else
%\ifx\@currsize\large\normalsize \else
%\ifx\@currsize\Large\large
%\fi\fi\fi\fi\fi\fi
%\rmfamily\@Roman{#2}}\relax}% 

\newcommand{\paa}{Pa\ensuremath{\alpha}}
\newcommand{\brg}{Br\ensuremath{\gamma}}
\newcommand{\msun}{\ensuremath{M_{\odot}}\xspace}			%  Msun
\newcommand{\mdot}{\ensuremath{\dot{M}}\xspace}
\newcommand{\lsun}{\ensuremath{L_{\odot}}\xspace}			%  Lsun
\newcommand{\rsun}{\ensuremath{R_{\odot}}\xspace}			%  Rsun
\newcommand{\lbol}{\ensuremath{L_{\mathrm{bol}}\xspace}}	%  Lbol
\newcommand{\ks}{K\ensuremath{_{\mathrm{s}}}}		%  Ks
\newcommand{\hh}{\ensuremath{\textrm{H}_{2}}\xspace}			%  H2
\newcommand{\dens}{\ensuremath{n(\hh) [\percc]}\xspace}
\newcommand{\formaldehyde}{\ensuremath{\textrm{H}_2\textrm{CO}}\xspace}
\newcommand{\formamide}{\ensuremath{\textrm{NH}_2\textrm{CHO}}\xspace}
\newcommand{\formaldehydeIso}{\ensuremath{\textrm{H}_2~^{13}\textrm{CO}}\xspace}
\newcommand{\methanol}{\ensuremath{\textrm{CH}_3\textrm{OH}}\xspace}
\newcommand{\ortho}{\ensuremath{\textrm{o-H}_2\textrm{CO}}\xspace}
\newcommand{\para}{\ensuremath{\textrm{p-H}_2\textrm{CO}}\xspace}
\newcommand{\oneone}{\ensuremath{1_{1,0}-1_{1,1}}\xspace}
\newcommand{\twotwo}{\ensuremath{2_{1,1}-2_{1,2}}\xspace}
\newcommand{\threethree}{\ensuremath{3_{1,2}-3_{1,3}}\xspace}
\newcommand{\threeohthree}{\ensuremath{3_{0,3}-2_{0,2}}\xspace}
\newcommand{\threetwotwo}{\ensuremath{3_{2,2}-2_{2,1}}\xspace}
\newcommand{\threetwoone}{\ensuremath{3_{2,1}-2_{2,0}}\xspace}
\newcommand{\fourtwotwo}{\ensuremath{4_{2,2}-3_{1,2}}\xspace} % CH3OH 218.4 GHz
\newcommand{\methylcyanide}{\ensuremath{\textrm{CH}_{3}\textrm{CN}}\xspace}
\newcommand{\ketene}{\ensuremath{\textrm{H}_{2}\textrm{CCO}}\xspace}
\newcommand{\ethylcyanide}{\ensuremath{\textrm{CH}_3\textrm{CH}_2\textrm{CN}}\xspace}
\newcommand{\cyanoacetylene}{\ensuremath{\textrm{HC}_{3}\textrm{N}}\xspace}
\newcommand{\methylformate}{\ensuremath{\textrm{CH}_{3}\textrm{OCHO}}\xspace}
\newcommand{\dimethylether}{\ensuremath{\textrm{CH}_{3}\textrm{OCH}_{3}}\xspace}
\newcommand{\gaucheethanol}{\ensuremath{\textrm{g-CH}_3\textrm{CH}_2\textrm{OH}}\xspace}
\newcommand{\acetone}{\ensuremath{\left[\textrm{CH}_{3}\right]_2\textrm{CO}}\xspace}
\newcommand{\methyleneamidogen}{\ensuremath{\textrm{H}_{2}\textrm{CN}}\xspace}
\newcommand{\Rone}{\ensuremath{\para~S_{\nu}(\threetwoone) / S_{\nu}(\threeohthree)}\xspace}
\newcommand{\Rtwo}{\ensuremath{\para~S_{\nu}(\threetwotwo) / S_{\nu}(\threetwoone)}\xspace}
\newcommand{\JKaKc}{\ensuremath{J_{K_a K_c}}}
\newcommand{\water}{H$_{2}$O\xspace}		%  H2O
\newcommand{\feii}{\ion{Fe}{2}}		%  FeII
\newcommand{\uchii}{\ion{UCH}{ii}\xspace}
\newcommand{\UCHII}{\ion{UCH}{ii}\xspace}
\newcommand{\hchii}{\ion{HCH}{ii}\xspace}
\newcommand{\HCHII}{\ion{HCH}{ii}\xspace}
\newcommand{\hii}{\ion{H}{ii}\xspace}
\newcommand{\hi}{H~{\sc i}\xspace}
\newcommand{\Hii}{\hii}
\newcommand{\HII}{\hii}
\newcommand{\Xform}{\ensuremath{X_{\formaldehyde}}}
\newcommand{\kms}{\textrm{km~s}\ensuremath{^{-1}}\xspace}	%  km s-1
\newcommand{\nsample}{456\xspace}
\newcommand{\CFR}{5\xspace} % nMPC / 0.25 / 2 (6 for W51 once, 8 for W51 twice) REFEDIT: With f_observed=0.3, becomes 3/2./0.3 = 5
\newcommand{\permyr}{\ensuremath{\mathrm{Myr}^{-1}}\xspace}
\newcommand{\pers}{\ensuremath{\mathrm{s}^{-1}}\xspace}
\newcommand{\tsuplim}{0.5\xspace} % upper limit on starless timescale
\newcommand{\ncandidates}{18\xspace}
\newcommand{\mindist}{8.7\xspace}
\newcommand{\rcluster}{2.5\xspace}
\newcommand{\ncomplete}{13\xspace}
\newcommand{\middistcut}{13.0\xspace}
\newcommand{\nMPC}{3\xspace} % only count W51 once.  W51, W49, G010
\newcommand{\obsfrac}{30}
\newcommand{\nMPCtot}{10\xspace} % = nmpc / obsfrac
\newcommand{\nMPCtoterr}{6\xspace} % = sqrt(nmpc) / obsfrac
\newcommand{\plaw}{2.1\xspace}
\newcommand{\plawerr}{0.3\xspace}
\newcommand{\mmin}{\ensuremath{10^4~\msun}\xspace}
%\newcommand{\perkmspc}{\textrm{per~km~s}\ensuremath{^{-1}}\textrm{pc}\ensuremath{^{-1}}\xspace}	%  km s-1 pc-1
\newcommand{\kmspc}{\textrm{km~s}\ensuremath{^{-1}}\textrm{pc}\ensuremath{^{-1}}\xspace}	%  km s-1 pc-1
\newcommand{\sqcm}{cm$^{2}$\xspace}		%  cm^2
\newcommand{\percc}{\ensuremath{\textrm{cm}^{-3}}\xspace}
\newcommand{\perpc}{\ensuremath{\textrm{pc}^{-1}}\xspace}
\newcommand{\persc}{\ensuremath{\textrm{cm}^{-2}}\xspace}
\newcommand{\persr}{\ensuremath{\textrm{sr}^{-1}}\xspace}
\newcommand{\peryr}{\ensuremath{\textrm{yr}^{-1}}\xspace}
\newcommand{\perkmspc}{\textrm{km~s}\ensuremath{^{-1}}\textrm{pc}\ensuremath{^{-1}}\xspace}	%  km s-1 pc-1
\newcommand{\perkms}{\textrm{per~km~s}\ensuremath{^{-1}}\xspace}	%  km s-1 
\newcommand{\um}{\ensuremath{\mu \textrm{m}}\xspace}    % micron
\newcommand{\microjy}{\ensuremath{\mu\textrm{Jy}}\xspace}    % micron
\newcommand{\mum}{\um}
\newcommand{\htwo}{\ensuremath{\textrm{H}_2}}
\newcommand{\Htwo}{\ensuremath{\textrm{H}_2}}
\newcommand{\HtwoO}{\ensuremath{\textrm{H}_2\textrm{O}}}
\newcommand{\htwoo}{\ensuremath{\textrm{H}_2\textrm{O}}}
\newcommand{\ha}{\ensuremath{\textrm{H}\alpha}}
\newcommand{\hb}{\ensuremath{\textrm{H}\beta}}
\newcommand{\so}{SO~\ensuremath{5_6-4_5}\xspace}
\newcommand{\SO}{SO~\ensuremath{1_2-1_1}\xspace}
\newcommand{\ammonia}{NH\ensuremath{_3}\xspace}
\newcommand{\twelveco}{\ensuremath{^{12}\textrm{CO}}\xspace}
\newcommand{\thirteenco}{\ensuremath{^{13}\textrm{CO}}\xspace}
\newcommand{\ceighteeno}{\ensuremath{\textrm{C}^{18}\textrm{O}}\xspace}
\def\ee#1{\ensuremath{\times10^{#1}}}
\newcommand{\degrees}{\ensuremath{^{\circ}}}
% can't have \degree because I'm getting a degree...
\newcommand{\lowirac}{800}
\newcommand{\highirac}{8000}
\newcommand{\lowmips}{600}
\newcommand{\highmips}{5000}
\newcommand{\perbeam}{\ensuremath{\textrm{beam}^{-1}}}
\newcommand{\ds}{\ensuremath{\textrm{d}s}}
\newcommand{\dnu}{\ensuremath{\textrm{d}\nu}}
\newcommand{\dv}{\ensuremath{\textrm{d}v}}
\def\secref#1{Section \ref{#1}}
\def\eqref#1{Equation \ref{#1}}
\def\facility#1{#1}
%\newcommand{\arcmin}{'}

\newcommand{\necluster}{Sh~2-233IR~NE}
\newcommand{\swcluster}{Sh~2-233IR~SW}
\newcommand{\region}{IRAS 05358}

\newcommand{\nwfive}{40}
\newcommand{\nouter}{15}

\newcommand{\vone}{{\rm v}1.0\xspace}
\newcommand{\vtwo}{{\rm v}2.0\xspace}
\newcommand\mjysr{\ensuremath{{\rm MJy~sr}^{-1}}}
\newcommand\jybm{\ensuremath{{\rm Jy~bm}^{-1}}}
\newcommand\nbolocat{8552\xspace}
\newcommand\nbolocatnew{548\xspace}
\newcommand\nbolocatnonew{8004\xspace} % = nbolocat-nbolocatnew
\renewcommand\arcdeg{\mbox{$^\circ$}\xspace} 
\renewcommand\arcmin{\mbox{$^\prime$}\xspace} 
\renewcommand\arcsec{\mbox{$^{\prime\prime}$}\xspace} 

\newcommand{\todo}[1]{\textcolor{red}{#1}}
\newcommand{\okinfinal}[1]{{#1}}
%% only needed if not aastex
%\newcommand{\keywords}[1]{}
%\newcommand{\email}[1]{}
%\newcommand{\affil}[1]{}


%aastex hack
%\newcommand\arcdeg{\mbox{$^\circ$}}%
%\newcommand\arcmin{\mbox{$^\prime$}\xspace}%
%\newcommand\arcsec{\mbox{$^{\prime\prime}$}\xspace}%

%\newcommand\epsscale[1]{\gdef\eps@scaling{#1}}
%
%\newcommand\plotone[1]{%
% \typeout{Plotone included the file #1}
% \centering
% \leavevmode
% \includegraphics[width={\eps@scaling\columnwidth}]{#1}%
%}%
%\newcommand\plottwo[2]{{%
% \typeout{Plottwo included the files #1 #2}
% \centering
% \leavevmode
% \columnwidth=.45\columnwidth
% \includegraphics[width={\eps@scaling\columnwidth}]{#1}%
% \hfil
% \includegraphics[width={\eps@scaling\columnwidth}]{#2}%
%}}%


%\newcommand\farcm{\mbox{$.\mkern-4mu^\prime$}}%
%\let\farcm\farcm
%\newcommand\farcs{\mbox{$.\!\!^{\prime\prime}$}}%
%\let\farcs\farcs
%\newcommand\fp{\mbox{$.\!\!^{\scriptscriptstyle\mathrm p}$}}%
%\newcommand\micron{\mbox{$\mu$m}}%
%\def\farcm{%
% \mbox{.\kern -0.7ex\raisebox{.9ex}{\scriptsize$\prime$}}%
%}%
%\def\farcs{%
% \mbox{%
%  \kern  0.13ex.%
%  \kern -0.95ex\raisebox{.9ex}{\scriptsize$\prime\prime$}%
%  \kern -0.1ex%
% }%
%}%

\def\Figure#1#2#3#4#5{
\begin{figure*}[!htp]
\includegraphics[scale=#4,width=#5]{#1}
\caption{#2}
\label{#3}
\end{figure*}
}

\def\WrapFigure#1#2#3#4#5#6{
\begin{wrapfigure}{#6}{0.5\textwidth}
\includegraphics[scale=#4,width=#5]{#1}
\caption{#2}
\label{#3}
\end{wrapfigure}
}

% % #1 - filename
% % #2 - caption
% % #3 - label
% % #4 - epsscale
% % #5 - R or L?
% \def\WrapFigure#1#2#3#4#5#6{
% \begin{wrapfigure}[#6]{#5}{0.45\textwidth}
% %  \centercaption
% %  \vspace{-14pt}
%   \epsscale{#4}
%   \includegraphics[scale=#4]{#1}
%   \caption{#2}
%   \label{#3}
% \end{wrapfigure}
% }

\def\RotFigure#1#2#3#4#5{
\begin{sidewaysfigure*}[!htp]
\includegraphics[scale=#4,width=#5]{#1}
\caption{#2}
\label{#3}
\end{sidewaysfigure*}
}

\def\FigureSVG#1#2#3#4{
\begin{figure*}[!htp]
    \def\svgwidth{#4}
    \input{#1}
    \caption{#2}
    \label{#3}
\end{figure*}
}

% originally intended to be included in a two-column paper
% this is in includegraphics: ,width=3in
% but, not for thesis
\def\OneColFigure#1#2#3#4#5{
\begin{figure}[!htpb]
\epsscale{#4}
\includegraphics[scale=#4,angle=#5]{#1}
\caption{#2}
\label{#3}
\end{figure}
}

\def\SubFigure#1#2#3#4#5{
\begin{figure*}[!htp]
\addtocounter{figure}{-1}
\epsscale{#4}
\includegraphics[angle=#5]{#1}
\caption{#2}
\label{#3}
\end{figure*}
}

%\def\FigureTwo#1#2#3#4#5{
%\begin{figure*}[!htp]
%\epsscale{#5}
%\plottwo{#1}{#2}
%\caption{#3}
%\label{#4}
%\end{figure*}
%}

\def\FigureTwo#1#2#3#4#5#6{
\begin{figure*}[!htp]
\subfigure[]{ \includegraphics[scale=#5,width=#6]{#1} }
\subfigure[]{ \includegraphics[scale=#5,width=#6]{#2} }
\caption{#3}
\label{#4}
\end{figure*}
}

\def\FigureTwoAA#1#2#3#4#5#6{
\begin{figure*}[!htp]
\subfigure[]{ \includegraphics[scale=#5,width=#6]{#1} }
\subfigure[]{ \includegraphics[scale=#5,width=#6]{#2} }
\caption{#3}
\label{#4}
\end{figure*}
}

\newenvironment{rotatepage}%
{}{}
   %{\pagebreak[4]\afterpage\global\pdfpageattr\expandafter{\the\pdfpageattr/Rotate 90}}%
   %{\pagebreak[4]\afterpage\global\pdfpageattr\expandafter{\the\pdfpageattr/Rotate 0}}%


\def\RotFigureTwoAA#1#2#3#4#5#6{
\begin{rotatepage}
\begin{sidewaysfigure*}[!htp]
\subfigure[]{ \includegraphics[scale=#5,width=#6]{#1} }
\\
\subfigure[]{ \includegraphics[scale=#5,width=#6]{#2} }
\caption{#3}
\label{#4}
\end{sidewaysfigure*}
\end{rotatepage}
}

\def\RotFigureThreeAA#1#2#3#4#5#6#7{
\begin{rotatepage}
\begin{sidewaysfigure*}[!htp]
\subfigure[]{ \includegraphics[scale=#6,width=#7]{#1} }
\\
\subfigure[]{ \includegraphics[scale=#6,width=#7]{#2} }
\\
\subfigure[]{ \includegraphics[scale=#6,width=#7]{#3} }
\caption{#4}
\label{#5}
\end{sidewaysfigure*}
\end{rotatepage}
\clearpage
}

\def\FigureThreeAA#1#2#3#4#5#6#7{
\begin{figure*}[!htp]
\subfigure[]{ \includegraphics[scale=#6,width=#7]{#1} }
\subfigure[]{ \includegraphics[scale=#6,width=#7]{#2} }
\subfigure[]{ \includegraphics[scale=#6,width=#7]{#3} }
\caption{#4}
\label{#5}
\end{figure*}
}



\def\SubFigureTwo#1#2#3#4#5{
\begin{figure*}[!htp]
\addtocounter{figure}{-1}
\epsscale{#5}
\plottwo{#1}{#2}
\caption{#3}
\label{#4}
\end{figure*}
}

\def\FigureFour#1#2#3#4#5#6{
\begin{figure*}[!htp]
\subfigure[]{ \includegraphics[width=3in]{#1} }
\subfigure[]{ \includegraphics[width=3in]{#2} }
\subfigure[]{ \includegraphics[width=3in]{#3} }
\subfigure[]{ \includegraphics[width=3in]{#4} }
\caption{#5}
\label{#6}
\end{figure*}
}

\def\FigureFourPDF#1#2#3#4#5#6{
\begin{figure*}[!htp]
\subfigure[]{ \includegraphics[width=3in,type=pdf,ext=.pdf,read=.pdf]{#1} }
\subfigure[]{ \includegraphics[width=3in,type=pdf,ext=.pdf,read=.pdf]{#2} }
\subfigure[]{ \includegraphics[width=3in,type=pdf,ext=.pdf,read=.pdf]{#3} }
\subfigure[]{ \includegraphics[width=3in,type=pdf,ext=.pdf,read=.pdf]{#4} }
\caption{#5}
\label{#6}
\end{figure*}
}

\def\FigureThreePDF#1#2#3#4#5{
\begin{figure*}[!htp]
\subfigure[]{ \includegraphics[width=3in,type=pdf,ext=.pdf,read=.pdf]{#1} }
\subfigure[]{ \includegraphics[width=3in,type=pdf,ext=.pdf,read=.pdf]{#2} }
\subfigure[]{ \includegraphics[width=3in,type=pdf,ext=.pdf,read=.pdf]{#3} }
\caption{#4}
\label{#5}
\end{figure*}
}

\def\Table#1#2#3#4#5{
%\renewcommand{\thefootnote}{\alph{footnote}}
\begin{table}
\caption{#2}
\label{#3}
    \begin{tabular}{#1}
        \hline\hline
        #4
        \hline
        #5
        \hline
    \end{tabular}
\end{table}
%\renewcommand{\thefootnote}{\arabic{footnote}}
}


%\def\Table#1#2#3#4#5#6{
%%\renewcommand{\thefootnote}{\alph{footnote}}
%\begin{deluxetable}{#1}
%\tablewidth{0pt}
%\tabletypesize{\footnotesize}
%\tablecaption{#2}
%\tablehead{#3}
%\startdata
%\label{#4}
%#5
%\enddata
%\bigskip
%#6
%\end{deluxetable}
%%\renewcommand{\thefootnote}{\arabic{footnote}}
%}

%\def\tablenotetext#1#2{
%\footnotetext[#1]{#2}
%}

\def\LongTable#1#2#3#4#5#6#7#8{
% required to get tablenotemark to work: http://www2.astro.psu.edu/users/stark/research/psuthesis/longtable.html
\renewcommand{\thefootnote}{\alph{footnote}}
\begin{longtable}{#1}
\caption[#2]{#2}
\label{#4} \\

 \\
\hline 
#3 \\
\hline
\endfirsthead

\hline
#3 \\
\hline
\endhead

\hline
\multicolumn{#8}{r}{{Continued on next page}} \\
\hline
\endfoot

\hline 
\endlastfoot
#7 \\

#5
\hline
#6 \\

\end{longtable}
\renewcommand{\thefootnote}{\arabic{footnote}}
}

\def\TallFigureTwo#1#2#3#4#5#6{
\begin{figure*}[htp]
\epsscale{#5}
\subfigure[]{ \includegraphics[width=#6]{#1} }
\subfigure[]{ \includegraphics[width=#6]{#2} }
\caption{#3}
\label{#4}
\end{figure*}
}

		% file containing author's macro definitions

\begin{document}
%\title{Cores and Outflows and Chemistry in the W51 Protoclusters}
\title{Thermal Feedback in the high-mass star and cluster forming region W51:
       Massive Protostars like their food well-done}
\titlerunning{Thermal Feedback W51}
\authorrunning{Ginsburg et al}
% for future reference, this is probably a better approach:
% https://github.com/dfm/peerless/blob/af483ced97045c213650ed807c430b2f87d2c8c9/document/ms.tex#L104
% assuming it's compatible with A&A
\newcommand{\eso}{$^{1}$}
\newcommand{\nrao}{$^{2}$}
\newcommand{\radboud}{$^{3}$}
\newcommand{\morelia}{$^{5}$}
\newcommand{\allegro}{$^{4}$}
\newcommand{\excellence}{$^{6}$}
\newcommand{\casa}{$^{7}$}
\newcommand{\cfa}{$^{8}$}
\newcommand{\lasp}{$^{9}$}
\newcommand{\sofia}{$^{*}$}
\newcommand{\jodrell}{$^{*}$}
\newcommand{\iah}{$^{*}$}
\newcommand{\mpia}{$^{*}$}
\newcommand{\zah}{$^{*}$}
\newcommand{\exclus}{$^{*}$}
\newcommand{\arcetri}{$^{*}$}
\newcommand{\uofa}{$^{*}$}
\newcommand{\mpe}{$^{*}$}
\newcommand{\ucsd}{$^{*}$}
\newcommand{\ljmu}{$^{*}$}
\newcommand{\herts}{$^{*}$}


\author{
Adam Ginsburg{\eso,\nrao},
% W.~M. Goss{\nrao},
Ciriaco Goddi{\radboud$^{,}$\allegro},
Roberto Galv{\'a}n-Madrid{\morelia},
James E. Dale{\herts},
John Bally{\casa},
% Cara D.  Battersby{\cfa},
% (not joining) Allison Youngblood{\lasp},
% Ravi Sankrit{\sofia},
Rowan Smith{\jodrell},
Jeremy Darling{\casa},
J.~M.~Diederik Kruijssen{\zah},
% Hauyu Baobab Liu{\eso}
Erik Rosolowsky {\uofa},
% (not joining) Malcolm Walmsley {}
Robert Loughnane{\morelia},
Leonardo Testi {\eso,\arcetri,\exclus},
Nate Bastian {\ljmu},
Jaime E. Pineda {\mpe},
% Ke Wang {\eso}
% Joao Alves {\vienna}
Elisabeth Mills {\ucsd}\\
\begin{flushleft}
\institutions
\end{flushleft}
        }

\institute{
    {\eso}{
           \it{
               European Southern Observatory, Karl-Schwarzschild-Stra{\ss}e 2, D-85748 Garching bei München, Germany
               }
           } \\ 
    %{\saudi}{\it{Astron. Dept., King Abdulaziz University, P.O. Box 80203,
    %Jeddah 21589, Saudi Arabia}}\\
    %%{\edmonton}{\it{University of Alberta, Department of Physics, 4-181 CCIS, Edmonton AB T6G 2E1 Canada}} \\ 
    %{\yale}{\it{Department of Astronomy, Yale University, P.O. Box 208101, New Haven, CT 06520-8101 USA}} \\ 
    %%{\puertorico}{\it{Department of Physical Sciences, University of Puerto Rico, P.O. Box 23323, San Juan, PR 00931}}
    %{\mpifr}{\it{Max Planck Institute for Radio Astronomy, auf dem Hugel, Bonn}}
    {\nrao}{\it{National Radio Astronomy Observatory, Socorro, NM 87801 USA\\
                      \email{aginsbur@nrao.edu} 
                      }} \\
    {\mpe}{\it{Max-Planck-Institut f\"ur extraterrestrische Physik, Giessenbachstrasse 1, 85748 Garching, Germany}}\\
    {\arcetri}{ \it{ INAF-Osservatorio Astrofisico di Arcetri, Largo E. Fermi 5, I-50125, Florence, Italy } } \\ 
    {\exclus}{ \it{ Excellence Cluster Universe, Boltzman str. 2, D-85748 Garching bei M\"unchen, Germany } } \\
    %{\oxford}{\it{Oxford}}
    %{\chalmers}{\it{Chalmers}}
    {\radboud}{\it{Department of Astrophysics/IMAPP, Radboud University Nijmegen, PO Box 9010, 6500 GL Nijmegen, the Netherlands}} \\
    {\allegro}{\it{ALLEGRO/Leiden Observatory, Leiden University, PO Box 9513, 2300 RA Leiden, the Netherlands}} \\
    {\morelia}{\it{Instituto de Radioastronom{\'i}a y Astrof{\'i}sica, UNAM, A.P. 3-72, Xangari, Morelia, 58089, Mexico}} \\
    %{\excellence}{\it{University Observatory/Excellence Cluster `Universe' Scheinerstra{\ss}e 1, 81679 M{\"u}nchen, Germany}} \\
    {\herts}{\it{Centre for Astrophysics Research, University of Hertfordshire, College Lane, Hatfield, AL10 9AB, UK}}\\
    {\casa}{\it{CASA, University of Colorado, 389-UCB, Boulder, CO 80309}} \\ 
    %{\cfa}{\it{Harvard-Smithsonian Center for Astrophysics, 60 Garden
    %           Street, Cambridge, MA 02138, USA}} \\ 
    %{\lasp}{\it{LASP, University of Colorado, 600 UCB, Boulder, CO 80309}}\\
    %{\sofia}{\it{SOFIA Science Center, NASA Ames Research Center, M/S 232-12, Moffett Field, CA 94035, USA}}\\
    {\jodrell}{\it{Jodrell Bank Centre for Astrophysics, School of Physics and Astronomy, University of Manchester, Oxford Road, Manchester M13 9PL, UK}} \\
    %{\mpia}{\it{Max-Planck Institut f{\"u}r Astrophysik, Karl-Schwarzschild-Stra{\ss}e 1, 85748 Garching, Germany}} \\
    {\zah}{\it{Astronomisches Rechen-Institut, Zentrum f{\"u}r Astronomie der Universit{\"a}t Heidelberg, M{\"o}nchhofstra{\ss}e 12-14, 69120 Heidelberg, Germany}}\\
    {\uofa}{\it{Dept. of Physics, University of Alberta, Edmonton, Alberta, Canada}}\\
    {\ljmu{\it{ Astrophysics Research Institute, Liverpool John Moores University, 146 Brownlow Hill, Liverpool L3 5RF, UK }}}\\
    {\ucsd{\it{\todo{University of California, San Diego TODO finish this}}}}
    }


%\abstract{
\begin{abstract}
    We present ALMA observations of a $3\times1.5$ pc area in the W51
    high-mass star-forming complex.  We identify dust continuum sources and
    measure the gas and dust temperature through both rotational diagram
    modeling of \methanol and brightness-temperature-based limits.  The
    observed region contains three high-mass YSOs that appear to be at the
    earliest stages of their formation, with no signs of ionizing radiation
    from their central sources.  The new data reveal  high gas and dust
    temperatures ($T > 100$ K) extending out to about 5000 AU from each of
    these sources.
    %, indicating that the forming MYSOs are able to heat a large
    %volume of gas in which fragmentation is suppressed. 
    The extended warm gas provides evidence that, during the process of
    forming, these high-mass stars heat a large volume (and correspondingly
    large mass) of gas in their surroundings, suppressing fragmentation and
    therefore keeping a large reservoir available to feed from.  By contrast,
    the more mature massive stars that illuminate compact \hii regions have
    little effect on their surrounding dense gas, suggesting that these main
    sequence stars have completed most or all of their accretion.  
    The high luminosity of the massive protostars, combined with a lack of
    centimeter continuum from these sources, implies that they are not on
    the main sequence while they accrete the majority of the mass; instead,
    they may be bloated and cool.
    
    %These warm cores are presently stable against Jeans
    %fragmentation, but at their current density and lower temperature, they
    %would not have been, so they must have been assembled from a larger volume
    %of gas.

    %No distinct fragments
    %are observed within this heated zone, though 75 sources representing either
    %prestellar cores or some stage of protostars are detected in the observed
    %field. 

    %These observations support a variant of a 
    %monolithic core accretion model for high-mass star formation, 
    %(The observations provide a solution
    %to the `fragmentation-induced starvation' scenario, which might otherwise have
    %limited massive star growth even further than direct feedback.)
\end{abstract}

\ifpdf
\maketitle
\fi

%\section{Paper 1 to-do list}
%\begin{enumerate}
%    %\item Make a photometry table - one from the hand-extracted, one from
%    %    dendrogram-extracted?  Section \ref{sec:photometry}
%    %    see make\_photometry\_table
%    %\item MOVE TO DISKS PAPER Is any discussion of the gas kinematics warranted?  That should
%    %    perhaps be relegated to the (lack of?) disks paper
%    % \item DONE create Table ref{tab:linelist} = \ref{tab:linelist} (Section
%    %     \ref{sec:temperature})
%    %\item DONE Show a `core' mass histogram in Section \ref{sec:temperature}
%    %\item DONE Repeat Section \ref{sec:W51e2e} for e8 and north
%    %\item DONE - commented out: move Section \ref{sec:cheminterp} to discussion
%    %\item Add a section ``building on Ginsburg+ 2016''
%    %    (what did I mean by this?  looking at the feedback in IRS2? nice idea
%    %    but no longer fits well with the dominant theme)
%    %\item (for another paper): detailed SED study of e2w.  What is its
%    %    luminosity?  spectral type?  What about e2e?  the latter is more appropriate
%    %    for this paper
%    %\item for the large scales, where is most of the luminosity?  What regions
%    %    generate most of the energy as perceived from $>$100pc distance?  I suspect
%    %    it is the IRS1 region, which has no star formation and maybe no stars...
%    %    (probably best to do this with other data [SOFIA] in hand)
%    \item Make a lower-limit temperature map from the peak brightness
%        temperature of the full spectrum
%        [work begun in brightest\_molecule\_map, but not complete b/c need full data
%        on external HDs / orion to do this]
%
%\end{enumerate}

\section{Introduction}
High-mass stars are the drivers of galaxy evolution, cycling enriched materials
into the interstellar medium (ISM) and illuminating it.  During their formation
process, however, these stars are nearly undetectable because of their rarity
and their opaque surroundings.  We therefore know relatively little about how
massive stars acquire their mass and what their immediate surroundings look
like at this early time.  We expect, though, that the physical conditions
should be changing rapidly.

The stellar initial mass function (IMF) appears to be a universal distribution
\citep{Bastian2010a}.  However, massive  O-stars (with $M>50 \msun$)
almost always form in a clustered fashion \citep[in proto-clusters or
proto-associations;][]{de-Wit2004a,de-Wit2005a,Parker2007a}. %bressert2012? 
Their presence, and the strong feedback they produce, may directly influence
how the IMF around them is formed.  If feedback from these stars is relevant
while most of the  mass surrounding them is still in gas (not yet in stars),
the mass function in such clusters cannot be decided by ISM properties
(initial conditions) alone. 
% actually, this relies on stellar feedback \citep[e.g., as described
% in][]{Krumholz2011c}.

Models of high-mass star formation universally have difficulty collapsing enough
material to a stellar radius to form very massive stars.  Generally, these models
produce a high-mass star with enough luminosity to halt further
\emph{spherical} accretion at a very early stage, with $M_* \sim 10-20\msun$.
Radiation pressure provides a fundamental limit on how much mass can be
accreted \citep{Wolfire1987a,Osorio1999a}, but geometric effects can circumvent
this limit and allow further accretion
\citep{Yorke2002a,Krumholz2005b,Krumholz2009a,Krumholz2009b,Kuiper2012a,Kuiper2013c,Rosen2016a}.
Additionally, fragmentation-induced starvation can limit the amount of mass
available to the most massive star, instead breaking up massive cores into many
lower-mass fragments \citep{Peters2010a}.  These simulations still have limited
physics and can only produce stars up to $M\sim80$ \msun even in the current
best 3D cases  \citep{Kuiper2015a,Kuiper2016a}.  The question of how massive
stars acquire their mass, and especially whether they ever form Keplerian
disks, remains open \citep{Beltran2016b}.

Nature is clearly capable of producing massive stars larger than those produced
in simulations.  Within the LMC, stars up to $M\sim300$ \msun have been
spectroscopically identified \citep{Crowther2016a}.  Within our own Galaxy,
very massive stars have been found in compact, high-mass clusters such as NGC
3603 and the Arches \citep{Crowther2010a}.  While it is difficult to identify
and characterize the most massive stars in our own galaxy because the UV
features best capable of establishing their spectral types are 
extinguished,
it is still possible to find examples of very massive stars close to their
birth environments using infrared lines.  \citet{Barbosa2008a} identified an O3
and an O4 star ($M\gtrsim50$ \msun) within the W51 IRS2 region, demonstrating
that this region has at some time formed stars on the high
end tail of the IMF.  It remains to be seen whether W51 will form any very
massive stars ($M>100$ \msun), but it is  an appropriate environment to
investigate the process.

The W51 cloud contains two protocluster regions, IRS2 and e1/e2, which each
contain $M\gtrsim10^4$ \msun of gas and have large far-infrared luminosities
that
indicate the presence of embedded, recently-formed or forming massive stars
\citep{Harvey1986a,Sievers1991a,Ginsburg2012a,Ginsburg2016b}.  Previous
millimeter and centimeter observations have revealed the gas reservoir
that is forming new stars and, because of the high masses of the individual
cores detected, indicated that these new stars are likely to be massive
\citep{Zhang1997a,Eisner2002a,Zapata2009a,Tang2009a,
Zapata2010a,Shi2010b,Shi2010a,Koch2010a,Koch2012a,Koch2012b,Tang2012a,Goddi2016a}.  The
W51 protoclusters, while distant \citep[5.4 kpc;][]{Sato2010a}, therefore
provide a powerful laboratory for studying high-mass star formation in an
environment where feedback from massive stars is already evident, but 
formation is still ongoing.

The protocluster region within W51 exhibits many signs of strong feedback.  In
particular, there are many giant \hii regions detected in the infrared through
radio \citep{Mehringer1994a,Ginsburg2015a}.  These \hii region bubbles exist
on many scales, and the driving populations of OB stars have been identified
\citep{Kumar2004a,Ginsburg2016a}.  While the larger W51 cloud, which stretches
about 100 pc along Galactic longitude, shows some signs of interaction with a
supernova remnant \citep{Brogan2013a,Ginsburg2015a}, there is as yet no
sign that supernovae have occurred within the W51 IRS2 or e1/e2 protocluster
regions.  They are in the relatively short stage after which high mass stars
have formed but before the gas has been exhausted or expelled.

% we're proving this?
This combination of feedback and ongoing formation is essential for testing
components of high mass star formation theory that are relatively inaccessible
to simulations.  \citet{Krumholz2006a} suggests that accretion heating during
the formation of high-mass stars can heat massive cores to $\gtrsim100$ K and
therefore suppress fragmentation into smaller stars, which would be expected
for cold cores, though these models have $T>100$ K out to only $R\lesssim100$
AU.  While simulations have verified the conclusion that early-stage accretion
heating can control the mass scale within low-mass star
forming regions \citep{Krumholz2007c,Offner2011b, Bate2012a,Bate2014b}, there
have been neither theoretical nor observational tests of this model for
high-mass stars. 

\Figure{figures/rgb_overview_aplpy_withlabels.png}
{An overview of the W51A region as seen by ALMA and the VLA.  The main regions
discussed in this paper are labeled.  W51 e8 is a mm dust source, while W51 e1
is the neighboring \hii region.  Similarly, W51 IRS2 is the \hii region, and
W51 North is the brightest mm source in that area.  The colors are a composite
of millimeter emission lines: CO in blue, \methanol in orange, and HC$_3$N in
purple.  The 1.3mm continuum is shown in green.  The white hazy emission
shows VLA Ku-band free-free continuum emission
\citep{Ginsburg2016b}.}
{fig:overview}{1}{18cm}


 We present an
observational study of the high-mass star-forming region W51.  In Section
\ref{sec:observations}, we describe the observations and data reduction
process.  Section \ref{sec:results} describes the analysis:
We discuss source identification (\S \ref{sec:sourceid}),
%in Sections \ref{sec:W51e2e} and \ref{sec:w51e8andnorth} and then
the mass and flux recovered on different
spatial scales (\S \ref{sec:massbudget}),
the observed chemical distribution (\S \ref{sec:chemistrymaps}), and
temperatures inferred from \methanol lines (\S \ref{sec:methanol}).
%Section \ref{sec:nonionizingradiation} shows that the heating is not coming
%from \hii regions.
%Section \ref{sec:outflows} and its subsections describe outflows 
%finally!  so many subsections!!!
Section \ref{sec:discussion} discusses scales and types of feedback (\S
\ref{sec:feedbackscales}), outflows (\S \ref{sec:outflowdiscussion}),
accretion and outflows (\S \ref{sec:accretionandoutflows}),
and fragmentation (\S \ref{sec:fragmentation}).  
We conclude in Section \ref{sec:conclusion}.
Additional interesting features in the W51 data not directly relevant
to our main topic, the formation of high-mass stars,
are discussed in the Appendices, including
an interesting bubble (\S \ref{sec:e5bubble}), some remarkable outflows 
(\S \ref{sec:outflows}), and a characterization of the lower-mass sources
(\S \ref{sec:contsrcs}).

%including source
%identification (\S \ref{sec:sourceid}), spatial distribution (\S
%\ref{sec:corespatialdistribution}), and photometry (\S \ref{sec:photometry}).
%%, and statistical distributions (\S \ref{sec:distributionfunctions}).  
%We then examine the source temperatures (\S \ref{sec:temperature}) in order to
%determine their nature (\S \ref{sec:contsourcenature}).





\section{Observations}
\label{sec:observations}
As part of ALMA Cycle 2 program 2013.1.00308.S, we observed a
$\sim2\arcmin\times1\arcmin$ region centered between W51 IRS2 and W51 e1/e2
with a 37-pointing mosaic.  Two configurations of the 12m array were used,
achieving a resolution of 0.2\arcsec.  Additionally, a 12-pointing mosaic was
performed using the 7m array, hypothetically probing scales up to
$\sim28$\arcsec.  The full UV coverage included baselines over
the range $\sim12$ to $\sim1500$ m.
The spectral windows (SPWs) covered are listed in Table \ref{tab:spw},
and the lines they cover described in Section \ref{sec:obslines}.
%(figure \ref{fig:uvcov}).

% figure produced directly in CASA; not included in repository b/c
% it requries having access to the full data set
% \Figure{figures/visibility_weight_vs_uvdist_linear.png}
% {A weighted histogram of the visibility weights as a function of UV distance;
% this approximately shows the amount of data received at each baseline length.}
% {fig:uvcov}{1}{16cm}

% Additionally, we will comment briefly on project 2015.1.01569.S (PI: Goddi),
% which observed two fields centered on e2e/e8 and North with very long baselines
% (resolution 0.04\arcsec) and provides additional insights that our 0.2\arcsec
% data do not yield alone.  Full details of that data set will be released in a
% future work.

\begin{table*}[htp]
\caption{Spectral Setup}
\begin{tabular}{llll}
\label{tab:spw}
SpwID & Minimum Frequency & Maximum Frequency & Channel Width \\
 & $\mathrm{GHz}$ & $\mathrm{GHz}$ & $\mathrm{kHz}$ \\
\hline
0 & 218.11930228 & 218.619301 & -122.07 \\
1 & 218.36288652 & 220.355073 & -488.281 \\
2 & 230.376575 & 232.36876148 & 488.281 \\
3 & 232.981075 & 234.97326148 & 488.281 \\
\hline
\end{tabular}
\end{table*}


\subsection{Data Reduction}
Data reduction was performed using CASA 4.5.2-REL (r36115), including
reprocessing of datasets that were delivered with earlier versions.  The
QA2-produced visibility data products were combined using the standard inverse
variance weighting.  Two sets of images were produced for different aspects of
the analysis, one including the 7m array data and one including only 12m data.
Except where otherwise noted, the 12m-only data were used in order to focus on
the compact structures.  The conversion from
flux density to brightness temperature is $T_B \approx 220
\mathrm{\,K}/(\mathrm{Jy}\,\perbeam)$ for a 0.33\arcsec beam (most of the spectral
line data) or $T_B=590\mathrm{\,K}/(\mathrm{Jy}\,\perbeam)$ for a 0.2\arcsec
beam (for the higher-resolution images of the continuum) assuming a central
frequency 226.6 GHz (see below).

Full details of the data reduction, including all scripts used, can be found on
the project's github
repository\footnote{\url{https://github.com/adamginsburg/W51_ALMA_2013.1.00308.S}}.


\subsubsection{Continuum}
A continuum image combining all 4 spectral windows was produced using
\texttt{tclean}.  We identified line-rich channels from a spectrum of source e8
and flagged them out prior to imaging\footnote{The velocity
range of e8, e2, and North is similar enough that a common range was acceptable
for this process.  Note also that, while the sources are line-rich,
failure to flag out the data results in a $<10\%$ error in the continuum estimates
(see S{\'a}nchez-Monge et al, in prep, showing that even the richest
sources in the Galaxy have $<40\%$ line contribution).}.  We then phase
self-calibrated the data on baselines longer than 100m to increase the dynamic
range.  The final image
was cleaned to a threshold of 5 mJy.  The lowest noise
level in the image, away from bright sources, is $\sim0.2$ mJy/beam
($M\sim0.14$ \msun at $T=20$ K using the extrapolation of
\citet{Ossenkopf1994a} opacity from \citet{Aguirre2011a} with $\beta=1.75$),
but near the bright sources e2 and IRS2, the noise reached as high as $\sim2$
mJy/beam.  Deeper cleaning was attempted, but these attempts produced
instabilities that resulted in divergent maps.  The combined image has a central
frequency of about 226.6 GHz assuming a flat spectrum source; a steep-spectrum
source, with $\alpha=4$, would have a central frequency closer to 227 GHz, a
difference that is negligible for all further analysis.


\subsubsection{Lines}
\label{sec:obslines}
We produced spectral image cubes of the lines listed in Tables
\ref{tab:linesspw0}, \ref{tab:linesspw1}, \ref{tab:linesspw2}, and
\ref{tab:linesspw3}.  For kinematic and moment analysis, the median value over
the spectral range [25,30],[80,95] \kms was used to estimate and subtract the
local continuum. 





% \subsection{Continuum Morphological Analysis}
% \label{sec:morphology}
% The largest detected structures include the W51 Main HII region bubble and the
% W51 IRS2 HII region, which are relatively uninteresting since their properties
% have been previously well-characterized using radio (JVLA) data.  More exciting
% are the bright dusty structures, especially the ``tail'' pointing south of W51
% e8, which can be described as a 0.25 pc by 0.05 pc filament. This structure has
% a very high surface brightness along its ridge, exceeding 40 mJy/beam in our TE
% maps (23 K or 3.7\ee{4} MJy sr$^{-1}$).  This high brightness implies a high
% intrinsic temperature, $T>30$ K (Section \ref{sec:temperature}).
% 
% This narrow filament is most prominent in the continuum. It is evident
% in some lines (\formaldehyde, $^{13}$CS, \ceighteeno), but not others (SO, ).
% It is surrounded by molecular emission that is only slightly fainter...
% ...in SO it's pretty uniform brightness...

% \subsection{Simulations}
% {\bf Unfortunately CASA's simulations produce reproducible incorrectness, in
% that I cannot get an image that matches the input image.  There seems to
% *always* be some flux scaling no matter what input unit is used.  Therefore, I
% don't trust any of the results of CASA simulations yet.  Perseus looks OK, but
% Aquila is just flat out wrong, and the scale-recovery simulations I attempted
% also failed.
% 
% Followup on the above paragraph: I've gotten the simulations worked out; CASA
% always treats data as if they are in Jy/beam even if Jy/pixel units are
% specified when using the \texttt{sm.predict} module.  However, I'm not
% convinced of their utility at this stage.}
% 
% The enhanced noise around bright sources is unavoidable.  We tested the noise
% properties of our data set using the CASA \texttt{simobserve} toolkit.  We
% obtained a Herschel Gould's Belt Survey image of the Perseus molecular cloud at
% 250 \um \citep[resolution 18\arcsec][]{} and scaled it down by $\sim40\times$
% to match the resolution of our ALMA data.  We used the \texttt{sm.predict} task
% to ``observe'' the Herschel data with our exact UV data set.  We then used
% \texttt{sm.corrupt} to make the noise properties approximately match those of
% our observations.
% 
% We used the Perseus data scaled from 250 \um to 1.3 mm assuming
% a relatively shallow $\beta=1.5$ and a constant temperature $T=20$ K
% (since temperature maps are not available).
% For the Aquila data, we used the column density and temperature maps to
% derive a synthetic 1.3 mm map assuming an opacity $\kappa_{505 \mathrm{GHz}} =
% 4$ g \percc.  Since Aquila is at a greater distance, the Herschel resolution is
% coarser (0.9 \arcsec at 5.4 kpc) than our best resolution of 0.2\arcsec, so it
% is best compared to lower-resolution tapered data.
% 
% 
% When imaging the Perseus data set, we put NGC 1333 at the image center.  At the
% noise levels in our data, only the central portions of NGC 1333 are detected,
% with three point sources recovered.  In this map, the noise properties are
% very uniform.  We are therefore unable to analyze the NGC
% 1333 data with the exact same parameters as were used on W51.  However, using
% similar parameters (but with a lower significance threshold), we detect only
% three sources.
% 
% In a second experiment, we scaled the peak flux density of the Perseus map to
% be $\sim100\times$ brighter than it should be, making it comparable to the flux
% density of W51e2 in the real ALMA observations.  In this map, even with deep
% cleaning, the noise around the bright sources remains very high and artifacts
% are evident.



\section{Results \& Analysis}
\label{sec:results}
\subsection{Source Identification}
\label{sec:sourceid}
% dendrogramming.py
We used the \texttt{dendrogram} method described by \citet{Rosolowsky2008c} and
implemented in \texttt{astrodendro} to identify sources.  We used a minimum
value of 1 mJy/beam ($\sim5\sigma$) and a minimum $\Delta=0.4$ mJy/beam
($\sim2\sigma$) with minimum 10 pixels (each pixel is 0.05\arcsec).  This
cataloging yielded over 8000 candidate sources, of which the majority are noise
or artifacts around the brightest sources.  To filter out these bad sources,
we created a noise map taking the local RMS of the \texttt{tclean}-produced
residual map, using a weighted RMS over a $\sigma=30$ pixel (1.5\arcsec)
gaussian.  We then removed
all sources with peak S/N $< 8$, mean S/N per pixel $< 5$, or minimum S/N per
pixel $ < 1$.  We also only included the smallest sources in the dendrogram,
the ``leaves''.  These parameters were tuned by checking against ``real''
sources identified by eye and selected using \texttt{ds9}: most real sources are
recovered and few spurious sources ($<10$) are
included.  The resulting catalog includes 113 sources.

The `by-eye' core extraction approach, in which we placed ds9 regions on all
sources that look `real', produced a more reliable but less complete (and less
quantifiable) catalog containing 75 sources.  This catalog is more useful in
the regions around the bright sources e2 and North, since these regions are
affected by substantial uncleaned PSF sidelobe artifacts.  In particular, the
dendrogram catalog includes a number of sources around e2/e8 that, by eye,
appear to be parts of continuous extended emission rather than local peaks;
``streaking'' artifacts in the reduced data result in their identification
despite our threshold criteria.  The dendrogram extraction also identified
sources within the IRS 2 \hii region that are not dust sources.  Dendrogram
extraction missed a few clear sources in the low-noise regions away from
W51 Main and IRS 2 because the identification criteria were too conservative.

When extracting properties of the `by-eye' sources, we used variable sized
circular apertures, where the apertures were selected to include all of the
detectable symmetric emission around a central peak up to a maximum radius
$r\sim0.6$\arcsec.  This approach is necessary, as some of the sources are not
centrally peaked and are therefore likely to be spatially resolved starless
cores.

Further information about and general discussion of the continuum sources is in
Appendix \ref{sec:contsrcs}.  For the rest of this section, we focus on only
the few brightest sources.  The general point source population is briefly
revisited in Section \ref{sec:faintsrcs_discussion}.


%*demonstrate this* Generally, the obviously interstellar features are narrower.

%idea: concentration parameter, compare to power-law cores of varying indices.
%Protostar / core ratio on this basis?  Evolutionary indicator of subregions?





\subsubsection{W51e2e mass and temperature estimates from continuum}
\label{sec:W51e2e}

% analysis done in total_mass_analysis and (originally) dust_properties
In a $0.21\arcsec\times0.19\arcsec$ beam ($1100\times1000$ au), the peak flux
density toward W51 e2e is 0.38 Jy, which corresponds to a brightness
temperature $T_B=225$ K.  This is a lower limit to the surface brightness of
the millimeter core, since an optical depth $\tau<1$ or a filling factor of the
emission $ff<1$ would both imply higher intrinsic temperatures.  The implied
luminosity, assuming blackbody emission from a spherical beam-filling source,
is $L = 4\pi r^2 \sigma_{sb} T^4 = 2.3\ee{4} \lsun$, where
$\sigma_{sb}=5.670373\ee{-5} \mathrm{\,g\,s^{-3}\,K^{-4}}$ is the
Stefan-Boltzmann constant.  Since any systematic
uncertainties imply a higher temperature, this estimate is a lower limit on the
source luminosity.  Such a luminosity corresponds to a B0.5V, 15 \msun main
sequence star with effective temperature $4\ee{4}$ K \citep[][see
Section \ref{sec:stellarproperties} for further discussion of
stellar types]{Pecaut2013a}.
% http://www.pas.rochester.edu/~emamajek/EEM_dwarf_UBVIJHK_colors_Teff.txt

% this mass estimation method is not self-consistent; T_dust = T_peak implies
% tau >= 1
% If we assume that $T_{dust} = T_{peak}$ and that the dust is optically thin, we
% derive a dust mass $M_{dust}\sim20$ \msun. 

If we assume that the dust is optically thick throughout our beam, and assume
an opacity constant $\kappa(227 \mathrm{GHz})=0.0083$ cm$^2$ g$^{-1}$  (which
incorporates and assumed gas-to-dust ratio of 100), the minimum mass
per beam to achieve $\tau\geq1$ is $M=18$ \msun beam$^{-1}$.  This  mass is not
a strict limit in either direction: if the dust is indeed optically thick,
there may be substantial hidden or undetected gas, while if the filling factor
is lower than 1, the dust may be much hotter and therefore optically thin and
lower  mass.  However, simulations and models both predict that the dust will
become highly optically thick at radii $r\lesssim1000$ au
\citep{Forgan2016a,Klassen2016a}, so it is likely that this measurement
provides  a lower limit on the total gas mass surrounding the protostar.
Therefore, unless the stars are extremely efficient at removing material or the
gas fragments significantly on $<1000$ AU scales, the stellar mass is likely to
at least double before accretion halts.

For an independent measurement of the temperature that is not limited to the
optically thick regions, we use the \methanol lines in band, calculating an LTE
temperature that is $200 < T < 600$ K out to $r<2$\arcsec ($r<10^4$ au; Section
\ref{sec:methanol}).  As noted in Section \ref{sec:methanol}, these
temperatures may be overestimates when the low-J lines of \methanol are
optically thick, but for now they are the best measurements we have available.
If the dust temperature matches the methanol temperature, it would be optically
thin ($\tau \lesssim 1/3$) and the central source dust mass would be only
$\sim6$ \msun.  However, this latter estimate discounts any substructure at
scales $<1000$ AU, which we know exists from the 2015.1.01596.S data.

An upper limit on the radio continuum emission from W51e2e is $S_{14.5 GHz} <
0.6$ mJy/beam in a FWHM=$0.34\arcsec$ beam, or $T_{B,max} < 30$ K
\citep{Ginsburg2016b}.  Assuming emission from an optically thick \hii region
with $T_e = 8500$ K \citep{Ginsburg2015a}, the upper limit on the emitting
radius is $R(\hii) < 110 AU$.  Similar limits are obtained from other
frequencies in those data.  The free-free contribution to the millimeter
flux is therefore negligible, and the central source is unlikely to be
ionizing.  Limits on the stellar properties are further discussed in
Section \ref{sec:stellarproperties}.


\subsubsection{W51 e8 and North mass and temperature from continuum}
\label{sec:w51e8andnorth}
% total_mass_analysis
We repeat the above analysis for e8 and North.  They have peak intensities
of 0.35 and 0.44 Jy/beam respectively, corresponding to peak brightness
temperatures of 205 and 256 K.
The lower limit luminosities of W51 e8 and North in a single beam, assuming the
brightest detected beam is optically thick, are 1.6\ee{4} and 3.9\ee{4} \lsun,
respectively.


W51 North has a free-free upper limit similar to that of W51e2e, but somewhat less
restrictive because the noise in that region is substantially higher.  W51 e8,
by contrast with the others, has a clear detection at cm wavelengths.  The
source e8n, which is offset from the peak mm emission by 0.13\arcsec\ (700 AU),
has $S_{25 GHz}=4.7$ mJy/beam, corresponding to $T_B=135$ K, which implies an
\hii region size $R=180$ AU if the emission is produced by  optically thick
free-free emission.  This could be part of an ionized jet or an ionizing binary
companion, but its offset from the central mm source
suggests that it is not a simple spherically symmetric HC\hii region.

The apparent dust masses in the central beams of e8 and North are the same
as in e2e, $M\sim18$ \msun, but these measurements are subject to the same
limits discussed in Section \ref{sec:W51e2e}.

\subsubsection{W51 d2: the little hot core that couldn't}
\label{sec:w51d2}
The source W51 d2 is something of an outlier in our sample.  Like the three
main hot cores, e2e, e8, and North, d2 has a small extended molecular hot core
around it, with $R\lesssim3000$ AU.  However, unlike these cores, d2 is a very
bright centimeter continuum source, $\sim17$ mJy at 15 GHz
\citep{Ginsburg2016a}.  Its millimeter continuum emission can readily be
explained as free-free emission, requiring a spectral index of only
$\alpha\sim0.6-0.7$ from the cm to account for all of its mm emission.  There
is little doubt that it contains a compact \hii region.  Because of this free-free
contamination, we cannot estimate the central core's dust mass.  If we assume
the free-free is optically thin at 36 GHz \citep[the highest-frequency cm-wave
measurement
we have available][]{Goddi2015b}, with $S_{36 \mathrm{GHz}} = 29$ mJy and
$S_{227 \mathrm{GHz}}=110$
mJy, the dust-produced flux would be $S_{227 \mathrm{GHz}} = 86$ mJy, or about
$\sim20-25\%$ as bright as the other three cores ($T_{B}=65$ K).  With such a modest
lower-limit brightness temperature, the dust source is likely to be optically
thin or less than beam-filling, making its upper limit dust mass much less than
$M\ll18\msun$; if we assume $T_{dust} = T_{line,max} = 220 K$, the upper limit
dust mass is $M<7$ \msun.  If d2 were a purely dust source, its lower limit
luminosity is a meager 160 \lsun, which is belied by its evidently ionizing
photosphere.

Additionally, unlike the three hot cores, d2 does not drive an outflow.  It
does, however, power a unique set of ammonia (\ammonia) masers \citep[][Wootten
\& Wilson in prep]{Gaume1993a,Wilson1990a,Henkel2013a}.  These features imply
it is in an intermediate evolutionary state between the larger compact \hii
regions and the hot cores that exhibit no centimeter continuum.



\subsection{The mass and light budget on different spatial scales}
\label{sec:massbudget}
An evolutionary indicator used for star-forming regions is the amount of mass
at a given density; a more evolved (or more efficiently star-forming) region will
have more mass at high densities.  We cannot measure the dense gas fraction
directly, but the amount of flux density recovered by an interferometer
provides an approximation.

% total_mass_analysis
For the ``total'' flux density in the region, we use the Bolocam Galactic Plane
Survey observations \citep{Aguirre2011a,Ginsburg2013a}, which are the closest
in frequency single-dish millimeter data available.  We assume a spectral index
$\alpha=3.5$ to convert the BGPS flux density measurements at 271.4 GHz to the
mean ALMA frequency of 226.6 GHz.  The ALMA data (specifically, the
0.2\arcsec resolution 12m-only data) have a total flux 23.2 Jy above a  conservative
threshold of 10 mJy/beam in our
mosaic; in the same area the BGPS data have a flux of 144 Jy, which scales down to
76.5 Jy.  The recovery fraction is 30$\pm3$\%, where the error bar accounts
for a change in $\alpha\pm0.5$.  The threshold of 10 mJy/beam corresponds to a
column threshold $N>1.3\ee{25}$ \percc for 20 K dust. This threshold also
corresponds to an optical depth of $\tau\approx0.5$, implying that a substantial
fraction of the cloud is either approaching optically thick or is warmer than 20
K.  For an unresolved spherical source in the $\sim0.2\arcsec$ beam, this
column density corresponds to a volume density $n>10^{8.1}$ \percc.
Of the area with significant emission, 23\% has $T_B>20$ K (34 mJy \perbeam)
and must have $T_{dust}>20$ K, guaranteeing that a substantial fraction of all
of the detected continuum emission is coming from warmer dust.


% total_mass_analysis also (end of it)
Even more impressive is the amount of the total flux density concentrated
into the three `massive cores', W51 e2e, e8, and North.  These three contain
12.3 Jy (within 1\arcsec or 5400 AU apertures) of the total 23.2 Jy in the
observed field - more than half of the total ALMA flux density, or 15\% of the
BGPS flux density.  In a \citet{Kroupa2001a} IMF, massive stars ($M>20$ \msun)
account for only 0.15\% of the mass, so in order for the gas mass distribution
to produce a `normal' stellar distribution, the high-mass-star-producing gas
must be much brighter (hotter) than that making low-mass stars, or the gas 
in these cores must be substantially redistributed and fragmented into a
mixture of high- and low-mass stars as the region evolves.
% In [47]: imf.kroupa.integrate(20, 1000)
% Out[47]: (0.001422768266487118, 1.1342325281121882e-11)

% \todo{TODO: determine the largest angular scale in the ALMA images.  Requires
% using the simulations.}

\subsection{Chemically Distinct Regions}
\label{sec:chemistrymaps}
%\subsubsection{Observations}
\label{sec:chemistrymapsobs}
The large ``hot cores'' in W51 (e2, e8, and North) are spatially well-resolved
and multi-layered.  These cores are detected in lines of many different species
spanning areas $\sim5\ee{3}-1\ee{4}$ AU across.  We describe some of the
specific notable chemical features in this section, but the overall point that
the three biggest hot cores have extended chemical structure is highlighted in
Figures \ref{fig:chemmapse2}, \ref{fig:chemmapse8}, and
\ref{fig:chemmapsnorth}, with a fainter hot core shown for contrast in Figure
\ref{fig:chemmapsALMAmm14}.

Surrounding W51e2e, there are relatively sharp-edged and uniform-brightness
regions in a few spectral lines over the range 51-60 \kms (Figure
\ref{fig:chemmapse2}, especially the \methanol and \methylformate lines).  Some
of these features are elongated in the direction
of the outflow, but most have significant extent orthogonal to the outflow.
The circularly symmetric features are prominent in \methanol, OCS, and
\dimethylether, weak but present in \formaldehyde and SO, and absent in
\cyanoacetylene and HNCO.

Around e8, a similar chemically enhanced region is observed, but in this case
\dimethylether is absent.  Toward W51 North, \methanol, \formaldehyde, and SO
exhibit the sharp-edged enhancement feature, while the other species do not.
%The enhancement is from 50-60 \kms.

By contrast, along the south end of the e8 filament, no such enhanced features
are seen; only \formaldehyde and the lowest transition of methanol, \methanol
\fourtwotwo, are evident.

The relative chemical structures of e2, e8, and North are  similar.
The same species are detected in all of the central cores.  However, in e2,
\dimethylether, \methylformate, \ethylcyanide, and Acetone (\acetone) are
significantly more extended than in the other sources.
\gaucheethanol is detected in W51 North, but is weak in e8 and almost absent
in e2 (Figures \ref{fig:chemmapse2}, \ref{fig:chemmapse8},
\ref{fig:chemmapsnorth}, \ref{fig:chemmapsALMAmm14}).

Different chemical groups exhibit different morphologies around e2, and this
approximate grouping is also seen around the other cores.  Species that are
elongated in the NW/SE direction are associated primarily with the outflow
(\cyanoacetylene, \ethylcyanide).  Other species are associated primarily with
the extended circular core (\methylformate, \dimethylether, \acetone).  Some
are only seen in the compact core ($R<0.4\arcsec\sim2000$ AU;
\methyleneamidogen, HNCO, \formamide, and vibrationally excited
\cyanoacetylene).  Only \methanol and OCS are associated with both the extended
core and the outflow, but not the greater extended emission.  \ketene seems to
be associated with only the extended core, but not the compact core. Finally,
there are the species that trace the broader ISM in addition to the cores and
outflows: \formaldehyde, $^{13}$CS, OCS, C$^{18}$O and SO.  Both HCOOH and
N$_2$D$^+$ are weak and associated only with the innermost e2e core.

The presence of these complex species symmetrically distributed at large
distances ($r\sim5000$ AU) from the central sources is an independent
indication of the gas heating provided by these sources.  The abundance
increase most likely corresponds to $T\gtrsim85$ K, the approximate
sublimation temperature of \methanol ice \citep{Green2009a}.

While we focused on the three main hot cores, which all have radii $\sim5000$
AU, there are a few others that have similar chemical enhancements, but
significantly smaller extents.  The sources d2 and ALMAmm31 can be seen in
Figure \ref{fig:chemmapsnorth} on the right (west) side of the map.  These both
have resolved chemical structure, but the structures are smaller than in the
main hot cores.  d2 is also unique in having a central ionizing source detected
in H30$\alpha$ and a (moderately) extended chemical envelope.


\Figure{figures/chemical_max_slabs_e2.png}
{Peak brightness maps of the e2 region in 47 different lines over the range 51 to 60
\kms.
The cutouts are $6\arcsec\times6\arcsec$ ($3.2\ee{4}\times3.2\ee{4}$ AU).
To illustrate the lower limit temperature implied by the observed brightness,
the maps are not continuum-subtracted.  
For additional contrast, contours are shown at 150, 200, 250, and 300 K
(red, green, blue, yellow).
There
is a strong `halo' of emission seen in the CH$_3$Ox lines and OCS.  Extended
emission is also clearly seen in SO, $^{13}$CS, and \formaldehyde, though these
lines more smoothly blend into their surroundings.  HNCO and \formamide have
smaller but substantial regions of enhancement with a sharp contrast to their
surroundings.  HC$_3$N traces the e2e outflow.  The bright H30$\alpha$ emission
marks the position of e2w, the hypercompact HII region that dominates the
centimeter emission in e2.
}{fig:chemmapse2}{1}{18cm}

\Figure{figures/chemical_max_slabs_e8.png}
{Peak brightness maps of the e8 region in 47 different lines over the range 52 to 63
\kms.
The cutouts are $6\arcsec\times6\arcsec$ ($3.2\ee{4}\times3.2\ee{4}$ AU).
To illustrate the lower limit temperature implied by the observed brightness,
the maps are not continuum-subtracted.  
For additional contrast, contours are shown at 150 and 200 K
(red and green, respectively).
As in e2 (Figure \ref{fig:chemmapse2}),
there is extended emission in the CH$_3$OH and OCS lines, but in contrast with e2,
the other CH$_3$Ox lines are more compact. SO is brighter than OCS in e8,
whereas the opposite is true in e2.
}{fig:chemmapse8}{1}{18cm}

\Figure{figures/chemical_max_slabs_north.png}
{Peak brightness maps of the W51 IRS2 region containing the North core in 47
different lines over the range 54 to 64 \kms.  The cutouts
are $10\arcsec\times10\arcsec$ ($5.4\ee{4}\times5.4\ee{4}$ AU).  To illustrate
the lower limit temperature implied by the observed brightness, the maps are
not continuum-subtracted.  For additional contrast, contours are shown at 150
and 200 K (red and green, respectively).  Qualitatively, the relative extents
of species seem comparable to e8 (Figure \ref{fig:chemmapse8}).  The
H30$\alpha$ point source is W51 d2.
}{fig:chemmapsnorth}{1}{18cm}

\Figure{figures/chemical_max_slabs_ALMAmm14.png}
{Peak brightness maps of the ALMAmm14 region in 47 different lines over the
range 58 to 67 \kms.
The cutouts are $5\arcsec\times5\arcsec$ ($2.7\ee{4}\times2.7\ee{4}$ AU).
ALMAmm14 is one of the brightest sources outside of
e2/e8/IRS2, but it is substantially fainter than those regions.  Still, it has
a notably rich chemistry.
}{fig:chemmapsALMAmm14}{1}{18cm}


\subsection{\methanol temperatures \& columns in the hot cores}
\label{sec:ch3ohtem}
\label{sec:methanol}
The  chemically enhanced regions appear to be associated with regions of
elevated gas temperature.  We examine the temperature structure directly by
analyzing the excitation of lines for which we have detected multiple
transitions with significant energy differences.  We do not use \formaldehyde
for this analysis despite its usefulness as a thermometer because it is clearly
optically thick (self-absorbed) in all lines in the hot cores.
This section presents the details of the temperature determination, while 
the implications of the temperature measurements will be discussed later,
throughout Section \ref{sec:discussion}.

We produce rotational diagrams for each spatial pixel covering all \methanol
lines detected at high significance toward at least one position\footnote{We
observe both A- and E-type \methanol, but assume the ratio $E/A=1$, 
as expected if the molecules have an even moderately high formation 
temperature $T\gtrsim20$ K \citep{Wirstrom2011a}.}.  The detected
lines span a range $45 < E_U < 800$ K, allowing robust measurements of the
temperature
assuming the lines are optically thin, in LTE, and the gas temperature is high
enough to excite the lines.  These conditions are likely to be satisfied in the
e2e, e8, and North cores, except for the optically thin requirement; the lower-J
lines in particular are optically thick across much of the extent of the cores. 
% Luckily,
% there are some lines in band that have much lower Einstein $A_{i,j}$ values 
% (i.e., \methanol $5_{-4,2} - 6_{-4,3}$ has $A\approx10^{-5.19}$, while
% \methanol $4_{2,2}-3_{1,2}$ has $A\approx10^{-4.33}$)
% but comparable upper-state energy levels, allowing us to probe higher column
% densities than would otherwise be possible.

% ch3oh_rotational_diagram_maps
\FigureTwo
{figures/ch3oh_temperature_map_e2.png}
{figures/ch3oh_column_map_e2.png}
{Methanol temperature and column density maps around e2. 
The maps are $5\arcsec\times5\arcsec$ ($2.7\ee{4}\times2.7\ee{4}$ AU).
The central regions around
the cores appear to have lower column densities because the lines become
optically thick and self-absorbed.  The contour in the temperature map is at
350 K, where red meat is typically considered ``well-done''.}
{fig:ch3ohe2}{1}{10cm}

Sample fitted rotational diagrams are displayed in Figure \ref{fig:ch3ohe2rot}.
The line intensities are computed from moment maps integrating over the range
(51, 60) \kms in continuum-subtracted spectral cubes, where the continuum
was estimated as the median over the ranges (25-35,85-95) \kms, except
for the J=25 lines, which had a continuum estimated from the 10th percentile
over the same range to exclude contamination from the SO outflow line wings.
The fitted species are listed in the order plotted in Table \ref{tab:methanol}.
Note that A- and E-type methanol are effectively independent chemical species
because they can only interchange in chemical reactions \citep{Rabli2010c},
but for the purpose of our coarse measurements, they appear to have similar
abundances.

\Figure{figures/ch3oh_rotation_diagrams_e2.png}
{A sampling of fitted rotation diagrams of the detected \methanol transitions.
These are shown to provide validation of the temperatures and column densities
derived and shown in Figure \ref{fig:ch3ohe2}.  The lower-left corner of each
panel shows the position from which the data were extracted in that figure in
units of figure fraction.  Errorbars show the measurement error on each point;
because these are plotted on a log scale, the errors are often smaller than
the plotted points.  The fitted temperature and
column are shown in the top right of each plot.
The central position is severely affected by absorption and can be ignored.
The corners do not have enough line detections to be fit.
}{fig:ch3ohe2rot}{1}{10cm}

\begin{table*}[htp]
\caption{\methanol lines used to determine temperature}
\begin{tabular}{lll}
\label{tab:methanol}
Line Name & Frequency & E$_U$ \\
 & $\mathrm{GHz}$ & $\mathrm{K}$ \\
\hline
E-CH$_3$OH $4_{2,2}-3_{1,2}$ & 218.44005 & 45.45988 \\
A-CH$_3$OH $4_{2,3}-5_{1,4}$ & 234.68345 & 60.9235 \\
E-CH$_3$OH $8_{0,8}-7_{1,6}$ & 220.07849 & 96.61336 \\
E-CH$_3$OH $5_{-4,2}-6_{-3,4}$ & 234.69847 & 122.72222 \\
A-CH$_3$OH $10_{2,9}-9_{3,6}$ & 231.28115 & 165.34719 \\
A-CH$_3$OH $18_{3,15}-17_{4,14}$ & 233.7958 & 446.58025 \\
E-CH$_3$OH $23_{5,19}-22_{6,17}$ & 219.99394 & 775.89371 \\
E-CH$_3$OH $25_{3,22}-24_{4,20}$ & 219.98399 & 802.17378 \\
\hline
\end{tabular}

\end{table*}


To validate some of the rotational diagram fits, we examined the modeled
spectra overlaid on the real (Figure \ref{fig:ch3ohe2epeaks}).  These generally
display significant discrepancies, especially at low J where self-absorption is
evident.  In Figure \ref{fig:ch3ohe2epeaks}, there is clearly a low-temperature
component slightly redshifted from the high-J peak that can be seen as a dip
within the line profile.  The presence of this unmodeled low-temperature
component renders our \methanol temperature measurements uncertain, biasing
them to be slightly high.  Nevertheless,
the general trend exhibited by \methanol temperatures matches expectations
if there is a central heating source.



\FigureTwo
{figures/ch3oh_rotdiagram_fits_SelectedPixel1.png}
{figures/ch3oh_rotdiagram_fits_SelectedPixel2.png}
{Spectra of the \methanol lines toward a pair of selected pixels just outside
of the e2e core. (a) is 0.55\arcsec  and (b) is 1.33\arcsec from e2e.  The red
curves show the LTE model fitted from a rotational diagram as shown in Figure
\ref{fig:ch3ohe2rot}.  The model is not a fit to the data shown, but is instead
a single-component LTE model fit to the integrated intensity of the lines
shown.  As such, the fit is not convincing, and it is evident that a
single-temperature, single-velocity model does not explain the observed lines.
Nonetheless, a component with the modeled temperature is likely to be present
in addition to a cooler component responsible for the self-absorption in the
low-J lines.  (a) shows a pixel close to the center of e2e, which is probably
optically thick in most of the shown transitions, while (b) shows a better case
where the highest-$A_{ij}$ (highest critical density) lines are overpredicted
but many of the others are well-fit.}
{fig:ch3ohe2epeaks}{1}{8cm}

Figure \ref{fig:ch3ohvscont} shows a comparison between the \methanol
$10_{2,9}-9_{3,6}$ line and the 225 GHz continuum.  While the brightest regions
in \methanol mostly have corresponding dust emission, the dust morphology
traces the \methanol morphology very poorly.  This difference suggests that the
enhanced brightness is not simply because of higher total column density.
We examine the dust-\methanol correspondence more quantitatively in Figure
\ref{fig:ch3ohtemX}; Figure \ref{fig:ch3ohtemX}d shows the poor correlation.

\FigureTwo{figures/e2_continuum_contours_on_ch3oh1029.png}
          {figures/e2_ch3oh1029_contours_on_continuum.png}
{Images showing \methanol $10_{2,9}-9_{3,6}$ and 225 GHz continuum emission,
with \methanol in grayscale and continuum in contours (left) and continuum in
grayscale, \methanol in contours (right).  The fainter (whiter) regions in the center
of the \methanol map correspond to the bright continuum cores and show where all lines
appear to be self-absorbed.}
{fig:ch3ohvscont}{1}{8cm}

Figure \ref{fig:methanolradialprofile} shows the observed brightness profiles
of \methanol line and dust continuum emission, which gives a lower limit on
the physical temperature probed by the \methanol and continuum.
Figure \ref{fig:ch3ohtemX}a shows a comparison of the \methanol temperature and
abundance.  The \methanol abundance is derived by comparing the rotational
diagram (RTD) fitted \methanol column density to the dust column density while
using the \methanol-derived temperature as the assumed dust temperature.  The
figure shows all pixels within a 3\arcsec (16200 AU) radius of e2e, with pixels
having low column density and high temperature (i.e., pixels with bad fits) and
those near e2w (which may be heated by a different source) excluded.  We used
moment-0 (integrated intensity) maps of the \methanol lines to perform these
RTD fits, which means we have ignored the line profile entirely and in some
cases underestimated the intensity of the optically thick lower-J lines: in the
regions of highest column, the column is  underestimated and the temperature is
overestimated, as can be seen in Figure \ref{fig:ch3ohe2epeaks}.


\Figure{figures/radialprofile_max_CH3OH_e2.png}
{Radial profiles of the azimuthally-averaged peak surface brightness of the
observed \methanol transitions along with the profile of the continuum
brightness.  These profiles indicate \emph{lower limit} gas temperatures
as a function of radius; the true temperature can be substantially higher even
if the lines are optically thick because of foreground, cold, self-absorbing
layers.
The radial profiles were constructed from images with 0.2\arcsec
resolution including only 12m data.  The lines are not continuum subtracted, so
they represent the true on-sky observed brightness.  The abundance bump is
evident at $r\sim1.5\arcsec$, while the consistently increasing high-J lines
(\methanol $23_{5,19}-22_{6,17}$ and $25_{3,22}-24_{4,20}$) demonstrate that
the excitation is continuing to increase toward the center, even after the
lower-J lines become optically thick.
%The central dip shows where the lines go into absorption, though they are only
%seen in absorption at $\sim55$ \kms.  The \methanol lines are
%continuum-subtracted.}
}
{fig:methanolradialprofile}{1}{12cm}

% plot_code/overlay_contours_on_ch3oh.py
\FigureFour
{figures/e2_CH3OH_LTE_temperature_vs_abundance.png}
{figures/e2_CH3OH_LTE_temperature_radial_profile.png}
{figures/e2_CH3OH_LTE_abundance_radial_profile.png}
{figures/e2_CH3OH_LTE_vs_dust_column.png}
{Comparison of the \methanol temperature, column density, and abundance.
(a) The relation between temperature and abundance.  There is a weak correlation,
but most of the high abundance regions are at high temperatures.
(b) Temperature vs distance from e2e.  There is a clear trend toward higher
temperatures closer  to the central source
(c) Abundance vs distance from e2e.  The apparent dip at $r<1\arcsec$ is
somewhat artificial, as it is driven by a rising dust emissivity that
corresponds to an increasing optical depth in the dust.  The \methanol column
in this inner region is likely to be underestimated. 
(d) \methanol vs dust column density.  }
{fig:ch3ohtemX}

A few features illustrate the effects of thermal radiative feedback on the gas.
The temperature jump starting inwards of  $r\sim1.5\arcsec$ (8100 AU; Figure
\ref{fig:ch3ohtemX}b) is
substantial, though the 100-200 K floor at greater radii is likely
artificial\footnote{The low-J transitions have significant optical depth
across the whole region, but in the inner part of the core, the temperature
measurement is dominated by the high-J transitions, which give a long
energy baseline for the fit.  In the core exterior, the high-J lines are
not detected, so the (possibly optically thick) low-J lines determine
the temperature fit, which results in much lower accuracy and greater
bias.}.
There is an abundance enhancement at the inner radii, but in the plot it
appears to be a radial bump rather than a pure increase.  The abundance
enhancement is probably real,
and is approximately a factor of $\sim5-10$.  The inner abundance dip
is caused by two coincident effects: first, the \methanol column becomes underestimated
because the low-J \methanol is \emph{self}-absorbed, and second, the dust
becomes optically thick, blocking additional \methanol emission, though this
latter effect is somewhat self-regulating since it also decreases the inferred
dust column (the denominator in the abundance expression).


% presently, this section adds nothing to the paper-in-progress...
% \subsection{Temperatures derived from \formaldehyde}
% The original goal of this project was to measure the gas temperatures in the
% moderate-density ($n\sim10^4-10^5$ \percc) gas that may correspond to
% pre-stellar material.  We performed the same analysis as was done in \citet{Ginsburg2016a}
% to create a \formaldehyde temperature map, but found very high temperatures even
% in the regions expected to be cold, with a temperature floor around 50 K.
% 
% There are a few possible explanations for this high thermal floor.  First is
% purely observational: the maps do not include any zero-spacing information and
% may therefore have resolved out some emission.  It is possible that the
% \threeohthree line is preferentially filtered, as it should be the brightest
% and most widespread of the triplet.  This possibility can be examined when
% zero-spacing data become available.
% 
% Second, it is possible that a large fraction of the area of the cloud is
% optically thick in at least the \formaldehyde \threeohthree transition.  Such a
% high optical depth is not expected since the observed brightness temperatures
% typically peak at $\sim1$ K and max out at $\lesssim15$ K outside of the
% central protoclusters.  Such a low brightness temperature for optically thick
% gas would imply that the molecules are subthermally excited but highly
% abundant.
% 
% Third, the temperatures could be genuinely high.  A Galactic molecular cloud
% should not be in thermal equilibrium at 50 K, but should readily cool to 10-20
% K, so such a cloud would have to be subject to extraordinary heating
% conditions.  Indeed, W51 is subject to some fairly extreme conditions, with one
% of the Galaxy's most luminous HII regions encompassing most of the molecular
% material.  Even with such heating, though, the densest gas should be able to
% cool well below 50 K.  The UV photons ionizing the HII region cannot penetrate
% these densest regions and the infrared radiation field is not strong enough to
% maintain such a high floor (is it?).  Curiously, the CO 3-2 maps of
% \citet{Parsons2012a} exhibit peak brightness temperatures up to 50 K within the
% mapped region, so it is plausible that the moderate-density medium is much
% warmer than in a typical cloud.
% 
% To evaluate this third option, current data are insufficient.  Ruling out option (1),
% filtering, would be helpful but not definitive.  Additional observations of J=2 and
% J=4 \formaldehyde transitions or the $H_2$$^{13}$CO J=3 lines would be enough to rule
% out hypotheses (1) and (2).
% 
% \Figure{figures/W51e2_full_h2co_aplpy.pdf}
% {An RGB composite image of the peak intensity of the three \formaldehyde lines.
% The red channel, \threeohthree, has an upper-state energy level 23 K, so redder
% regions are qualitatively cooler than whiter or bluer regions.  However, the solid
% white zones around W51 e2, e8, and north are areas where all three lines become
% very optically thick, so the color no longer implies anything about the temperature.}
% {fig:h2corgb}{1}{12cm}

\subsection{Radial mass profiles around the most massive cores}
\label{sec:radialmass}
In Figure \ref{fig:hmradprof}, we show the radial mass profiles extracted from the
three high-mass protostellar cores in W51: W51 North, W51 e2e, and W51 e8.
The plot shows the enclosed mass out to $\sim1\arcsec$ (5400 AU).  On larger
spatial scales, the enclosed mass rises more shallowly, indicating the end of the
core.
% too hard to measure this (though our data are capable of recovering spatial
% scales up to XXXXX).

All three sources show similar radial profiles, with each containing up to 3000
\msun within a compact radius of 5400 AU (0.03 pc).  However, the temperature
structure within these sources is certainly not homogeneous, and  likely a
large fraction of the total flux comes from $T\gtrsim300$ K heated material
\citep[Section \ref{sec:ch3ohtem}; ][]{Goddi2016a}.  If the observed dust were
all at 600 K instead of 40 K, the mass would be $17\times$ lower, $\sim100-200$
\msun, which we treat as a strict lower bound as it is unlikely that the dust
at more than $r\gtrsim1000$ AU is so warm.  Additionally, it is likely that a
substantial mass of cold dust is also present but undetectable because it is
hidden by the hotter dust.  Figure \ref{fig:hmradprof}b shows $M(R)$ using
$T_{dust}=T_{\scriptsize{\methanol}}$, which is a more reasonable guess at the
true mass profile (though it is still a lower limit; see \S
\ref{sec:ch3ohtem}).

\FigureTwo
{figures/cumulative_radial_flux_massivecores.png}
{figures/cumulative_radial_mass_of_TCH3OH_massivecores.png}
{The cumulative (a) flux density radial profile and (b) mass radial profile
centered on three massive protostellar cores.  The cores share similar profiles and
are likely dominated by hot dust in their innermost regions, but they are more
likely to be dominated by cooler dust in their outer, more massive regions.
The cumulative mass distribution inferred from assuming the gas is at a
constant temperature in (a) may therefore be deceptive.    In (b), we use the
temperature map computed from \methanol in Section \ref{sec:methanol}; this
plot should be at least qualitatively more realistic, though it is subject to
many uncertainties discussion in \S \ref{sec:methanol}.}
{fig:hmradprof}{1}{8cm}


% \Figure{figures/radialprofileexponent_of_TCH3OH_massivecores.png}
% {
% The radial mass profile exponent $\kappa_\rho$ as a function of radius.  The
% `fiducial value' used in the \citet{McKee2003a} model is $\kappa_\rho=1.5$, but
% importantly it is \emph{constant} at $t=0$.  A non-constant $\kappa_\rho$
% indicates that collapse is not self-similar, and the flattening toward smaller
% radii suggests that it is slower than freefall collapse.}
% {fig:kapparho}{1}{12cm}



\subsection{Ionizing vs non-ionizing radiation}
\label{sec:nonionizingradiation}
The formed and forming protostars are producing a total $\gtrsim10^7$ \lsun of
far infrared illumination \citep{Ginsburg2016b}.  This radiation heats the
cloud's molecular gas, affecting the initial conditions of future star
formation.

The ionizing radiation in W51 was discussed in detail in \citet{Ginsburg2016b}.
Ionizing radiation affects much of the cloud volume, but little of the
high-density prestellar material:  there is no evidence of increased molecular gas
temperatures in the vicinity of \hii regions.  While in Section
\ref{sec:chemistrymaps} we identify chemically enhanced regions as those where
radiative feedback has heated the dust and released ices into the gas phase, no
such regions are observed surrounding the compact \hii regions.

% This lack of molecular brightness enhancement is only an indirect indication
% that the \hii regions do not affect the surrounding dense gas temperature.  A
% direct proof that the ionizing sources are having no or minimal effect would
% require temperature measurements on the outskirts of the \hii regions, but this
% is not possible without prior detection of bright emission in thermometric
% transitions.

The chemical maps shown in Section \ref{sec:chemistrymaps} show the volumes of
gas clearly affected by newly-forming high-luminosity stars.  The
\methanol-enhanced region around W51e2e extends 0.04 pc, or 8500 AU (see Section
\ref{sec:ch3ohtem}). Other locally enhanced species, especially the nitrogenic
molecules HNCO and \formamide, occupy a smaller and more asymmetric region
around e2e and e2w (Figure \ref{fig:e2methanolhnco}).  These chemically enhanced
regions are most prominent around the weakest radio sources or regions with
no radio detection; they are most likely heated by direct infrared radiation
from these sources.

The luminosities of the other \uchii and \hchii regions throughout the observed
area are high enough, $L\gtrsim10^4$ \lsun, to produce chemically-enhanced
molecular envelopes if they were surrounded by dense ($n(\hh)\gtrsim10^4$
\percc) molecular gas.  Since few such regions are detected, we conclude that
these \hii regions are not surrounded by such high-density gas but instead are
traveling through a lower-density medium.

There are two counterexamples, e2w and d2, which are extremely compact \hchii regions
that  exhibit some enhanced molecular emission around them, though with a smaller
radial extent than the hot cores.  For e2w, it is difficult to estimate the extent
of the enhanced region, since e2w is embedded in a common core with e2e, but we can
set an upper limit of $1\arcsec\approx5400$ AU.  Around d2, the extent is
$0.6\arcsec\approx3000AU$.  Both of these objects likely turned on their ionizing
radiation (contracted onto the main sequence) only recently.  The enhanced
molecular emission is either from the remnant core that was heated during the
star's pre-ionizing phase, or it is presently being heated with photons that
have been absorbed and re-emitted as non-ionizing radiation.
% What is the cooling timescale?


\FigureTwo
{figures/W51e2_ch3oh_hnco_continuum_aplpy_kucontours.pdf}
{figures/W51e8_ch3oh_hnco_continuum_aplpy_kucontours.pdf}
{Image of \methanol $8_{0,8}-7_{1,6}$ (red), HCNO $10_{0,10}-9_{0,9}$ (green), and 225 GHz
continuum (blue) toward (a) W51e2 (b) W51e8.  The contours show Ku-band radio continuum
emission tracing the \hii regions (a) W51 e2w and (b) W51e1, e3, e4, e9, and
e10.  The \methanol emission is relatively symmetric around the high-mass
protostar W51 e2e and the weak radio source W51 e8, suggesting that these
forming stars are responsible for heating their surroundings.  By contrast, the
\hii regions do not exhibit any local molecular brightness enhancements (except
e8), indicating that the \hii regions are not heating their local dense
molecular gas.}
{fig:e2methanolhnco}{1}{8.5cm}
 

\Figure
{figures/W51north_ch3oh_hnco_continuum_aplpy_kucontours.pdf}
{Image of \methanol $8_{0,8}-7_{1,6}$ (red), HCNO $10_{0,10}-9_{0,9}$ (green),
and 225 GHz continuum (blue) toward  North, as in Figure
\ref{fig:e2methanolhnco}.  The contours show Ku-band radio continuum emission
tracing the diffuse IRS 2 \hii region.}
{fig:northmethanolhnco}{1}{8.5cm}

\subsection{Outflows}
\label{sec:mainoutflows}
While many outflows were detected, we defer their discussion to Appendix
\ref{sec:outflows}, as the details of these flows is not relevant to the main
point of the paper.  However, we note that out of the dozen or so outflows
detected, \emph{none} come from radio continuum sources (\hii regions).  All
outflows that have a clear origin come from millimeter-detected,
centimeter-faint sources, suggesting that these sources are accreting molecular
material and are not emitting ionizing radiation.

% \resolved{What direct tests can be used to show that \hii regions aren't heating
% their surroundings?  \formaldehyde is good, in principle, but maybe not in
% practice because of the possible optical depth issues.  Radio \ammonia might be
% OK, but it might also be affected by imaging artifacts from the bright radio
% sources.} [for future work w/Cara?]



%incorrect, and discussed in sec:ch3ohtem in detail
% The extraordinarily high column densities of \methanol make direct temperature
% estimation impossible; many or all of the observed \methanol lines are almost
% certainly optically thick.  Fitted rotational diagrams resulted in negative
% temperatures throughout the \methanol-enhanced region, implying that the levels
% are not thermally populated.  However, the total column densities from these
% rotational diagram fits are reasonable lower-limits on the column, and they
% exceed $N(\methanol)\gtrsim10^{20}$ \persc.
% (demonstrate this?)

% \done{If we assume $T_{ex}=100$ K, can we use the $5_{-4,2}-6_{-4,3}$ line
% to measure the column density?} [done with LTE modeling]
% Out of the five clearly detected \methanol lines, the $5_{-4,2}-6_{-4,3}$
% line seems to be subthermally excited and/or optically thin...



% \done{How does the \methanol brightness/column profile compare with the dust
% brightness/column?  Is it going up faster?  By how much?}
% answered in fig:ch3ohtemX


%Is there any evidence that the main-sequence stars that illuminate the \hii
%regions in W51 \citep{Ginsburg2016b} affect the pre-star-forming gas throughout
%W51?
%\resolved{Our ALMA program was designed to answer this question by measuring the
%temperature in the dense prestellar \formaldehyde-rich gas.  Naively, the data
%say "yes, the temperatures are all ridiculously high, $T>100$ K", but that
%can't be.  The \formaldehyde temperatures suggest that temperature is correlated
%with density, which unfortunately suggests instead that the \formaldehyde
%line optical depth is correlated with density.  It is therefore not straightforward
%to systematically examine the thermal feedback effects from MYSOs.}
%resolution: do this in the next project
%
%\resolved{Notes from chatting with Wing-Fai Thi:  \methanol has a similar condensation
%temperature to water, so the desorbed region is probably $\sim90-100$ K.  HNCO
%has a much \emph{lower} desorption temperature, so if it was coming from grain
%surfaces, it should be more widespread than \methanol.  Since it is not, the
%enhancement is most likely due to gas-phase chemistry.}
%
%\resolved{
%However, Ewine van Dishoeck pointed out that HNCO and \formamide can be mixed
%into ices that evaporate at a much higher temperature, consistent with the
%structure we observe.
%}



\section{Discussion}
\label{sec:discussion}
\subsection{The scales and types of feedback}
\label{sec:feedbackscales}
The most prominent features of our observations are the warm, chemically
enhanced regions surrounding the highest dust concentrations, and the
corresponding \emph{lack} of such features around the ionized nebulae.  This
difference implies that the immediate star formation process - that of gas
collapse and fragmentation from a molecular cloud - is primarily affected by
feedback from stars that are presently accreting and therefore emitting most of
their radiation in the infrared, \emph{not} from previous generations of
now-exposed main-sequence stellar photospheres.

On the scales relevant to the fragmentation process, i.e., the $\sim0.1$ pc
scales of prestellar cores, this decoupling can be explained simply.  Stellar
light is produced mostly in the UV, optical, and near-infrared.  As soon as a
star is exposed, either by consuming or destroying its natal core, that light
is able to stream to relatively large ($\gtrsim1$ pc) scales before being
absorbed.  At that point, the stellar radiation is poorly coupled to the scales
of direct star formation.  By contrast, stars embedded in their natal cores
will have all of their light reprocessed from UV/optical/NIR to the far-IR
within a $<0.1$ pc sphere, providing far-infrared illumination capable of
heating its surroundings.

The different effects of ionizing vs thermal radiation can be seen directly in
the three main massive star forming regions, e2, e8, and North.  Figures
\ref{fig:e2methanolhnco} and \ref{fig:northmethanolhnco} show both the
highly-excited warm molecular gas in color and the free-free emission from
ionized gas in contours.  As described in Section
\ref{sec:nonionizingradiation}, the spatial differences indicate that the
ionizing radiation sources - the exposed OB stars - have little effect on the
star-forming collapsing and fragmenting gas.

% subsubsection on the formation of the IMF?
The low impact of short-wavelength photospheric radiation on collapsing gas
suggests that second-generation star formation is relatively unaffected by its
surroundings.  Instead, the stars of the same generation - those currently
embedded and accreting - have the dominant regulating effect on the gas
temperature.  To the extent that gas temperature governs the IMF, then, the
formation of the IMF \emph{within clusters} is therefore predominantly
self-regulated, with little external influence.

%OOPS.  Good idea, but centroids are too inaccurate.
% There is additional evidence that the OB stars driving \hii regions are
% decoupled from the dense gas that is forming stars.  In the W51 e1 cluster
% of \hii regions, the molecular gas has a consistent velocity of
% $\v_{LSR}=55-56$ \kms.  The ionized gas, as measured through the H77$\alpha$
% recombination line \citep{Ginsburg2016b}, is uniformly moving at different
% velocities, $v_{\mathrm{e1}} = 

% something about implications for star formation on galactic scales


%\subsection{Question: Where does the radiation from the HII regions end up?}
%The \hii regions show no signs of heating around them.  However, we know that
%these must be $>10^4$ \lsun stars, and even the infrared radiation should be
%rising with luminosity (or temperature).  While most of the energy might go
%into ionizing the gas cloud, many of the photons must get reprocessed into the
%infrared at some point.  If optical/NIR photons were escaping, we should be
%able to see them unless the geometry is particularly unfavorable.

%\subsection{The continuum sources}
%
%We have detected 75 distinct compact `sources' and characterized some of
%their basic properties assuming they consist purely of gas and dust, but this
%interpretation is incomplete.  At the least, all of the sources with peak
%surface brightnesses $T_{B,max} > 30$ K are likely to contain central heating
%sources, i.e., stars or protostars.
%
%\subsection{Limits on accretion onto HII regions}
%\citet{Peters2010a} and \citet{Klaassen2012a} proposed that ultra- and
%hyper-compact \hii regions may be variably accreting.  When accretion is most
%active, the \hii region is confined and shrinks or may even be turned off.
%When accretion is slower or weaker, the \hii region expands, following
%approximately Str{\"o}mgren expansion \resolved{make sure that's actually what they
%say...}.  [commented out = removed]
%
%The observed lack of warm molecular gas around compact \hii regions suggests
%that they have not recently been accreting...
%
%d2 provides a counterpoint, however, as it is a hypercompact \hii region that *does*
%exhibit enhanced molecular emission in its surroundings

\subsubsection{Hot core chemical structure}
\label{sec:cheminterp}
%The enhancements in noted lines occur as factor of 3-10 increases in the peak
%brightness of most of the lines shown in Figure \ref{fig:chemmapse2}.  These
%enhancements could occur from increased total column density, increased
%abundance of the molecules, or increased excitation.
In Section \ref{sec:chemistrymaps}, we showed regions with enhanced emission
in a variety of complex chemical species over a large volume.  While it is
not generally correct to conclude that enhanced emission indicates enhanced
abundance, the additional analysis of the \methanol abundance 
in Section \ref{sec:ch3ohtem} suggests that there is a genuine enhancement in
complex chemical abundances toward these hot cores.


%In Section \ref{sec:ch3ohtem}, we examined the \methanol abundance
%and excitation conditions. 
%The detection of highly excited \methanol lines, including
%$18_{3,15}-17_{4,14}$ ($E_U=447$ K) and $25_{3,22}-24_{4,20}$ ($E_U=802$ K),
%suggests that excitation is part of the explanation for the brighter
%molecular emission.  However, LTE modeling reveals that the \methanol abundance
%increases by a factor of $\sim5-10$ from the protocluster gas inward toward the
%e2e core.  This abundance gradient is likely present in other molecules as
%well.

We have not performed a detailed abundance analysis of multiple species, but we
nonetheless suggest that these sharp-edged bubbles around the hot cores
represent desorption (sublimation) zones in which substantial quantities of
grain-processed materials are released into the gas phase.  The relatively
sharp edges likely reflect the radius at which the temperature exceeds
the sublimation temperature for each species \citep{Garrod2006a,Green2009a},
though some species may appear at temperatures above or below their sublimation
temperature if they are mixed into ices that have a different sublimation
temperature.  Other species may also form in the high-density, high-temperature
gas at smaller radii, such as the nitrogenic (HNCO, \formamide) species we
detected, suggesting that the cores are dominated by sublimated ices from
$R\sim2000-5000$ AU and by species formed in the gas phase at $R\lesssim2000$
AU.

Most of the lines identified in the hot cores e2e, e8, and North are also
present in a lower-luminosity hot core, ALMAmm14.  However, their extent is
greater toward the more luminous sources.  This difference suggests that an
examination of the relationship between the luminosity of the protostars and
the extent of their chemically enhanced zones will be useful for identifying
 very massive protostars in other regions.


\subsubsection{Outflows}
\label{sec:outflowdiscussion}
While the outflows described in Appendix \ref{sec:outflows} are impressive and
plentiful, they are obviously not the dominant form of feedback, as their area
filling factor is small compared to that of the various forms of radiative
feedback.  A low area filling factor implies a substantially smaller volume
filling factor and therefore a lower overall effect on the cloud.  However,
these outflows likely do punch holes through protostellar envelopes and the
surrounding cloud material, allowing radiation to escape.

The detection of widespread high-J \methanol emission around the highest-mass
protostars suggests that the use of \methanol as a bulk outflow tracer as
suggested by \citet{Kristensen2015a} is not viable in regions with forming
high-mass stars.  While mid-J \methanol emission associated with
the outflow (e.g., the J=10-9 transition) is detected, it is completely
dominated by the general `extended hot core' emission described in Section
\ref{sec:chemistrymaps}.

None of the outflows originate in \uchii or \hchii regions.  While a clear
origin cannot be determined for all of the outflows, it is clear that no cm
continuum sources lay at the base of any.  The lack of molecular outflows
toward these sources implies that they are accreting at most weakly.
% 
% or are
% accreting (and ejecting) only atomic or ionized material.  Since an ionized
% outflow might be directly apparent as a cm continuum jet, and any outflow
% would likely entrain molecular material, 

\subsection{The accreting phase of high-mass star formation in W51 is not ionized}
\label{sec:accretionandoutflows}
The strong outflows observed around the highest-mass forming stars, e2e, e8,
and North are clear indications of ongoing accretion onto these sources.
However, the bright \hii regions, including e2w, e1, and d2, all lack any sign
of an outflow or a surrounding rotating molecular structure.  Most of these
sources lack any surrounding molecular material at all.
%; e2w is an exception, being embedded in the e2e core, but it shows signs
% that it may be escaping from the core, rather than being its focus

Some models of high-mass star formation suggest that accretion continues
through the ionized (\hii region) phase \citep{Keto2002b,Keto2003a}.  The lack
of molecular material around the majority of the compact \hii regions in W51
suggests instead that most of the accretion is done by the time an \hii region
ignites.  Additionally, the W51e2 source, which was invoked as an example of an
ionized accretion flow in \citet{Keto2008a}, is  resolved into the e2e
hot core driving an outflow and the e2w \hchii region that is not, so the
evidence for ionized accretion onto e2w is diminished.

There is one clear example of a \hchii region surrounded by molecular gas in
our sample, the source d2.  However, it does not have an associated molecular
outflow, so there is no direct evidence of ongoing accretion.

% The correlation of the extended hot cores with outflow-driving sources suggests
% that they are actively accreting.  The heating within these hot cores may then
% be either from accretion luminosity or be produced by a central star.  With
% luminosities $L\gtrsim10^4$ \lsun, the latter is somewhat more likely.
% 
% that the high accretion rate indicated by the presence of outflows is necessary
% to reprocess stellar photospheric light into the infrared.
% rephrase, probably:
% Gas must be heated by FIR radiation.  High-mass stars produce mostly UV
% radiation.
% A. High accretion results in high luminosity infrared, which heats the gas.
% B. High accretion *reprocesses* stellar light into infrared, which heats the gas.
% C. High accretion bloats the stars.

\subsection{The accreting stars}
\label{sec:stellarproperties}
In Sections \ref{sec:W51e2e} and \ref{sec:w51e8andnorth}, we noted that the
\emph{lower limit} luminosities for the three most massive cores correspond to
early B-type photospheres.  Such stars should emit enough radiation to ignite
luminous compact \hii regions. %, $Q_{lyc}=5\ee{45}$ \pers \citep{Vacca1996a}.
The upper limits on the presence of such  \hii regions are constraining, with
$R_{\scriptsize{\hii}}<100$ AU, which is consistent with the presence of
late-type massive main-sequence stars, but inconsistent with any hotter or more
luminous
(spectral type $<$B0.5V).
If the stellar luminosities are an order of magnitude or more higher, which is
possible given the low-resolution constraints on the bolometric luminosity
\citep{Ginsburg2016a,Sievers1991a}
and likely if a large fraction of the stellar radiation is escaping along outflow
cavities \citep{Kuiper2012a,Zhang2013h}, the luminosity would be too large and
the UV radiation too small to be consistent with a main-sequence OB star.

Since we can provide only upper limits on the UV radiation, it is possible that
there is none at all.  The very large mass reservoir suggests that high
accretion rates are possible, and the bright molecular outflows show that
accretion is proceeding vigorously (though we have not quantified the rate).
Rapid accretion, and in particular rapid and \emph{variable} accretion, can
change the properties of the underlying star, bloating the star and reducing
its effective photospheric temperature \citep{Hosokawa2009a, Smith2012h,
Hosokawa2016a}.  Such stars can achieve radii $R\gtrsim200 \rsun \sim 1$ AU
while retaining photospheric temperatures $T\lesssim5000$ K.  It may be
possible to explore this scenario in more detail with higher-resolution and
lower-frequency data.

An alternative possibility is that the high accretion rates have created
a quenched \hii region \citep{Walmsley1995b,Osorio1999a,Keto2006a}.
In the spherically symmetric version of this scenario, the accretion rate
is faster than the ionization rate, such that there is always fresh neutral
material to ionize at the surface of the star.  The critical rate for
\hii region quenching is very small, $\dot{M} \sim 4\ee{-6}$ \msun \peryr
for a B0 star, so it is very likely that, even if the star has a hot
photosphere, it is not capable of driving an expanding \hii region.
The main reason to disregard this scenario is the assumption of spherical
symmetry: if the accretion is proceeding via a disk, as evidenced by the
presence of outflows, there ought to be a substantial fraction of the stellar
surface that is not directly accreting and therefore is not quenched.  If there
is a disk and an ionizing photosphere, there should be an expanding bipolar
\hii region \citep{Keto2006a}.  The lack of such a feature suggests that the
stellar photosphere is not emitting ionizing photons.

\subsubsection{Multiplicity}
\todo{Go into this more.}
High-mass stars preferentially form in multiples \citep{Zinnecker2007a}.
One explanation for the low ionizing luminosity but high total luminosity would
be the presence of many moderate-mass ($M\lesssim8$ \msun) stars forming
together.  The presence of single bipolar outflows suggests that, if there are
multiple systems, they are in all cases accreting from a common disk.  Given
the mass reservoirs available, though, there is little reason to believe that
multiplicity is the \emph{only} explanation for the high-luminosity,
low-ionizing luminosity sources: if multiples are forming, we should expect at
least one of them to reach O-star mass.

\subsection{Fragmentation: Jeans analysis}
\label{sec:fragmentation}
Fragmentation is one of the critical problems in high-mass star formation.
Assuming typical initial conditions for molecular clouds, with temperatures of
order 10 K, gas is expected to fragment into sub-solar-mass cores, preventing gaseous
material from accreting onto single high-mass stars \citep{Krumholz2015a}.
Even after high-mass stars successfully form, further fragmentation could
halt the growth of these stars and limit their final mass \citep{Peters2010a}.

Thermal Jeans fragmentation can be limited or suppressed entirely if the gas is
warm enough.  The high observed gas temperatures, $T\sim100-600$ K over
$\sim10^4$ AU, around the high mass protostars indicate that their radiative
feedback in the infrared has a dramatic effect on the gas.  The heated region
qualitatively matches that of \citet{Krumholz2006a}, who described
a core heated only by accretion luminosity down to $R=10$ AU and therefore
gave a lower limit on the total heating. 

We examined the temperature structure around the highest-mass cores in Section
\ref{sec:ch3ohtem} and the mass structure in Section \ref{sec:radialmass}.  We
put these together to measure the Jeans mass, $M_J = (\pi / 6) c_s^3
G^{-3/2} \rho^{-1/2}$
, and length, $\lambda_J = c_s G^{-1/2} \rho^{-1/2}$, 
in
Figure \ref{fig:mjeans}.  These plots show the azimuthally averaged $M_J$
and $R_J=\lambda_J/2$,
i.e., they show the Jeans mass if the medium
were of uniform density and temperature at the spherical-shell-average density
and azimuthal average temperature at each plotted radius.

The mass figure shows that the gas is stable on a beam size scale ($\sim1000$ AU),
while the length figure shows that on larger scales, the gas is unstable to
fragmentation.  However, on these larger scales, the Jeans length is about
the same as the hot core size, and the gas density and temperature structure is
highly non-uniform, violating the key assumptions of the Jeans analysis.  
A more sophisticated analysis of the fragmentation propensity of the gas
in such a core structure would be worthwhile but is beyond the scope of
this work.  We note more simply that, even if thermal fragmentation occurs, it
is likely to be greatly reduced compared to models with feedback limited to
shorter wavelengths (e.g., ionizing radiation) or with a lower total amplitude.

% \resolved{Interpretation
% of these plots is not straightforward, though: if we assume the `core' is
% spherically symmetric, the intersection of our telescope beam (which is
% approximately cylindrical) with the core is nontrivial: if the core has a
% profile $\kappa_\rho < 3$, the observed mass will  exceed the spherical shell
% mass, while if $\kappa_\rho > 3$, it will be an underestimate.}
% resolution: Hedged bets somewhat on fragmentation

% The Jeans length
% is comparable to the hot core size, so no fragmentation should be occurring.
% Additionally, the Jeans mass is anywhere from 3-10$\times$ higher than the
% (mass-weighted) peak of the stellar initial mass function (IMF),
% $M_{peak}\approx0.4$ \msun, at most radii.  The core is therefore $\sim100$
% Jeans masses, but stable against further fragmentation.
%implying that any fragments would be
%larger than the typical mass in star-forming regions.
%this is wrong
% More importantly, though, the Jeans length is significantly smaller than the
% size scale on which it is inferred throughout the entire heated core.
% The gas throughout these MYSO cores is therefore Jeans-stable and will not
% spontaneously fragment.

% If the gas at $R\sim5000$ AU is gravitationally unstable but is infalling at a
% speed $v \gtrsim 8 \kms$, it will reach the central core before a single
% free-fall time has passed, implying that the material will not successfully
% fragment.  If the exterior of the core is collapsing at such a high velocity,
% though, the implied accretion rate is $\sim0.1$ \msun \peryr.  


Within this large reservoir, there are few currently-detected fragments,
supporting the conclusions of the Jeans analysis.
In our data, within 6500 AU of W51 North, there is only 1 (ALMAmm35), around
e2e there is the HII region e2w and possibly 2-3 others between 5000 and 6500
AU, and around e8 there are none.  Admittedly, our data are not very sensitive
in the areas immediately surrounding these cores because of dynamic
range limitations, and it is not obvious that we should detect cores on such
a bright and high-column background.  Nonetheless, the lack of compact core
detections around the most massive sources is consistent with the interpretation
that fragmentation is suppressed.

%this discussion doesn't really add anything.
% Further fragmentation below our resolution limit of $\sim1000$ AU is possible
% (e.g., as seen in NGC 6334Imm1 \citep{Brogan2016a}), though the Jeans analysis
% suggests that these smaller scales are \emph{more} thermally stable.
% In the long baseline data to be presented in a future work (Goddi
% et al, in prep), very little additional fragmentation is observed.  A few more
% sources are detected toward e2e and North and many toward e8, but the nature of
% those more compact ($<200$ AU) sources suggests they are protostellar or
% stellar; they may represent previously-formed cluster members rather than
% presently-collapsing cores.  Additionally, unlike the centrally-clustered
% fragmentation seen in NGC 6334Imm1, the fragments that are detected appear
% spread out in W51, indicating they may not be within the larger hot cores but
% somewhere else along the line of sight.

%not really true
% The main sources all have the majority of their flux in relatively smooth
% structures.

\FigureTwo
{figures/azimuthalaverage_radial_mj_and_mbar_of_TCH3OH_massivecores.png}
{figures/azimuthalaverage_radial_rj_of_TCH3OH_massivecores.png}
{\todo{Shade-out sub-beam area}. The azimuthally averaged Jeans mass surrounding the
three most massive cores.  We used the \methanol temperature from
\ref{sec:methanol}, Figure \ref{fig:hmradprof}b in both the Jeans mass
calculation and the dust-based mass determination.
The density used for the mass calculation is assumed to be distributed
over spherical shells.  The dashed lines show the measured mean mass per
$\sim1000$ AU beam at each radius.  Since these masses are lower than
the local Jeans mass, the gas is stable against fragmentation.
The high variation seen at small radii (below 0.2\arcsec) is due to
sub-resolution noise.
%The dashed lines show the average mass per $\sim1000$ AU
%beam 
%The Jeans mass decreases
%toward the center at least in part because of the density increase, while the
%temperature only rises slowly.  
In (b), the dashed black line shows the $R_J=R$ line.  When $R_J > R$, i.e.,
for points falling above the black dashed line,
the gas is on average stable against fragmentation on the measured length
scale.  Since the medium is clearly \emph{not} isothermal or uniform density,
though, this only shows a rough approximation of the stability.
}
{fig:mjeans}{1}{8cm}



Given the current structure of the observed cores and their (marginal)
stability against fragmentation, it is unlikely that they could have existed at
all without the presence of a central heating source.  Should these $>200$
\msun cores have been present before high-mass star formation initiated,
resting at $T\sim20$ K as in a typical molecular cloud, they would have been
subject to Jeans fragmentation on a much smaller scale and would have formed a
cluster of smaller stars (\citet{Longmore2011a} reached the same conclusion
that high-mass cores cannot be formed with only low-mass stellar feedback as a
heating source by examining an earlier-stage high-mass star-forming region).
This prior instability implies that the mass currently in the core had to be
assembled from larger scales while suppressing or slowing collapse on smaller
scales, which is essentially the opposite of inside-out collapse
\citep{Naranjo-Romero2015a}.  In turn, such a core assembly implies that
aspects of both the `competitive accretion' and `core accretion' models may
apply, with mass dumping onto a sink source from large physical scales, yet
assembling a quasi-stable core.

The observed high gas temperature provides a mechanism for MYSOs to avoid the
``fragmentation-induced starvation'' problem discussed by
\citet{Peters2010a,Peters2010c} and \citet{Girichidis2012b}.  By suppressing
fragmentation in their surroundings, accreting massive stars are able to
sustain their mass reservoir.  A forming massive star can effectively create
its own core by cooking into one single serving the gas feast that would
otherwise form a small cluster.


\subsection{High-mass star formation within dense protoclusters: A cooperative
accretion scenario}
Since massive ($M\gtrsim250$ \msun) hot cores presently exist, and they
appear to be the precursors to a next generation of high mass stars, there
is a clear route to high-mass star formation given the presence of 
strong feedback sources.   However, this raises the question of what
the initial conditions were.  Did the
current generation of forming massive stars start life the same way
as previous generations?

One possibility hinted at by the density of hyper- and ultra-compact \hii
regions around each of the massive cores is that they did not.  Heating
(thermal feedback) from previous generations of moderate mass stars could have
warmed the gas, suppressing fragmentation into the sub-solar mass objects
typical in local clouds until enough gas was present to collapse into a
substantially larger object.  This toy model is analogous to the `cooperative
accretion' mode suggested by \citet{Zinnecker2007a}, but at a much earlier
stage in the cluster development when the gas is still molecular and dusty and
therefore capable of efficient cooling.  It is also similar to the results of
\citet{Krumholz2011a}, in which radiative heating drove up the peak of the IMF;
in this case, though, we suggest that the affected region is smaller (not the
whole cloud) and that the ``top-heavy IMF'' is a solution, not a problem.

In this scenario, the highest mass stars (probably ``very massive stars'',
$M\gtrsim50$ \msun) would only be capable of forming within dense, clustered
environments, since larger stars would be prevented from forming in other
environments by fragmentation.  The IMF would then be built up by an inverse
hierarchy, with progressively larger stars capable of forming over time until
the gas is either exhausted \citep{Ginsburg2016b} or expelled.

However, some of the assumptions in this scenario contradict our observations.
If the previous generation were responsible for substantial gas heating, we
might expect to see warm gas surrounding the \hchii regions.  Instead, we see
these stars barely interacting with the dense gas.  It is possible, though,
that these stars are only effective at dense gas heating \emph{before} they ignite
Lyman continuum emission and blow out cavities, and afterward they are merely
uninteractive witnesses to continued collapse \citep{Peters2010c}.

The alternatives to this hypothesis are that the initial conditions were
entirely set within the gas on these current scales, which seems implausible,
or that the current high density of main-sequence stars (i.e., the observed
\hchii regions) originated from a broader, lower-density distribution and
dynamically collapsed into their current clustered state.  If the
stars illuminating \hchii regions formed somewhat earlier in a more distributed
manner and fell into a common central potential, they would have had little
effect on the dense gas temperature.

\subsection{Comparison to high-mass star formation theories}
Two general classes of star formation theory, ``competitive accretion'' and
``turbulent core accretion'' have dominated the discussion of  high-mass star
formation, though they are
not entirely incompatible theories \citep{Schilke2016a}.  In their extreme forms,
they can be simplified to uniform seeds accreting different amounts of material
over their lifetime in the case of competitive accretion and a single, initial,
quasistatic high-mass gas core collapsing into a star in the case of turbulent
core accretion.

Our observations of high-mass hot cores extending to radii $\sim5000$ AU do
not directly contradict either of these extremes, but instead suggests
that the accretion models used within these theories must be modified.

% Real point: feedback suppresses formation of new fragments.
If there are many simultaneously formed low-mass fragments competing for
material, as in competitive accretion, it appears that the formation of a
dominant most massive star dramatically changes the conditions for accretion.
This single central source heats material enough to reduce the efficiency of
Bondi-Hoyle accretion by a factor of 10 for a stationary object, though the
reduction is negligible for an object moving near the Virial velocity ($v_{vir}
\gtrsim 5 \kms$ within the hot cores).  This extended warm region
suppresses the formation of new small fragments, as has been observed in simulations
with much weaker feedback \citep[e.g.][]{Bate2009a}. 
Most likely, the material around these most massive protostars was accumulated
from larger scales, as in \citet{Smith2009f}.  

% DONE: consider dropping kapparho discussion entirely; the measurements are
% pretty damned unreliable. (dropped kapparho discussion)
If the turbulent core model is used to describe these sources, a few 
issues arise.  First, a `cold' turbulent core with the current density configuration
would have been dramatically unstable to fragmentation on small scales.  It
therefore is unlikely that a static model, with a single preassembled
precursor core, is an appropriate initial condition for these forming high-mass
stars.
% First, the core structure is not a simple power-law with
% $\kappa_\rho=1.5$, as usually assumed
% \citep{McKee2003a,Zhang2011a}\footnote{The parameter $\kappa_\rho$ governs the
% density profile under an assumed power-law configuration, $\rho(r) =
% r^{-\kappa_{\rho}}$}; instead we find that the profile varies across the core,
% steepening from the center to the exterior.
% % Figure \ref{fig:kapparho} shows instead that the core is
% % overall steeper. 
%Of course, the observations do not correspond to the initial
%conditions, so this comparison could more fairly be made with a later
%`snapshot' from those models.
Second, the more sophisticated models of evolution from an initial turbulent
core suggest that most of the stellar (and accretion) radiation should escape
via growing outflow cavities \citep{Zhang2011a,Zhang2013h,Zhang2014e}.  The
circular symmetry seen in the hot core around e2e and its lack of
correspondence with the outflowing material (Figure
\ref{fig:outflowonmethanol}) indicates that a large fraction of the stellar or
accretion radiation is going directly into the core and not escaping along the
poles.  This failure of the outflow escape valves suggests a modification of
the core evolution within the turbulent core scenario is needed.

The presence of these large-scale, massive, hot cores raises a few questions
that are not presently well-addressed by any class of theories.  If the hot
cores successfully suppress further fragmentation in their envelopes, does that
imply that these cores will collapse into single stars?  If so, do the cores in
any way correspond to `initial conditions', or are the cores created by the
stars that form within them?  This question is more than merely semantics,
as the evolutionary progress within the `turbulent core' model depends on the
initial core structure, and the presence of nearby fragments governs whether
`competitive accretion' can occur at all.

There are also implications for `isolated' massive star formation.  The stars
we see forming are clearly not isolated - they are embedded within fairly rich
clusters including massive stars we can see already, and they are very likely
to be surrounded by substantial clusters of lower-mass stars.  However,
\emph{during their formation process}, they may be the only accreting stars
within a few thousand AU neighborhood - their thermal feedback may prevent any
neighbors from forming simultaneously.  This `enforced isolation' might be a
way for massive star formation to proceed similarly independent of the size of
the parent cloud: high-mass stars will form the same way whether in an
`isolated' or `clustered' region because they govern gas conditions in their
own surroundings.

% methanol_and_outflow.py
\Figure{figures/outflows_over_methanol_e2e.png}
{The e2e core as seen in the \methanol $8_{0,8}-7_{1,6}$ line with
the integrated \twelveco 2-1 outflow overlaid in red (73 to 180 \kms) and blue
(0 to 45 \kms).  The `core' is circularly symmetric, while the outflow is
clearly bipolar.}
{fig:outflowonmethanol}{1}{10cm}

%\done{Do I need a plot of gravitational pressure / thermal pressure vs radius?
%probably yes.}  DONE: This is basically the Jeans plots, even though it's not
%put in those terms...

%\section{An evolutionary sequence...}

% Expanding HII regions are more likely due to stars leaving their dense gas
% regions than blowing away their surroundings...


%-compare to ngc 6334i? (probably needs better resolution)

% ? -optical depth map using $T_B$ and T(\methanol)

% useful questions:
% how much of the 'core' mass will end up in a single star?  single system?

% \subsection{The core mass function}
% \done{continue this} REMOVED
% Naively, with the conclusions of Section \ref{sec:fragmentation} in hand, one
% might be tempted to conclude that the core-to-star mapping, which is a critical
% component of core mass function (CMF) based theories, is constant with mass.
% This conclusion cannot be safely drawn from our data, as the massive `core'
% we have observed is far larger than any other source in our sample.
%
% Did e2w "create" e2e by heating a core and stopping fragmentation?
% Is this a form of cooperative accretion?
%
% Does the core mass function actually work because accretion feedback heating
% keeps the massive core warm and puffy?
% Does the CMF fail because time-ordered star formation (e2w->e2e) means there
% is no instantaneously observable CMF?

%\subsection{Possible additional discussion points}
%-Can we determine whether the luminosity is powered by accretion or fusion?

% \subsection{Low-mass star formation in and around the MYSOs}
% \citet[TODO: Cyganowski 2016a][]{Cyganowski2016a} observed a region in which
% high-mass and low-mass star formation is occurring concurrently.  The same
% is almost certainly occurring in W51.  
% \todo{Is there more to say here?}

\subsection{The population of faint, compact sources}
\label{sec:faintsrcs_discussion}
In Section \ref{sec:sourceid}, we reported the identification of 75 or 113
sources, depending on identification method, but we have not discussed them since.
In this section, we briefly discuss why we have chosen to ignore these sources
and offer some cautions and prospects for the interpretation of millimeter sources
in crowded star-forming regions. 
% One of the original aims of this study was to measure
% the `core mass function', but after examining the data we determined that this
% is not possible.

First, observational considerations have hindered the detection of `cores' down
to our theoretical mass limit $M\sim1\msun$: the  extended millimeter continuum
structure resulted in imaging artifacts that limited the dynamic range in the
vicinity of bright sources.  This effect means that we have a wildly varying
completeness to compact sources across our image, and the completeness is
difficult or perhaps impossible to measure.

Second, the nature of the sources makes determination of their masses quite
difficult.  In Appendix \ref{sec:contsrcs}, we discuss our attempts to
determine the temperature and structure of the individual sources and
characterize them based on their morphology.  Most of these sources are
centrally concentrated with high central brightness temperatures, suggesting
that they are protostellar, which in turn means that any dust-derived mass is
subject to dramatic uncertainties in the unresolved temperature structure.
More diffuse and cool sources, those that appear similar to local prestellar
cores, are theoretically detectable in our data, and some even are detected
(those with classification \texttt{-C-} in Table \ref{tab:photometry1}), but in
practice can only be seen in regions with no other substructure, especially
free-free emission, that are rare in the observed area.

These fainter sources are the most prominent members of the next generation of
stars forming in this cloud, but we do not yet have the tools needed to
understand them.  In order to probe the formation of the IMF, we need to be
able to determine the masses of stars forming in the millimeter sources, which
likely requires multiwavelength data.  

These limitations prevent us from drawing any robust conclusions from these
data.  However, they are presented in full detail in Appendix
\ref{sec:contsrcs}, including description of the spatial structure,
characterization technique, and flux distribution.



\section{Conclusions}
\label{sec:conclusion}

We have presented ALMA 1.3 mm (227 GHz) continuum and line observations of the
high-mass protocluster W51.  We examined the three most massive forming stars
and the surrounding population of forming stars.  


The key observational results include:
\begin{itemize}
    \item We identified chemically enhanced regions around high-mass protostars
        and suggest they are radiatively heated zones in which previously
        frozen-out chemical species have desorbed into the gas phase.
    \item We measured the temperature and mass structure surrounding the three
        highest-mass cores, W51 e2e, e8, and North.  All three have masses
        $M>200$ \msun within $R\lesssim5000$ AU.  The core temperatures
        inferred from LTE modeling of \methanol are $100 \mathrm{K} < T < 600
        \mathrm{K}$, which brightness-based lower limits $T>100$ K confirm.
        Their centers are likely to be optically thick in the dust continuum.
    \item The flux density recovered in the ALMA map above a threshold of 10
        mJy \perbeam is $\sim30\%$ of that seen in single-dish data.  At this
        brightness, cold (20 K) dust would correspond to a column density
        $N(\hh)>10^{25}$ \persc, leading us to conclude that a large fraction
        of the cloud - likely all of the gas emitting at over 10 mJy \perbeam -
        is warmer than 20 K.
    \item We demonstrated three techniques for estimating the temperature of
        dust continuum sources with ALMA: limits from dust brightness
        temperature, limits from line brightness temperature, and LTE modeling
        of \methanol lines.  While these methods do not necessarily agree
        or provide direct and accurate measurements of the temperature, they
        provide strong enough constraints to draw substantial physical conclusions.
    \item We cataloged and classified 75 continuum sources on the basis of their
        dust emission, line emission, and morphology.  These sources are
        prestellar cores, dust-enshrouded protostars, and hypercompact \hii
        regions.
\end{itemize}

From these observations, we have inferred the following: 
\begin{itemize}
    \item During the earliest stages of their formation, before they have 
        ignited \hii regions, high-mass protostars heat a large volume
        ($R\sim5000 AU$), and correspondingly large mass, of gas around them.
    \item Older massive stars, those with surrounding \hii regions, appear
        to have little effect on the temperature of dense gas around them.
        Instead, their feedback primarily affects larger ($\gtrsim 0.1$ pc)
        scales.
        %While some are embedded within larger hot cores, they do not appear to
        %be the main heating sources.  Their luminosity must therefore escape
        %without being locally reprocessed into infrared radiation.
    \item Because older generation stars have little impact on the star-forming
        molecular gas, it appears that the star formation process and the IMF
        are \emph{self-regulated} within clusters; they are unaffected by
        external forces like main-sequence OB-star feedback.
    \item Heated massive cores surrounding the highest-mass protostars in
        W51 show no evidence of ongoing fragmentation and are warm enough
        to suppress Jeans fragmentation.  Hot massive cores therefore serve as
        a mass reservoir for accretion onto possibly single massive stars.
    \item Hot massive cores are heated by the accreting stars within them,
        implying that young massive stars self-regulate their  core
        structures.
\end{itemize}



\textbf{Acknowledgments}
The National Radio Astronomy Observatory is a facility of the National Science
Foundation operated under cooperative agreement by Associated Universities,
Inc.
This paper makes use of the following ALMA data: 2013.1.00308.S 
and 2015.1.01596.S.
ALMA is a partnership of ESO (representing its member states), NSF (USA) and
NINS (Japan), together with NRC (Canada), NSC and ASIAA (Taiwan), and KASI
(Republic of Korea), in cooperation with the Republic of Chile. The Joint ALMA
Observatory is operated by ESO, AUI/NRAO and NAOJ.
JMDK gratefully acknowledges funding in the form of an
Emmy Noether Research Group from the Deutsche Forschungsgemeinschaft (DFG),
grant number KR4801/1-1.

\textbf{Code Packages Used}:

\begin{itemize}
    \item CASA \url{https://casa.nrao.edu/}
    \item pyspeckit \url{http://pyspeckit.bitbucket.org} \citet{Ginsburg2011c}
    \item aplpy \url{https://aplpy.github.io/}
    \item wcsaxes \url{http://wcsaxes.readthedocs.org}
    \item spectral cube \url{http://spectral-cube.readthedocs.org}
    \item ds9 \url{http://ds9.si.edu}
    \item \texttt{dust\_emissivity} \url{dust_emissivity.readthedocs.org}
\end{itemize}

\ifstandalone
\bibliographystyle{apj_w_etal}  % or "siam", or "alpha", or "abbrv"
%\bibliography{thesis}      % bib database file refs.bib
\bibliography{bibdesk}      % bib database file refs.bib
\fi


%\resolved{idea: dust is more concentrated than gas (compare PSDs).  Can we do this to
%sims and determine which lines are thick?} not doing this

\appendix

\section{Additional observational details}
In this Appendix, we present the complete list of imaged lines.

\begin{table*}[htp]
\caption{Spectral Lines in SPW 0}
\begin{tabular}{ll}
\label{tab:linesspw0}
Line Name & Frequency \\
 & $\mathrm{GHz}$ \\
\hline
H$_2$CO $3_{0,3}-2_{0,2}$ & 218.22219 \\
H$_2$CO $3_{2,2}-2_{2,1}$ & 218.47564 \\
E-CH$_3$OH $4_{2,2}-3_{1,2}$ & 218.44005 \\
CH$_3$OCHO $17_{3,14}-16_{3,13}$E & 218.28083 \\
CH$_3$OCHO $17_{3,14}-16_{3,13}$A & 218.29787 \\
CH$_3$CH$_2$CN $24_{3,21}-23_{3,20}$ & 218.39002 \\
Acetone $8_{7,1}-7_{4,4}$AE & 218.24017 \\
O$^{13}$CS 18-17 & 218.19898 \\
CH$_3$OCH$_3$ $23_{3,21}-23_{2,22}$AA & 218.49441 \\
CH$_3$OCH$_3$ $23_{3,21}-23_{2,22}$EE & 218.49192 \\
CH$_3$NCO $25_{1,24} - 24_{1,23}$ & 218.5418 \\
CH$_3$SH $23_2-23_1$ & 218.18612 \\
\hline
\end{tabular}

\end{table*}


\begin{table*}[htp]
\caption{Spectral Lines in SPW 1}
\begin{tabular}{ll}
\label{tab:linesspw1}
Line Name & Frequency \\
 & $\mathrm{GHz}$ \\
\hline
H$_2$CO $3_{2,1}-2_{2,0}$ & 218.76007 \\
HC$_3$N 24-23 & 218.32471 \\
HC$_3$Nv$_7$=1 24-23a & 219.17358 \\
HC$_3$Nv$_7$=1 24-23a & 218.86063 \\
HC$_3$Nv$_7$=2 24-23 & 219.67465 \\
OCS 18-17 & 218.90336 \\
SO $6_5-5_4$ & 219.94944 \\
HNCO $10_{1,10}-9_{1,9}$ & 218.98102 \\
HNCO $10_{2,8}-9_{2,7}$ & 219.73719 \\
HNCO $10_{0,10}-9_{0,9}$ & 219.79828 \\
HNCO $10_{5,5}-9_{5,4}$ & 219.39241 \\
HNCO $10_{4,6}-9_{4,5}$ & 219.54708 \\
HNCO $10_{3,8}-9_{3,7}$ & 219.65677 \\
CH$_3$OH $8_{0,8}-7_{1,6}$ & 220.07849 \\
CH$_3$OH $25_{3,22}-24_{4,20}$ & 219.98399 \\
CH$_3$OH $23_{5,19}-22_{6,17}$ & 219.99394 \\
C$^{18}$O 2-1 & 219.56036 \\
H$_2$CCO 11-10 & 220.17742 \\
HCOOH $4_{3,1}-5_{2,4}$ & 219.09858 \\
CH$_3$OCHO $17_{4,13}-16_{4,12}$A & 220.19027 \\
CH$_3$CH$_2$CN $24_{2,22}-23_{2,21}$ & 219.50559 \\
Acetone $21_{1,20}-20_{2,19}$AE & 219.21993 \\
Acetone $21_{1,20}-20_{1,19}$EE & 219.24214 \\
Acetone $12_{9,4}-11_{8,3}$EE & 218.63385 \\
H$_2$$^{13}$CO $3_{1,2}-2_{1,1}$ & 219.90849 \\
SO$_2$ $22_{7,15}-23_{6,18}$ & 219.27594 \\
SO$_2$ $v_2=1$ $20_{2,18}-19_{3,17}$ & 218.99583 \\
SO$_2$ $v_2=1$ $22_{2,20}-22_{1,21}$ & 219.46555 \\
SO$_2$ $v_2=1$ $16_{3,13}-16_{2,14}$ & 220.16524 \\
\hline
\end{tabular}
\par
The Categories column consists of three letter codes as described in Section \ref{sec:contsourcenature}.In column 1, \texttt{F} indicates a free-free dominated source,\texttt{f} indicates significant free-free contribution,and \texttt{-} means there is no detected cm continuum.In column 2, the peak brightness temperature is used toclassify the temperature category.\texttt{H} is `hot' ($T>50$ K), \texttt{C} is `cold' ($T<20$ K), and \texttt{-} is indeterminate (either $20<T<50$K or no measurement)In column 3, \texttt{c} indicates compact sources, and \texttt{-} indicates a diffuse source.
\end{table*}


\begin{table*}[htp]
\caption{Spectral Lines in SPW 2}
\begin{tabular}{ll}
\label{tab:linesspw2}
Line Name & Frequency \\
 & $\mathrm{GHz}$ \\
\hline
$^{12}$CO $2-1$ & 230.538 \\
OCS 19-18 & 231.06099 \\
HNCO $28_{1,28}-29_{0,29}$ & 231.873255 \\
CH$_3$OH $10_{2,9}-9_{3,6}$ & 231.28115 \\
$^{13}$CS 5-4 & 231.22069 \\
NH$_2$CHO $11_{2,10}-10_{2,9}$ & 232.27363 \\
H30$\alpha$ & 231.90093 \\
CH$_3$OCHO $12_{4,9}-11_{3,8}$E & 231.01908 \\
CH$_3$CH$_2$OH $5_{5,0}-5_{4,1}$ & 231.02517 \\
CH$_3$OCH$_3$ $13_{0,13}-12_{1,12}$AA & 231.98772 \\
N$_2$D+ 3-2 & 231.32183 \\
g-CH$_3$CH$_2$OH $13_{2,11}-12_{2,10}$ & 230.67255 \\
g-CH$_3$CH$_2$OH $6_{5,1}-5_{4,1}$ & 230.79351 \\
g-CH$_3$CH$_2$OH $16_{5,11}-16_{4,12}$ & 230.95379 \\
g-CH$_3$CH$_2$OH $14_{0,14}-13_{1,13}$ & 230.99138 \\
SO$_2$ $v_2=1$ $6_{4,2}-7_{3,5}$ & 232.21031 \\
CH$_3$SH $16_2-16_1$ & 231.75891 \\
CH$_3$SH $7_3-8_2$ & 230.64608 \\
\hline
\end{tabular}
\par
The Categories column consists of three letter codes as described in Section \ref{sec:contsourcenature}.In column 1, \texttt{F} indicates a free-free dominated source,\texttt{f} indicates significant free-free contribution,and \texttt{-} means there is no detected cm continuum.In column 2, the peak brightness temperature is used toclassify the temperature category.\texttt{H} is `hot' ($T>50$ K), \texttt{C} is `cold' ($T<20$ K), and \texttt{-} is indeterminate (either $20<T<50$K or no measurement)In column 3, \texttt{c} indicates compact sources, and \texttt{-} indicates a diffuse source.
\end{table*}


\begin{table*}[htp]
\caption{Spectral Lines in SPW 3}
\begin{tabular}{ll}
\label{tab:linesspw3}
Line Name & Frequency \\
 & $\mathrm{GHz}$ \\
\hline
A-CH$_3$OH $4_{2,3}-5_{1,4}$ & 234.68345 \\
E-CH$_3$OH $5_{-4,2}-6_{-3,4}$ & 234.69847 \\
A-CH$_3$OH $18_{3,15}-17_{4,14}$ & 233.7958 \\
$^{13}$CH$_3$OH $5_{1,5}-4_{1,4}$ & 234.01158 \\
PN $5-4$ & 234.93569 \\
NH$_2$CHO $11_{5,6}-10_{5,5}$ & 233.59451 \\
Acetone $12_{11,2}-11_{10,1}$AE & 234.86136 \\
SO$_2$ $16_{6,10}-17_{5,13}$ & 234.42159 \\
CH$_3$NCO $27_{2,26} - 26_{2,25}$ & 234.08812 \\
CH$_3$SH $15_2-15_1$ & 234.19145 \\
\hline
\end{tabular}

\end{table*}


\section{A bubble around e5}
\label{sec:e5bubble}
There is evidence of a bubble in the continuum around e5 with a radius of
6.2\arcsec (0.16 pc; Figure \ref{fig:e5bubble}).  The bubble is completely
absent in the centimeter continuum, so the observed emission is from dust.  The
bubble edge can be seen from 58 \kms to 63 \kms in \ceighteeno and
\formaldehyde, though it is not contiguous in any single velocity channel.
There is a collection of compact sources (protostars or cores) along the
southeast edge of the bubble.

The presence of such a bubble in dense gas, but its absence in ionizing gas, is
surprising.  The most likely mechanism for blowing such a bubble is ionizing
radiative feedback, especially around a source that is currently a hypercompact
HII region, but since no free-free emission is evident within or on the edge of
the bubble, it is at least not presently driving the bubble.  A plausible
explanation for this discrepancy is that e5 was an exposed O-star within the
past Myr, but has since begun accreting heavily (or has traveled into a region
of high density) and therefore had its HII
region shrunk.  This model is marginally supported by the presence of a `pillar'
of dense material pointing from e5 toward the south.

The total flux in the north half of the `bubble', which shows no signs of
free-free contamination, is about 1.5 Jy.  The implied mass in just this
fragment of the bubble is about $M\sim350$ \msun for a relatively high assumed
temperature $T=50$ K.  The total mass of the bubble is closer to $M\sim1000$
\msun, though it may be lower ($\sim500$ \msun) if the southern half is
dominated by free-free emission.

With such a large mass, the implied density of the original cloud, assuming it
was uniformly distributed over a 0.2 pc sphere, is $n(\hh) \approx 2-5\ee{5}$
\percc.

% this analysis courtesy Jim Dale
To evaluate the plausibility of the \hii-region origin of the bubble, we compare
to classical equations for \hii regions.
The Str\"omgren radius is \\
\begin{eqnarray}
R_{\rm s}=\left(\frac{3Q_{\rm H}}{4\pi\alpha_{\rm B} n^{2}}\right)^{\frac{1}{3}}.
\end{eqnarray} 
For $Q_{\rm H}\sim10^{49}$ \pers (corresponding to an $M\approx40$\msun
main-sequence star), $\alpha_{\rm B}=3\times10^{-13}$\,cm$^{3}$\,s$^{-1}$,
$R_{\rm s}\approx0.01$\,pc.\\
\\
The Spitzer solution for HII region expansion gives\\
\begin{eqnarray}
R_{\rm HII}(t)=R_{\rm s}\left(1+\frac{7}{4}\frac{c_{\rm II}t}{R_{\rm s}}\right)^{\frac{4}{7}}.
\end{eqnarray} 
With $c_{\rm II}=7.5$\,km\,s$^{-1}$ and $t=10^{4}$\,yr,
$R_{\rm HII}(t)\approx0.04$\,pc, while at $t=10^5$\,yr, it is $R_{\rm
HII}\approx0.16$\,pc, which is comparable to the observed radius
($r_{obs} \sim 0.13-0.19$ pc)\\
\\
Whitworth et al. 1994 give the fragmentation timescale as
\begin{eqnarray}
t_{\rm frag}\sim1.56\left(\frac{c_{\rm s}}{0.2{\rm km\,s}^{-1}}\right)^{\frac{7}{11}}\left(\frac{Q_{\rm H}}{10^{49}{\rm s}^{-1}}\right)^{-\frac{1}{11}}\left(\frac{n}{10^{3}{\rm cm}^{-3}}\right)^{-\frac{5}{11}}{\rm Myr}.
\end{eqnarray} 
Plugging in our numbers gives $t_{\rm frag}\approx1.0\times10^{5}$\,yr, or
$10\times$ longer than the expansion time.\\
\\
% The corresponding radius at which fragmentation occurs is\\
% \begin{eqnarray}
% R_{\rm frag}\sim5.8\left(\frac{c_{\rm s}}{0.2{\rm km\,s}^{-1}}\right)^{\frac{4}{11}}\left(\frac{Q_{\rm H}}{10^{49}{\rm s}^{-1}}\right)^{\frac{1}{11}}\left(\frac{n}{10^{3}{\rm cm}^{-3}}\right)^{-\frac{6}{11}}{\rm pc},
% \end{eqnarray}
% which gives us $R_{\rm frag}\approx0.17$\,pc.\\

These values are consistent with a late O-type star having been exposed,
driving an \hii region, for $\sim10^4-10^5$ years, after which a substantial
increase in the local density quenched the ionizing radiation from the star,
trapping it into a hypercompact ($r<0.005$ pc) configuration.  The
recombination timescale is short enough that the ionized gas would disappear
almost immediately after the continuous ionizing radiation source was hidden.
This is essentially the scenario laid out in \citet{de-Pree2014a} as an
explanation for the compact \hii region lifetime problem.  In this case,
however, it also seems that the \hii region has effectively driven the
``collect'' phase of what will presumably end in a collect-and-collapse style
triggering event.

Technically, it is possible that e5 actually represents an optically thick
high-mass-loss-rate wind rather than an ultracompact HII region. 
%but I think we
%can rule this out on physical plausibility considerations if we compare to 
%wind models.
For example, $\eta$ Car would have a flux of $\sim0.5$ Jy at 2 cm
and $\sim5$ Jy at 1 mm at the distance of W51.  While we cannot rule out
this possibility, it would render the association of e5 with the `bubble'
purely coincidental.

\FigureTwo{figures/e5_bubble.png}{figures/e5_bubble_robust2.png}
{The bubble around source e5.  The bubble interior shows no sign of centimeter
emission, though the lower-left region of the shell - just south of the
``cores'' - coincides with part of the W51 Main ionized shell.  The source of
the ionization is not obvious.
({\it Left}): A robust -2.0 image with a small (0.2\arcsec) beam and poor
recovery of large angular scale emission.  This image highlights the presence
of protostellar cores on the left edge of the bubble and along a filament just
south of the central source.
({\it Right}): A robust +2.0 image with a larger (0.4\arcsec) beam and better
recovery of large angular scales.  The contours show radio continuum (14.5 GHz)
emission at 1.5, 3, and 6 mJy/beam.  While some of the detected 1.4 mm emission
in the south could be free-free emission, the eastern and northern parts of the
shell show no emission down to the 50 $\mu$Jy noise level of the Ku-band map,
confirming that they consist only of dust emission.
}{fig:e5bubble}{1}{8cm}


\section{Outflows}
\label{sec:outflows}
We detected many outflows, primarily in CO 2-1 and SO $6_5-5_4$.  The flows are
weakly detected in some other lines, e.g. \formaldehyde, but we defer
discussion of outflow chemistry to a future work.

In this section, we discuss some of the unique outflows and unique features of
outflows in the W51 region.  We show the most readily identified outflows in
Figures \ref{fig:outflowscontinuumnorth} - \ref{fig:e8cooutflow}.
%\ref{fig:outflowscontinuume2}, 
% \ref{fig:e2ecooutflow}, and

\subsection{The Lacy jet}
\label{sec:lacyjet}
A high-velocity outflow was discovered within the W51 IRS2 region by
\citet{Lacy2007a}, and subsequently detected in H77$\alpha$ by
\citet{Ginsburg2016b}.  We have discovered the CO counterpart to this
outflow, which comes from near the continuum source ALMAmm31 (Figure
\ref{fig:lacyjet}).  Strangely, though, the outflow is not directly centered on
the millimeter continuum source, but is slightly offset.  The outflow shows red- and
blue-shifted lobes that form the base of the ionized outflow reported by
\citet[][Figure \ref{fig:outflowscontinuumnorth}]{Lacy2007a}.

The presence of the Lacy jet is important for ruling out outflows from \hii
regions.  It provides clear evidence that a molecular outflow that is
subsequently ionized can be easily detected in existing radio recombination
line data.  If outflows of comparable mass were being launched from the stars
at the centers of \hchii regions (e.g., e2w), we would detect these flows.
Their absence provides an upper limit on the outflow rate - and presumably the
accretion rate - onto these sources.  While we cannot yet make that limit
quantitative, it is clear that the \hchii region sources are accreting
substantially less than the dust continuum sources.


%\removed{Make a figure of this \& describe data reduction if it is to be included.}
%Additionally, we have reduced archival VLT SINFONI observations of the region
%and discovered a 2-micron \hh knot positioned directly between the cold
%molecular (CO) and the ionized components of the flow.  This \hh emission
%reveals the position at which the CO is breaking out of the cloud and into the
%\hii region.

\FigureTwo
{figures/rgb_CO_continuum_outflows_aplpy_wideLacy.png}
{figures/rgb_SO_continuum_outflows_aplpy_wideLacy.png}
{Outflows shown in red and blue for (a) CO 2-1 and (b) SO $6_5-5_4$ with
continuum in green.  This symmetric molecular outflow forms the base of the
\citet{Lacy2007a} ionized outflow detected further to the east.
The continuum source is offset from the line joining the red and blue outflow lobes.}
{fig:lacyjet}{1}{8cm}


\Figure{{figures/NACO_green_outflows_aplpy_CONTours_hires_h77acontour}.png}
{Outflows in the W51 IRS2 region.  The green emission is NACO K-band continuum
\citep{Figueredo2008a,Barbosa2008a}, with ALMA 1.4 mm continuum contours in white and
H77$\alpha$ contours in blue.  The \citet{Lacy2007a} jet is prominent in
H77$\alpha$.}
{fig:outflowscontinuumnorth}{1}{12cm}

\subsection{north}
The outflow from W51 North is extended and complex.
A jet-like high-velocity feature appears directly to the north of W51 North in
both CO and SO (Figure \ref{fig:outflowscontinuumnorth}).  However, in SO, this feature begins to emit at $\sim47$ \kms
and continues to $\sim 100$ \kms.  The CO emission below $<70$ \kms is
completely absent, presumably obscured by foreground material.  The blueshifted
component, by contrast with the red, points to the southeast and is barely
detected in CO, but again cleanly in SO.  It is sharply truncated, extending
only $\sim1 \arcsec$ ($\sim5000$ AU).  Unlike the Lacy jet, there is no
evidence that this outflow transitions into an externally ionized state.

The northernmost point of the W51 North outflow may coincide with
the \citet{Hodapp2002a} \hh and [Fe II] outflow.  There is some CO 2-1
emission coincident with the southernmost point of the \hh features,
and these all lie approximately along the W51 North outflow vector.
However, the association is only circumstantial.


\subsection{The e2e outflow}
The dominant outflow in W51, which was previously detected by the SMA
\citep{Shi2010b,Shi2010a}, comes from the source e2e.  This outflow is
remarkable for its high velocity, extending nearly to the limit of our spectral
coverage in $^{12}$CO.  The ends of the flow cover at least $-50 < v_{LSR} <
160$ \kms, or a velocity $v\pm100$ \kms.  

The morphology is also notable.  Both ends of the outflow are sharply truncated
at $\sim2.5\arcsec$ (0.07 pc) from e2e (Figure \ref{fig:outflowscontinuume2}).
To the southeast, the high-velocity flow lies along a line that is consistent
with the extrapolation from the northwest flow, but at lower velocities ($10 <
v_{LSR} < 45$ \kms), it jogs toward a more north-south direction (Figure
\ref{fig:e2ecooutflow}).  In the
northwest, the redshifted part of this flow ($70 < v_{LSR} < 120$ \kms)
apparently collides with a \emph{blue}shifted flow from another source ($22 <
v_{LSR} < 45$ \kms), suggesting that these outflows intersect, though such a
scenario seems  implausible given their small volume filling factor.

% really?  important implications?  Maybe not. [edit: -important]
The extreme velocity and morphology carry a few implications for the
accretion process in W51.  The sharp symmetric truncation at the outflow ends,
combined with the extraordinary velocity, suggests that the outflow is freshly
carving a cavity in the surrounding dense gas.  The observed velocities are
high enough that their bow shocks likely dissociated all molecules, so some
ionized gas is likely present at the endpoints; this ionized gas has not been
detected in radio images because of the nearby 100 mJy HCHII region e2w.  The
dynamical age of the outflow is $\sim600$ years at the peak observed velocity,
which is a lower limit on the true age of the outflow.


\Figure{{figures/Alma1.4mmcont_green_outflows_aplpy_CONTours}.png}
{Outflows in red and blue overlaid on mm continuum in green with cm continuum
contours in white.  The northern source is e2, the southern source at the tip
of the long continuum filament is e8.}
{fig:outflowscontinuume2}{1}{12cm}


\Figure{{figures/e2e_CO2-1_channelmaps}.png}
{Channel maps of the e2e outflow in CO 2-1.  The dashed line approximately
connects the northwest and southeast extrema of the flow.}
{fig:e2ecooutflow}{1}{18cm}

\subsection{e8}
There are at least four distinct outflows coming from the e8 filament.
The e8 core is launching a redshifted outflow to the northwest.  A blueshifted
outflow is coming from somewhere south of the e8 peak and pointing straight
east.  While these originate quite near each other, they seem not to have
a common source, since the red and blue streams are not parallel (Figures
\ref{fig:outflowscontinuume2} and \ref{fig:e8cooutflow}).  The e8 outflows are too
confused and asymmetric for simple interpretation.


\Figure{{figures/e8_CO2-1_channelmaps}.png}
{Channel maps of the e8 outflow in \twelveco 2-1.  The outflows here are more
erratic, with fewer clearly-connected red and blue lobes.}
{fig:e8cooutflow}{1}{18cm}

\section{Details of the extracted sources}
\label{sec:contsrcs}
We provide additional information and details about the continuum
source extraction, along with complete catalogs, in this Appendix.

\subsection{The spatial distribution of continuum sources}
\label{sec:corespatialdistribution}
The detected continuum sources are not uniformly distributed across the
observed region.  The most notable feature in the spatial distribution is their
alignment: most continuum sources collect along approximately linear features.
This is especially evident in W51 IRS2, where the core density is very high and
there is virtually no deviation from the line.  The e8 filament is also notably
linear, though there are a few sources detected just off the filament. 

On a larger scale, the e8 filament points toward e2, apparently tracing a
slightly longer filamentary structure that is either lower-column or resolved
out by our data.  With some imagination, this might be extended along the
entire northeast ridge to eventually connect in a broad half-circle with the
IRS2 filament (Figure \ref{fig:corepositions}).  This morphology hints at a
possible sequential star formation event, where some central bubble has swept
gas into these filaments.  However, there is reason to be skeptical of this
interpretation: this ring has no counterpart in ionized gas as would be
expected if it were driven as part of an expanding \hii region or a wind
bubble, and there is little reason to expect such circular symmetry from an
isolated molecular cloud, so the star forming circle may be merely a
coincidental alignment.

Whether it is physical or not, there is a relative lack of millimeter continuum
sources within the circle.  There is no lack of molecular gas, however, as both
CO and \formaldehyde emission fill the full field of view.

\Figure{figures/core_spatial_distribution.png}
{The spatial distribution of the hand-identified core sample.
The black outer contour shows the observed field of view.  The dashed circle
(with $r=1$ pc) shows a hypothetical ring of star formation.
The velocities shown are the mean of the velocity of peak intensity for many
lines.
}{fig:corepositions}{1}{16cm}


\subsection{Photometry}
\label{sec:photometry}
We created a catalog of the hand-extracted sources including their peak and mean
intensity, their centroid, and their geometric properties.  For each source,
we further extracted aperture photometry around the centroid in 6 apertures:
0.2, 0.4, 0.6, 0.8, 1.0, and 1.5\arcsec.  We performed the same aperture
photometry on the W51 Ku-band images from \citet{Ginsburg2016a} to estimate the
free-free contribution to the observed intensity measurements.  The
free-free contribution at $\sim227$ GHz will fall in a range between optically
thick, spectral index $\alpha_\nu=2$, and optically thin, $\alpha_\nu=0.1$,
which correspond to factors of $S_{227 \mathrm{GHz}} = 227 S_{15 \mathrm{GHz}}$
and $S_{227 \mathrm{GHz}} = 1.3 S_{15 \mathrm{GHz}}$, respectively.  These
measurements are reported in Table
\ref{tab:photometry1}.

The source flux density and intensity distribution are shown in Figure
\ref{fig:fluxhistograms}.  The most common nearest-neighbor separation between
cataloged sources is $\sim0.3\arcsec$, which implies that the larger apertures
double-count some pixels.  The smallest separation is 0.26\arcsec, so the
0.2\arcsec\ aperture contains almost only unique pixels.  The corresponding
masses are shown in Figure \ref{fig:masshistograms} assuming the dust
temperature is equal to the source's peak line brightness temperature (Section
\ref{sec:temperature}).


Except where noted below, the hand-selected sources are used for further
analysis as they are more reliable.
%However, to encourage reproducible results
%- which hand-extracted source positions defy - we also provide the
%algorithmically-extracted and selected dendrogram catalog.

\begin{table*}[htp]
\caption{Continuum Source IDs and photometry Part 1}
\begin{tabular}{lllllllllllllllllllllllllllllllllllllllllllllllllllllllllllllllllllll}
\label{tab:photometry1}
Source ID & RA & Dec & $S_{\nu}(0.2\arcsec)$ & $S_{\nu}(0.4\arcsec)$ & $T_{B,max}$ & M$(T_B, 0.2\arcsec)$ & M$(T_B, \mathrm{peak})$ & Categories \\
 &  &  & $\mathrm{mJy}$ & $\mathrm{mJy}$ & $\mathrm{K}$ & $\mathrm{M_{\odot}}$ & $\mathrm{M_{\odot}}$ &  \\
\hline
ALMAmm1 & 19:23:42.864 & 14:30:07.92 & 3.7 & 6.9 & 11 & 2.6 & 2.6 & fCc \\
ALMAmm2 & 19:23:42.394 & 14:30:07.86 & 4.2 & 7.3 & 4 & 3 & 12 & fCc \\
ALMAmm3 & 19:23:42.398 & 14:30:06.08 & 4.2 & 11 & nan & 3 & 2.9 & f-- \\
ALMAmm4 & 19:23:42.614 & 14:30:02.14 & 7.7 & 16 & 11 & 5.4 & 6.1 & -Cc \\
ALMAmm5 & 19:23:42.658 & 14:30:03.63 & 1 & 19 & 5.9 & 7.3 & 8.8 & -Cc \\
ALMAmm6 & 19:23:42.758 & 14:30:04.97 & 2.8 & 9.2 & 3.8 & 2 & 1.1 & fC- \\
ALMAmm7 & 19:23:40.702 & 14:30:24.5 & 3.5 & 7.4 & 1.4 & 2.5 & 7.1 & -Cc \\
ALMAmm9 & 19:23:41.481 & 14:30:14.6 & 21 & 46 & 5.8 & 15 & 9 & -Cc \\
ALMAmm10 & 19:23:38.738 & 14:30:47.66 & 3.6 & 7.8 & 5.3 & 2.6 & 1 & -Cc \\
ALMAmm11 & 19:23:38.684 & 14:30:45.57 & 19 & 36 & 12 & 14 & 12 & -Cc \\
ALMAmm12 & 19:23:38.755 & 14:30:45.54 & 5.2 & 2 & 11 & 3.7 & 11 & -C- \\
ALMAmm13 & 19:23:38.825 & 14:30:40.31 & 7 & 12 & 11 & 5 & 8.6 & -Cc \\
ALMAmm14 & 19:23:38.57 & 14:30:41.79 & 67 & 14 & 36 & 23 & 23 & --c \\
ALMAmm15 & 19:23:38.486 & 14:30:40.86 & 14 & 31 & 35 & 4.9 & 8.1 & --c \\
ALMAmm16 & 19:23:38.2 & 14:31:06.85 & 23 & 45 & 5.6 & 16 & 32 & -Cc \\
ALMAmm17 & 19:23:42.214 & 14:30:54.31 & 15 & 26 & 12 & 11 & 5 & fCc \\
ALMAmm18 & 19:23:42.293 & 14:30:55.29 & 15 & 31 & 5.6 & 11 & 4.1 & -Cc \\
ALMAmm19 & 19:23:42.307 & 14:30:56.49 & 4.9 & 13 & 21 & 3.2 & 1.4 & --- \\
ALMAmm20 & 19:23:41.64 & 14:31:01.75 & 8 & 21 & 6.6 & 5.7 & 4 & -C- \\
ALMAmm21 & 19:23:41.981 & 14:31:10.52 & 8.9 & 15 & nan & 6.3 & 31 & --c \\
ALMAmm22 & 19:23:41.909 & 14:31:11.38 & 8.6 & 23 & nan & 6.1 & 18 & --- \\
ALMAmm23 & 19:23:40.496 & 14:31:03.94 & 22 & 65 & 26 & 11 & 5.2 & --- \\
ALMAmm24 & 19:23:39.953 & 14:31:05.35 & 29 & 6 & 72 & 46 & 32 & -Hc \\
ALMAmm25 & 19:23:42.132 & 14:30:40.57 & 18 & 38 & 3.8 & 13 & 6 & fCc \\
ALMAmm26 & 19:23:43.102 & 14:30:53.66 & 13 & 32 & 8.9 & 9.5 & 14 & -Cc \\
ALMAmm27 & 19:23:42.967 & 14:30:56.18 & 1 & 27 & 5 & 7.2 & 8.2 & fC- \\
ALMAmm28 & 19:23:43.68 & 14:30:32.24 & 16 & 33 & 19 & 11 & 6.5 & -Cc \\
ALMAmm29 & 19:23:41.933 & 14:30:30.45 & 9 & 16 & nan & 6.4 & 11 & f-c \\
ALMAmm30 & 19:23:43.164 & 14:30:54.12 & 8.3 & 2 & 4.3 & 5.9 & 2 & fCc \\
ALMAmm31 & 19:23:39.754 & 14:31:05.24 & 17 & 36 & 46 & 43 & 16 & --c \\
ALMAmm32 & 19:23:39.724 & 14:31:05.15 & 87 & 23 & 94 & 1 & 13 & -H- \\
ALMAmm33 & 19:23:39.828 & 14:31:05.23 & 19 & 51 & 48 & 48 & 18 & --- \\
ALMAmm34 & 19:23:39.878 & 14:31:05.19 & 8 & 24 & 9 & 1 & 1 & -H- \\
ALMAmm35 & 19:23:39.991 & 14:31:05.77 & 2 & 58 & 34 & 76 & 18 & --- \\
ALMAmm36 & 19:23:39.518 & 14:31:03.33 & 22 & 47 & 11 & 16 & 6.5 & -Cc \\
\hline
\end{tabular}

\end{table*}

\begin{table*}[htp]
\caption{Continuum Source IDs and photometry Part 2}
\begin{tabular}{lllllllllllllllllllllllllllllllllllllllllllllllllllllllllllllllllll}
\label{tab:photometry}
Source ID & RA & Dec & $S_{\nu}(0.2\arcsec)$ & $S_{\nu}(0.4\arcsec)$ & $T_{B,max}$ & M$(T_B, 0.2\arcsec)$ & M$(T_B, \mathrm{peak})$ & Categories \\
 & $\mathrm{{}^{\circ}}$ & $\mathrm{{}^{\circ}}$ &  &  & $\mathrm{K}$ & $\mathrm{M_{\odot}}$ & $\mathrm{M_{\odot}}$ &  \\
\hline
ALMAmm41 & 290.9327 & 14.5112 & 0.02 & 0.03 & 3 & 8.6 & 13 & --c \\
ALMAmm43 & 290.9150 & 14.5178 & 0.02 & 0.05 & 17 & 14 & 4.8 & fC- \\
ALMAmm44 & 290.9086 & 14.5182 & 0.01 & 0.02 & 5.7 & 6 & 28 & -Cc \\
ALMAmm45 & 290.9115 & 14.5187 & 0.01 & 0.02 & 4.4 & 5.2 & 1.8 & -C- \\
ALMAmm46 & 290.9243 & 14.5147 & 0.01 & 0.04 & 24 & 8.5 & 3 & --- \\
ALMAmm47 & 290.9274 & 14.5179 & 0.01 & 0.01 & 4.9 & 4.5 & 9.2 & -Cc \\
ALMAmm48 & 290.9287 & 14.5162 & 0.01 & 0.02 & 8.1 & 6.7 & 13 & -Cc \\
ALMAmm49 & 290.9300 & 14.5142 & 0.01 & 0.05 & 21 & 9.8 & 7.6 & --- \\
ALMAmm5 & 290.9277 & 14.5010 & 0.01 & 0.02 & 5.9 & 7.3 & 8.8 & -Cc \\
ALMAmm50 & 290.9301 & 14.5141 & 0.02 & 0.05 & 14 & 14 & 6.3 & -C- \\
ALMAmm51 & 290.9300 & 14.5139 & 0.02 & 0.05 & 16 & 13 & 2.7 & -C- \\
ALMAmm52 & 290.9117 & 14.5107 &  & 0.01 & 7.9 & 3.5 & 1 & -Cc \\
ALMAmm53 & 290.9119 & 14.5117 & 0.01 & 0.02 & 13 & 5.7 & 5.6 & -Cc \\
ALMAmm54 & 290.9122 & 14.5099 & 0.01 & 0.01 & 2.5 & 3.6 & 9.7 & -Cc \\
ALMAmm55 & 290.9309 & 14.5140 & 0.01 & 0.03 & 4.5 & 7.7 & 6.5 & -C- \\
ALMAmm56 & 290.9310 & 14.5143 & 0.01 & 0.03 & 5.3 & 6.4 & 5.4 & -C- \\
ALMAmm57 & 290.9239 & 14.5147 & 0.01 & 0.01 & 2 & 4 & 1.2 & fCc \\
ALMAmm6 & 290.9282 & 14.5014 &  & 0.01 & 3.8 & 2 & 1.1 & fC- \\
ALMAmm7 & 290.9196 & 14.5068 &  & 0.01 & 1.4 & 2.5 & 7.1 & -Cc \\
ALMAmm9 & 290.9228 & 14.5041 & 0.02 & 0.05 & 5.8 & 15 & 9 & -Cc \\
d2 & 290.9159 & 14.5180 & 0.16 & 0.43 & 99 & 18 & 15 & -H- \\
e1mm1 & 290.9327 & 14.5074 & 0.16 & 0.42 & 23 & 95 & 31 & --- \\
e2e & 290.9332 & 14.5096 & 0.69 & 1.9 & 84 & 94 & 61 & -H- \\
e2e peak & 290.9332 & 14.5096 & 0.74 & 1.8 & 1 & 8 & 68 & -Hc \\
e2nw & 290.9328 & 14.5100 & 0.22 & 0.51 & 4 & 69 & 33 & --c \\
e2se & 290.9337 & 14.5093 & 0.04 & 0.1 & 93 & 4.4 & 7.3 & -H- \\
e2w & 290.9330 & 14.5096 & 0.54 & 1.2 & 85 & 72 & 6 & fHc \\
e3mm1 & 290.9326 & 14.5069 & 0.05 & 0.18 & 23 & 29 & 9.4 & --- \\
e5 & 290.9244 & 14.5157 & 0.03 & 0.03 & nan & 18 & 34 & F-c \\
e8mm & 290.9329 & 14.5078 & 0.68 & 1.7 & 18 & 43 & 16 & -H- \\
eEmm1 & 290.9334 & 14.5070 & 0.04 & 0.1 & 39 & 12 & 7 & --- \\
eEmm2 & 290.9333 & 14.5071 & 0.04 & 0.1 & 22 & 24 & 6.5 & --- \\
eEmm3 & 290.9335 & 14.5075 & 0.04 & 0.1 & 22 & 27 & 8.1 & --c \\
eSmm1 & 290.9326 & 14.5065 & 0.07 & 0.18 & 34 & 26 & 2 & --- \\
eSmm2 & 290.9325 & 14.5062 & 0.07 & 0.17 & 29 & 31 & 23 & --c \\
eSmm2a & 290.9323 & 14.5062 & 0.05 & 0.13 & 24 & 28 & 18 & --- \\
eSmm3 & 290.9323 & 14.5059 & 0.05 & 0.1 & 29 & 23 & 16 & --c \\
eSmm4 & 290.9326 & 14.5059 & 0.04 & 0.1 & 27 & 17 & 1 & --- \\
eSmm6 & 290.9325 & 14.5055 & 0.05 & 0.11 & 23 & 29 & 15 & --c \\
north & 290.9169 & 14.5182 & 0.72 & 1.7 & 69 & 12 & 59 & -Hc \\
\hline
\end{tabular}
\par
The Categories column consists of three letter codes as described in Section \ref{sec:contsourcenature}.In column 1, \texttt{F} indicates a free-free dominated source,\texttt{f} indicates significant free-free contribution,and \texttt{-} means there is no detected cm continuum.In column 2, the peak brightness temperature is used toclassify the temperature category.\texttt{H} is `hot' ($T>50$ K), \texttt{C} is `cold' ($T<20$ K), and \texttt{-} is indeterminate (either $20<T<50$K or no measurement)In column 3, \texttt{c} indicates compact sources, and \texttt{-} indicates a diffuse source.
\end{table*}


% \subsubsection{Distribution Functions}
% \label{sec:distributionfunctions}
% We fit power law distributions to each aperture's flux distribution using the
% packages \texttt{plfit} and \texttt{PowerLaw}
% \citep[https://github.com/keflavich/plfit,
% https://github.com/jeffalstott/powerlaw;][]{Clauset2007a,Alstott2014a}.
% 
% The
% powerlaws steepen slightly from $\alpha=2.0\pm0.12$ to $\alpha=2.2\pm0.16$ for
% larger apertures.  The minimum flux density represented by a power law
% increases from $\sim20$ mJy for the peak flux density distribution to 0.4 Jy
% for the largest aperture (14-280 \msun at 20K).  These slopes are shallower
% than the Salpeter-like slope for the mass function derived by
% \citep{Konyves2015a} for their sample, though with only modest significance
% ($<3-\sigma$).  Of course, these measurements are of the continuum flux
% density, not directly of the mass, and so a direct comparison may not be
% appropriate.  We revisit this question after assessing the dust temperature in
% Section \ref{sec:temperature}.


%\Figure{figures/dendro_core_flux_histograms.png}
%{Histograms of the core flux densities measured with circular apertures centered
%on the dendrogram-extracted core centroids.  The aperture size is listed 
%in the y-axis label.  Free-free-dominated sources are excluded.}
%{fig:dendrofluxhistograms}
%{1}{16cm}

\Figure{figures/core_flux_histogram_apertureradius.png}
{Histograms of the core flux densities measured with circular apertures centered
on the hand-extracted core positions.  The aperture size is listed 
in the y-axis label.  For the top plot, labeled `Peak', this is the peak
intensity in Jy/beam.  For the rest, it is the integrated flux density
in the specified aperture.  The unfilled data show all sources and the hashed
data are for starless core candidates (Section \ref{sec:contsourcenature}).
See Figure \ref{fig:masshistograms} for the corresponding masses.}
{fig:fluxhistograms}
{1}{16cm}

% can probably argue safely that the 'background' is filtered out as much
% as we would want...
%For spectral extraction, we also measured a background spectrum averaging
%over an annulus with twice the radius of the source extraction.


%\subsection{The source flux density distribution \& the core mass function}
%\removed{maybe}


%Not used any more: the fitting approach is better, it just took longer to
%implement
% \subsection{Spectral Lines \& Velocities}
% \label{sec:losvelo}
% To determine the line-of-sight velocity of each source, we extracted a spectrum
% from an 0.5\arcsec aperture centered on the source and from a 0.5-1.0\arcsec
% annulus around it.  We then searched each spectrum for the brightest pixel and
% associated it with the likeliest spectral line.  We repeated this in each of
% our 4 spectral windows, then averaged the 4 velocities to get an estimate of
% the source velocity.   This process also allowed us to identify the brightest
% lines in each window and the brightest overall line observed, which we use
% later for temperature estimation.


% \subsection{Dense Gas Kinematics}
% \label{sec:kinematics}
% We examine the gas kinematics throughout the cloud, but especially near the 
% massive cores.
% 
% The ambient cloud, which consists of gas that has not yet condensed into
% compact prestellar objects, is evident in absorption against the mm cores at
% 55-58 \kms (toward e2) and 57-60 \kms (toward e8).  Narrower velocity
% components related to the known high-velocity streams are detected around 68
% \kms toward both sources.  These absorption features are seen in all 
% \formaldehyde transitions, \methanol $4_{2,2}-3_{1,2}$, OCS 18-17,
% but it was less obvious or absent in HNCO $10_{1,10}-9_{1,9}$ and OCS 19-18.
% 
% In the material surrounding the e2 and e8 ``cores'', one particularly notable
% feature is that the cores themselves show a redshifted centroid velocity
% relative to their surroundings in nearly all of the bright lines (H2CO, OCS,
% SO, \ceighteeno).  The observed shift is up to $\lesssim2$ \kms.  The shift is
% a sign of infall.  Given the high continuum brightness, the cores are likely
% optically thick in the continuum (Section \ref{sec:} XXX), therefore obscuring
% all molecular emission behind them.  We are seeing only gas in the
% foreground, and this gas is clearly moving toward the cores.
% 
% The source ALMAmm14 shows a similar kinematic signature....


\subsection{Temperature estimation of the continuum sources}
\label{sec:temperature}
The temperature is a critical ingredient for determining the total mass of each
continuum source or region. Since we do not have any means of directly
determining the dust temperature, as the SED peak is well into the THz regime
and inaccessible to any existing instruments at the requisite resolution, we
employ alternative indicators.  Above a density $n\gtrsim10^5-10^6$ \percc,
the gas and dust become strongly collisionally coupled, meaning the gas
temperature should accurately reflect the dust temperature.  Below this density,
the two may be decoupled.

The average dust temperature, as estimated from Herschel Hi-Gal SED fits
\citep{Molinari2016a,Wang2015a}, is 38 K when including the 70 \um data or 26 K
when excluding it.  This average is obtained over a $\sim45\arcsec$ ($\sim 1$
pc) beam and therefore is likely to be strongly biased toward the hottest dust
in the \hii regions.  Despite these
uncertainties, this bulk measurement provides us with a reasonable range to
assume for the uncoupled, low-density dust, which (weakly) dominates the mass
(see Section \ref{sec:massbudget}).

One constraint on the dust temperature we can employ is the absolute surface
brightness.  For some regions, especially the e8 filament and the hot cores,
%noted in Section \ref{sec:morphology},
the surface brightness is substantially
brighter than is possible for a beam-filling, optically thick blackbody at 20
K, providing a lower limit on the dust temperature ranging from 20 K (35
mJy/beam) to $\sim300$ K (0.5 Jy/beam).  Toward most of this emission, optically-thick
free-free emission can be strongly ruled out as the driving mechanism: 
existing data limits the free-free contribution to be $<50\%$ if it is
optically thick and negligible ($\ll1\%$) if it is optically thin at radio
wavelengths \citep{Ginsburg2016b, Goddi2016a}.
%(Ginsburg et al, 2016; Goddi2016a).

To gain a more detailed measurement of the dust temperature in regions where it
is likely to be coupled to the gas, we use the peak brightness temperature
$T_{B,max}$ of spectral lines along the line of sight.  If the observed
molecule is in local thermal equilibrium, as is expected if the density is high
enough to be collisionally coupled to the dust, and it is optically thick, the
brightness temperature provides an approximate measurement of the local
temperature near the $\tau=1$ surface.  If any of these assumptions do not
hold, $T_{B,max}$ will set a lower limit on the true gas temperature.  Only 
two mechanisms can push $T_{B,max} > T_{dust}$: nonthermal (maser) emission, which
is not known for any of the observed lines nor expected given the reasonable
$T_B$ observed, or a dust-emitting region that has a smaller beam filling
factor than the gas-emitting region, which is unlikely when the dust emission
structure is resolved, as is the case toward most sources.

One potential problem with this approach is whether the gas becomes optically thick
before probing most of the dust, in which case spectral line self-absorption
will occur.  Some transitions of more abundant molecules, e.g., CO and
\formaldehyde, are likely to be affected by this issue.  However, many of the
molecules included in the observations (Tables
\ref{tab:linesspw0}-\ref{tab:linesspw3}) have lower abundances, especially in
lower-density gas, and are likely to be optically thin along most of the lines
of sight.

Some sources have no detected line emission aside from the molecular cloud
species CO and \formaldehyde, which are also associated with more diffuse gas
and not isolated to the compact continuum sources but in these cases peak
locally on the compact source.  The minimum density requirement imposed by a
continuum detection at our limit of 1.6 mJy is $n>10^{7.5}$ \percc for a
spherical source.  At such high density, it is unlikely that the species are
undetected because they are subthermally excited.  More likely, the
line-nondetection sources have an underyling emission source that is very
compact, optically thick, and/or cold.

Figure \ref{fig:peaktbhist} shows the distribution of peak line brightnesses
for the continuum sources.  The spectra used to determine this brightness are
the spectra obtained from the brightest continuum pixel within the source
aperture.  To obtain the peak line brightness, we fit Gaussian profiles to each
identified line listed in Tables \ref{tab:linesspw0}-\ref{tab:linesspw3},
rejecting those with poor fits.  The line brightnesses reported in the figure
are the sum of the continuum-subtracted peak line brightness and the continuum
brightness (i.e., they are the raw observed peak brightness).  Excepting CO and
\formaldehyde, which are excluded from the plot, \methanol is the brightest
line toward most sources. 

% core_table_plots
% from merge_spectral_fits_with_photometry
\Figure{figures/brightest_line_histogram.png}
{Histogram of the brightest line toward each continuum source.
The bars are colored by the molecular species associated with the brightest
line that is not associated with extended molecular cloud emission,
i.e., CO and its isotopologues and \formaldehyde are excluded.}
{fig:peaktbhist}{1}{10cm}

We use these peak line brightness temperatures to compute the masses of the
continuum sources.  For sources with $T_{B,max} < 20$ K, we assume $T_{dust} =
20$ K to avoid producing unreasonably high masses; in such sources the lines
are likely to be optically thin and/or subthermally excited.  This correction is
illustrated in Figure \ref{fig:moftbvsm20k}.

\Figure{figures/aperture_mass20K_vs_massTB.png}
{The mass computed assuming the dust temperature is the peak brightness
temperature vs. that computed assuming $T_{dust}=20$ K  for the aperture extracted
continuum sources.
%The faded sources in the top left of the diagram are those with $M(T_{B,max}) >
%M(20\textrm{K})$; the circles along the dashed line show their $M(20\mathrm{K})$.
The dashed line shows $M(T_{B,max}) =
M(20\textrm{K})$ and the dotted line shows $M(T_{B,max}) = 0.1 
M(20\textrm{K})$ 
}{fig:moftbvsm20k}{1}{10cm}

This section has provided some simple temperature estimates across all of the
detected continuum sources.  In Section \ref{sec:ch3ohtem}, we  examine
the thermal structure of the hot cores in more detail.

%\Figure{figures/dendro_peakTB_vs_selfconsistentcontinuum.png}
%{The peak line brightness vs the continuum brightness of our target sources
%within $\sim1\arcsec$ beams.  The points are color-coded by the brightest
%observed line.  The dashed line shows where the brightness temperature of the
%continuum matches that of the lines; technically this means that it should be
%impossible for any point to be below the line.  
%%However, in this
%%iteration of the plot, the lines are extracted from broader apertures than the
%%continuum, so the continuum peak brightness is capable of being higher than the
%%line peak brightness.  DONE:[removed] replace this plot once spectra have been
%%appropriately extracted from the high-resolution spectral data cube.
%}
%{fig:peaktb}{1}{10cm}


\Figure{figures/core_mass_histogram_apertureradius.png}
{Histograms of the core masses computed from the flux density measurements
shown in Figure \ref{fig:fluxhistograms} using the peak brightness temperature 
toward the center of that source as the dust temperature.
The aperture size is listed in the y-axis label.  For the top plot, labeled
`Peak', the mass is computed from peak
intensity in Jy/beam.  For the rest, it is the integrated flux density in
the specified aperture in Jy.  The unfilled data show all sources and the
hashed
data are for starless core candidates (Section \ref{sec:contsourcenature}).}
{fig:masshistograms}
{1}{16cm}

\subsection{The nature of the continuum sources}
\label{sec:contsourcenature}
Millimeter continuum sources in star-forming regions are usually assumed to be
either protostars or starless cores.  However, in this high-mass star-forming
region, we have to consider not only those possibilities but also potential
free-free sources and high-luminosity main-sequence stars embedded in dust.

To distinguish these possibilities, we measure both the spectral lines and
features of the continuum emission toward the compact continuum sources.  Main
sequence OB stars and their illuminated ionized nebulae are in principle easily
identified by their free-free emission, so we use centimeter continuum and
radio recombination line emission to identify these sources.  Starless cores,
protostellar cores, and their variants are more difficult to identify, so we
used a combination of gas temperature and continuum concentration parameter to
classify them.

To estimate the gas temperature toward the compact sources, we fit each of up
to $\sim50$ lines (see Tables \ref{tab:linesspw0}-\ref{tab:linesspw3}) with
Gaussian profiles to attempt to determine the relative line strengths toward
each source.  Most sources were detected in at least $\sim5-10$ lines, though
some of these are associated with interstellar rather than circumstellar
material, i.e., \formaldehyde, CO, $^{13}$CS.  For sources with detections in
non-interstellar lines, we used the peak brightness temperature of the line as
an estimated lower limit on the core temperature.

In the continuum, we measured a `concentration parameter', which is the ratio
of the flux density in a 0.2\arcsec aperture to that in a 0.2\arcsec-0.4\arcsec
annulus divided by three to account for the annulus' larger area.  A uniform
source with $r>0.4\arcsec$ would have a concentration $C=1$ by this
definition, while an unresolved point source would have a Gaussian profile
resulting in $C=14$.  Only one source approaches this extreme, the HII region
e5, while the rest have $C\leq7$.  We set the threshold for a `concentrated'
source to be $C>2$, which is arbitrary, but does a reasonable job of
distinguishing the sources with a clear central concentration from those that
have none.


We classified each of the 75 hand-selected sources on the following parameters:
\begin{enumerate}
    \item Free-free dominated sources ($S_{15 GHz} > 0.5 S_{227 GHz}$) are \hii
        regions
    \item Free-free contaminated sources ($S_{15 GHz} > 0.1 S_{227 GHz}$) are
        likely to be dust-dominated but with \hii region contamination; these
        are either dusty sources superposed on or embedded in a large \hii
        region or they are compact, dusty \hii regions
    \item Starless core candidates were identified as those with cold peak
        brightness temperatures $T_B < 20$ K and with a high concentration
        parameter ($C>2$)
    \item Hot core candidates are those with peak $T_B>50$ K and $C>2$
    \item Extended cold core and hot core candidates are those with $T_B<20$ K
        and $T_B>50$ K and $C<2$, respectively.
    \item The remaining sources with $S_{15 GHz} < 0.1 S_{227 GHz}$ and $50 >
        T_B > 20$ K were classed as uncertain compact ($C>2$) or uncertain extended
        ($C<2$)
\end{enumerate}

These classifications are set in the `Categories' column of Table
\ref{tab:photometry1}.  They serve as a broad guideline for further
analysis.   


% \section{Intersection of cylindrical and spherical annuli in a power-law
% density profile}


\end{document}
